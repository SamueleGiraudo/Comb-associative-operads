%%%%%%%%%%%%%%%%%%%%%%%%%%%%%%%%%%%%%%%%%%%%%%%%%%%%%%%%%%%%%%%%%%%%%%%%
%%%%%%%%%%%%%%%%%%%%%%%%%%%%%%%%%%%%%%%%%%%%%%%%%%%%%%%%%%%%%%%%%%%%%%%%
%%%%%%%%%%%%%%%%%%%%%%%%%%%%%%%%%%%%%%%%%%%%%%%%%%%%%%%%%%%%%%%%%%%%%%%%
\section{The magmatic operad, quotients, and rewrite relations}
\label{sec:operad_Mag}
We set in this preliminary section our notations about operads. We also
provide a definition for the magmatic operad and introduce tools to
handle with some of its quotients involving rewrite systems on syntax
trees.
\medbreak

%%%%%%%%%%%%%%%%%%%%%%%%%%%%%%%%%%%%%%%%%%%%%%%%%%%%%%%%%%%%%%%%%%%%%%%%
%%%%%%%%%%%%%%%%%%%%%%%%%%%%%%%%%%%%%%%%%%%%%%%%%%%%%%%%%%%%%%%%%%%%%%%%
\subsection{Nonsymmetric operads}
A \Def{nonsymmetric operad in the category of sets}, or a
\Def{nonsymmetric operad} for short, is a graded set
\begin{equation}
    \Oca = \bigsqcup_{n \geq 1} \Oca(n)
\end{equation}
together with maps
\begin{equation}
    \circ_i : \Oca(n) \times \Oca(m) \to \Oca(n + m - 1),
    \qquad 1 \leq i \leq n, 1 \leq m,
\end{equation}
called \Def{partial compositions}, and a distinguished element
$\Unit \in \Oca(1)$, the \Def{unit} of $\Oca$. This data has to satisfy,
for any $x, y, z \in \Oca$, the three relations
\begin{subequations}
\begin{equation} \label{equ:operad_axiom_1}
    (x \circ_i y) \circ_{i + j - 1} z = x \circ_i (y \circ_j z),
    \qquad
    1 \leq i \leq |x|, 1 \leq j \leq |y|,
\end{equation}
\begin{equation} \label{equ:operad_axiom_2}
    (x \circ_i y) \circ_{j + |y| - 1} z = (x \circ_j z) \circ_i y,
    \qquad
    1 \leq i < j \leq |x|,
\end{equation}
\begin{equation} \label{equ:operad_axiom_3}
    \Unit \circ_1 x = x = x \circ_i \Unit,
    \qquad 1 \leq i \leq |x|.
\end{equation}
\end{subequations}
Since we consider in this work only nonsymmetric operads, we shall call
these simply \Def{operads}.
\medbreak

Let us provide some elementary definitions and notations about operads.
If $x$ is an element of $\Oca$ such that $x \in \Oca(n)$ for an
$n \geq 1$, the \Def{arity} $|x|$ of $x$ is $n$. If $\Oca_1$ and
$\Oca_2$ are two operads, a map $\phi : \Oca_1 \to \Oca_2$ is an
\Def{operad morphism} is it respects the arities, sends the unit of
$\Oca_1$ to the unit of $\Oca_2$, and commutes with partial composition
maps. We say that $\Oca_2$ is a \Def{suboperad} of $\Oca_1$ if $\Oca_2$
is a subset of $\Oca_1$ containing the unit of $\Oca_1$ and the partial
composition maps of $\Oca_2$ are the ones of $\Oca_1$ restricted on
$\Oca_2$. For any subset $\GeneratingSet$ of $\Oca$, the
\Def{operad generated} by $\GeneratingSet$ is the smallest suboperad
$\Oca^\GeneratingSet$ of $\Oca$ containing $\GeneratingSet$. When
$\Oca = \Oca^\GeneratingSet$, we say that $\GeneratingSet$ is a
\Def{generating set} of $\Oca$. In this case, and when additionally
$\GeneratingSet$ is minimal among all the subsets of $\Oca$ satisfying
this property, $\GeneratingSet$ is a \Def{minimal} generating set of
$\Oca$. An \Def{operad congruence} is an equivalence relation
$\Congr$ on $\Oca$ such that $\Congr$ respects the arities, and for any
$x, x', y, y' \in \Oca$ such that $x \Congr x'$ and $y \Congr y'$,
$x \circ_i y$ is $\Congr$-equivalent to $x' \circ_i y'$ for any valid
integer $i$. Given an operad congruence $\Congr$, the
\Def{quotient operad} $\Oca/_{\Congr}$ of $\Oca$ by $\Congr$ is the
operad of all $\Congr$-equivalence classes endowed with the partial
composition maps defined in the obvious way. In the case where all the
sets $\Oca(n)$, $n \geq 1$, are finite, the \Def{Hilbert series}
$\HilbertSeries_\Oca(t)$ of $\Oca$ is the series defined by
\begin{equation}
    \HilbertSeries_\Oca(t) := \sum_{n \geq 1} \# \Oca(n) t^n.
\end{equation}
\medbreak

%%%%%%%%%%%%%%%%%%%%%%%%%%%%%%%%%%%%%%%%%%%%%%%%%%%%%%%%%%%%%%%%%%%%%%%%
%%%%%%%%%%%%%%%%%%%%%%%%%%%%%%%%%%%%%%%%%%%%%%%%%%%%%%%%%%%%%%%%%%%%%%%%
\subsection{Binary trees and magmatic operad}
A \Def{binary tree} is either the \Def{leaf} $\Leaf$ or a pair
$(\Tfr_1, \Tfr_2)$ of binary trees. We use the standard terminology
about binary trees (such as \Def{root}, \Def{internal node},
\Def{child}, {\em etc.}) in this work. Let us recall the main notions.
The \Def{arity} $|\Tfr|$ (resp. \Def{degree} $\Deg(\Tfr)$) of a binary
tree $\Tfr$ is its number of leaves (resp. internal nodes). We shall
draw binary trees the root to the top. For instance,
\begin{equation}
    \begin{tikzpicture}[xscale=.15,yscale=.18,Centering]
        \node(0)at(0.00,-4.50){};
        \node(2)at(2.00,-6.75){};
        \node(4)at(4.00,-6.75){};
        \node(6)at(6.00,-4.50){};
        \node(8)at(8.00,-4.50){};
        \node[NodeST](1)at(1.00,-2.25){\begin{math}\Product\end{math}};
        \node[NodeST](3)at(3.00,-4.50){\begin{math}\Product\end{math}};
        \node[NodeST](5)at(5.00,0.00){\begin{math}\Product\end{math}};
        \node[NodeST](7)at(7.00,-2.25){\begin{math}\Product\end{math}};
        \draw[Edge](0)--(1);
        \draw[Edge](1)--(5);
        \draw[Edge](2)--(3);
        \draw[Edge](3)--(1);
        \draw[Edge](4)--(3);
        \draw[Edge](6)--(7);
        \draw[Edge](7)--(5);
        \draw[Edge](8)--(7);
        \node(r)at(5.00,2){};
        \draw[Edge](r)--(5);
    \end{tikzpicture}
\end{equation}
is the graphical representation of the binary tree
\begin{math}
    ((\Leaf, (\Leaf, \Leaf)), (\Leaf, \Leaf)).
\end{math}
\medbreak

The \Def{magmatic operad} $\Mag$ is the graded set of all binary trees
where $\Mag(n)$, $n \geq 1$, is the set of all binary trees of arity
$n$. The partial composition maps of $\Mag$ are grafting of trees: given
two binary trees $\Tfr$ and $\Sfr$, $\Tfr \circ_i \Sfr$ is the binary
tree obtained by grafting the root of $\Sfr$ onto the $i$th leaf
(numbered from left to right) of $\Tfr$. For instance,
\begin{equation}
    \begin{tikzpicture}[xscale=.15,yscale=.18,Centering]
        \node(0)at(0.00,-4.50){};
        \node(2)at(2.00,-6.75){};
        \node(4)at(4.00,-6.75){};
        \node(6)at(6.00,-4.50){};
        \node(8)at(8.00,-4.50){};
        \node[NodeST](1)at(1.00,-2.25){\begin{math}\Product\end{math}};
        \node[NodeST](3)at(3.00,-4.50){\begin{math}\Product\end{math}};
        \node[NodeST](5)at(5.00,0.00){\begin{math}\Product\end{math}};
        \node[NodeST](7)at(7.00,-2.25){\begin{math}\Product\end{math}};
        \draw[Edge](0)--(1);
        \draw[Edge](1)--(5);
        \draw[Edge](2)--(3);
        \draw[Edge](3)--(1);
        \draw[Edge](4)--(3);
        \draw[Edge](6)--(7);
        \draw[Edge](7)--(5);
        \draw[Edge](8)--(7);
        \node(r)at(5.00,2){};
        \draw[Edge](r)--(5);
    \end{tikzpicture}
    \enspace \circ_4 \enspace
    \begin{tikzpicture}[xscale=.17,yscale=.18,Centering]
        \node(0)at(0.00,-4.67){};
        \node(2)at(2.00,-4.67){};
        \node(4)at(4.00,-4.67){};
        \node(6)at(6.00,-4.67){};
        \node[NodeST](1)at(1.00,-2.33){\begin{math}\Product\end{math}};
        \node[NodeST](3)at(3.00,0.00){\begin{math}\Product\end{math}};
        \node[NodeST](5)at(5.00,-2.33){\begin{math}\Product\end{math}};
        \draw[Edge](0)--(1);
        \draw[Edge](1)--(3);
        \draw[Edge](2)--(1);
        \draw[Edge](4)--(5);
        \draw[Edge](5)--(3);
        \draw[Edge](6)--(5);
        \node(r)at(3.00,2){};
        \draw[Edge](r)--(3);
    \end{tikzpicture}
    \enspace = \enspace
    \begin{tikzpicture}[xscale=.15,yscale=.14,Centering]
        \node(0)at(0.00,-6.00){};
        \node(10)at(10.00,-12.00){};
        \node(12)at(12.00,-12.00){};
        \node(14)at(14.00,-6.00){};
        \node(2)at(2.00,-9.00){};
        \node(4)at(4.00,-9.00){};
        \node(6)at(6.00,-12.00){};
        \node(8)at(8.00,-12.00){};
        \node[NodeST](1)at(1.00,-3.00){\begin{math}\Product\end{math}};
        \node[NodeST](11)at(11.00,-9.00){\begin{math}\Product\end{math}};
        \node[NodeST](13)at(13.00,-3.00){\begin{math}\Product\end{math}};
        \node[NodeST](3)at(3.00,-6.00){\begin{math}\Product\end{math}};
        \node[NodeST](5)at(5.00,0.00){\begin{math}\Product\end{math}};
        \node[NodeST](7)at(7.00,-9.00){\begin{math}\Product\end{math}};
        \node[NodeST](9)at(9.00,-6.00){\begin{math}\Product\end{math}};
        \draw[Edge](0)--(1);
        \draw[Edge](1)--(5);
        \draw[Edge](10)--(11);
        \draw[Edge](11)--(9);
        \draw[Edge](12)--(11);
        \draw[Edge](13)--(5);
        \draw[Edge](14)--(13);
        \draw[Edge](2)--(3);
        \draw[Edge](3)--(1);
        \draw[Edge](4)--(3);
        \draw[Edge](6)--(7);
        \draw[Edge](7)--(9);
        \draw[Edge](8)--(7);
        \draw[Edge](9)--(13);
        \node(r)at(5.00,2.5){};
        \draw[Edge](r)--(5);
    \end{tikzpicture}
\end{equation}
is a partial composition in $\Mag$. The number of binary of arity
$n \geq 1$ is the $n$th Catalan number $\Catalan(n)$ and hence, the
Hilbert series of $\Mag$ is
\begin{equation}
    \HilbertSeries_{\Mag}(t)
    = \sum_{n \geq 1} \Catalan(n) t^n
    = \sum_{n \geq 1} \binom{2n - 1}{n - 1} \frac{1}{n} t^n.
\end{equation}
\medbreak

The operad $\Mag$ is the free operad generated by one binary element. It
satisfies the following universality property. Let
$\GeneratingSet := \GeneratingSet(2) := \{\Product\}$
be the graded set containing exactly one element $\Product$ of arity
$2$. For any operad $\Oca$ and any map
$f : \GeneratingSet(2) \to \Oca(2)$, there exists a unique operad
morphism $\phi : \Mag \to \Oca$ such that $f = \phi \circ \Corolla$,
where $\Corolla$ is the map sending $\Product$ to the unique binary tree
of degree $1$. In other terms, the diagram
\begin{equation}
    \begin{tikzpicture}[xscale=1.4,yscale=1.2,Centering]
        \node(G)at(0,0){\begin{math}\GeneratingSet\end{math}};
        \node(O)at(2,0){\begin{math}\Oca\end{math}};
        \node(AG)at(0,-2){\begin{math}\Mag\end{math}};
        \draw[Map](G)--(O)node[midway,above]{\begin{math}f\end{math}};
        \draw[Injection](G)--(AG)node[midway,left]
            {\begin{math}\Corolla\end{math}};
        \draw[Map,dashed](AG)--(O)node[midway,right]
            {\begin{math}\phi\end{math}};
    \end{tikzpicture}
\end{equation}
commutes.
\medbreak

\Todo{TO DO}

Given a binary tree $\Tfr$, we denote by $ \PrefixWord(\Tfr)$ the
\Def{prefix word} of $\Tfr$, that is the word on $\{\Zero, \Two\}$
obtained by a left to right depth-first traversal of $\Tfr$ and by
writing $\Zero$ (resp. $\Two$) when a leaf (resp. an internal node) is
encountered. The set of all words on $\{\Zero, \Two\}$ is endowed with
the lexicographic order $\leq$ induced by $\Zero < \Two$.
\medbreak

%%%%%%%%%%%%%%%%%%%%%%%%%%%%%%%%%%%%%%%%%%%%%%%%%%%%%%%%%%%%%%%%%%%%%%%%
%%%%%%%%%%%%%%%%%%%%%%%%%%%%%%%%%%%%%%%%%%%%%%%%%%%%%%%%%%%%%%%%%%%%%%%%
\subsection{Rewrite systems and presentations}


\Todo{TO DO}


If $\Rew$ is a rewrite rule on $\Mag$ such that $\Sfr \Rew \Sfr'$
implies $|\Sfr| = |\Sfr'|$, we denote by $\RewContext$ the
\Def{rewrite relation induced} by $\Rew$. Formally we have
\begin{math}
    \Tfr\circ_i\left(\Sfr\circ\left[\Rfr_1,\dots,\Rfr_{n}\right]\right)
    \RewContext
    \Tfr\circ_i\left(\Sfr'\circ\left[\Rfr_1,\dots,\Rfr_{n}\right]\right)
\end{math},
if $\Sfr \Rew \Sfr'$ where $n = |\Sfr|$, and $\Tfr$, $\Rfr_1$, \dots,
$\Rfr_n$ are binary trees. In other words, one has
$\Tfr \RewContext \Tfr'$ if it is possible to obtain $\Tfr'$ from $\Tfr$
by replacing a subtree $\Sfr$ of $\Tfr$ by $\Sfr'$ whenever
$\Sfr \Rew \Sfr'$. We use here the standard terminology
(\Def{terminating}, \Def{confluent}, \Def{convergent}, \Def{branching
pair}, \Def{joinable}, \Def{normal form}, {\em etc.}) about rewrite
relations and rewrite systems~\cite{BN98}.
\medbreak

Given an operad $\Oca \simeq \Mag/_{\Congr}$ where $\Congr$ is an operad
congruence of $\Mag$, we denote by $[\Tfr]_{\Congr}$ the
$\Congr$-equivalence class of any $\Tfr \in \Mag$. Besides, we say that
$\Rew$ is an \Def{orientation} of $\Congr$ if the reflexive, transitive,
and symmetric closure of $\RewContext$ is $\Congr$. We say that $\Rew$
is a \Def{convergent orientation} if $\RewContext$ is convergent. When
$\Rew$ is a convergent orientation of $\Congr$, the set of all normal
forms of $\RewContext$ is a \Def{Poincaré-Birkhoff-Witt basis} of the
operad $\Oca$ and its elements are exactly the binary trees avoiding,
as subtrees, the trees appearing as left members in~$\Rew$.
\medbreak

We shall use the following criterion to prove that a rewrite relation on
$\Mag$ is terminating.
\medbreak

\begin{Lemma}\label{lem:prefix_word_termination}
    Let $\Rew$ be a rewrite rule on $\Mag$. If for any
    $\Tfr, \Tfr' \in \Mag$ such that $\Tfr \Rew \Tfr'$ one has
    $\PrefixWord(\Tfr) > \PrefixWord(\Tfr')$, then the rewrite relation
    induced by $\Rew$ is terminating.
\end{Lemma}
\medbreak

Moreover, we shall use the following result appearing in~\cite{Gir16}
specialized on rewrite relation on $\Mag$ to prove that a terminating
rewrite relation is convergent.
\medbreak

\begin{Lemma} \label{lem:degree_confluence}
    Let $\Rew$ be a rewrite rule on $\Mag$ wherein all trees $\Tfr$ and
    $\Tfr'$ such that $\Tfr \Rew \Tfr'$ have degrees at most $\ell$.
    Then, if the rewrite relation $\RewContext$ induced by $\Rew$ is
    terminating and all its branching pairs of degrees at most
    $2\ell - 1$ are joinable, $\RewContext$ is convergent.
\end{Lemma}
\medbreak
