% Author: Cyrille Chenavier, Christophe Cordero, Samuele Giraudo
% Creation: may. 2018
% Modifications: may. 2018

\documentclass[10pt,reqno]{amsart}

%%%%%%%%%%%%%%%%%%%%%%%%%%%%%%%%%%%%%%%%%%%%%%%%%%%%%%%%%%%%%%%%%%%%%%%%
%%%%%%%%%%%%%%%%%%%%%%%%%%%%%%%%%%%%%%%%%%%%%%%%%%%%%%%%%%%%%%%%%%%%%%%%
%%%%%%%%%%%%%%%%%%%%%%%%%%%%%%%%%%%%%%%%%%%%%%%%%%%%%%%%%%%%%%%%%%%%%%%%
\usepackage[utf8x]{inputenc}
\usepackage[english]{babel}
\usepackage{amsmath,amsfonts,amssymb,amsthm,shuffle}
\usepackage[T1]{fontenc}
\usepackage[math]{anttor}

% Layout.
\usepackage[top=3.5cm,bottom=3.5cm,left=3.6cm,right=3.6cm]{geometry}

% Colors of hyperlinks.
\usepackage{xcolor}
% Author: Samuele Giraudo
% Creation: mar. 2018
% Modifications: mar. 2018

\definecolor{ColBlack}{RGB}{0,0,0} % Black.
\definecolor{ColWhite}{RGB}{255,255,255} % White.
\definecolor{Col1}{RGB}{133,6,6} % Rouge sang.
\definecolor{Col2}{RGB}{198,8,0} % Rouge ponceau.
\definecolor{Col3}{RGB}{174,74,52} % Rouge tomette.
\definecolor{Col4}{RGB}{103,113,121} % Gris de Payne.
\definecolor{Col5}{RGB}{90,94,107} % Ardoise.
\definecolor{Col6}{RGB}{70,63,50} % Taupe.

\newcommand{\ColBlack}[1]{\textcolor{ColBlack}{#1}}
\newcommand{\ColWhite}[1]{\textcolor{ColWhite}{#1}}
\newcommand{\ColA}[1]{\textcolor{Col1}{#1}}
\newcommand{\ColB}[1]{\textcolor{Col2}{#1}}
\newcommand{\ColC}[1]{\textcolor{Col3}{#1}}
\newcommand{\ColD}[1]{\textcolor{Col4}{#1}}
\newcommand{\ColE}[1]{\textcolor{Col5}{#1}}
\newcommand{\ColF}[1]{\textcolor{Col6}{#1}}


\usepackage[hyperindex=true,frenchlinks=true,colorlinks=true,
citecolor=Col2,linkcolor=Col3,urlcolor=Col4,linktocpage,
pagebackref=true]{hyperref}

% Tikz.
\usepackage{tikz}
\usetikzlibrary{shapes}
\usetikzlibrary{fit}
\usetikzlibrary{decorations.pathmorphing}

% Misc.
\usepackage{mathtools}
\usepackage{dsfont}
\usepackage{wasysym}
\usepackage{stmaryrd}
\usepackage{cite}
\usepackage{subfig}
\usepackage{multirow}
\usepackage{enumitem}
\usepackage{multicol}
\usepackage{cancel}

%%%%%%%%%%%%%%%%%%%%%%%%%%%%%%%%%%%%%%%%%%%%%%%%%%%%%%%%%%%%%%%%%%%%%%%%
%%%%%%%%%%%%%%%%%%%%%%%%%%%%%%%%%%%%%%%%%%%%%%%%%%%%%%%%%%%%%%%%%%%%%%%%
%%%%%%%%%%%%%%%%%%%%%%%%%%%%%%%%%%%%%%%%%%%%%%%%%%%%%%%%%%%%%%%%%%%%%%%%
% Line space.
\linespread{1.15}
\renewcommand{\arraystretch}{1.4}

% Vertical space for equations.
\setlength{\abovedisplayskip}{5pt}
\setlength{\belowdisplayskip}{5pt}

% Alphabetic footnote marks.
\renewcommand{\thefootnote}{\alph{footnote}}

% To allow cutting equations in several pages.
\allowdisplaybreaks

% Numbering of equations.
\numberwithin{equation}{subsection}

% Depth of the table of contents.
\setcounter{tocdepth}{2}

% Indentation in the table of contents.
\makeatletter
\def\l@section{\@tocline{1}{3pt}{1pc}{5pc}{}}
\def\l@subsection{\@tocline{2}{2pt}{2pc}{5pc}{}}
\makeatother

% Environments definitions.
\newtheorem{Theorem}{Theorem}[subsection]
\newtheorem{Proposition}[Theorem]{Proposition}
\newtheorem{Lemma}[Theorem]{Lemma}

% Better comparison symbols.
\renewcommand{\leq}{\leqslant}
\renewcommand{\geq}{\geqslant}

%%%%%%%%%%%%%%%%%%%%%%%%%%%%%%%%%%%%%%%%%%%%%%%%%%%%%%%%%%%%%%%%%%%%%%%%
%%%%%%%%%%%%%%%%%%%%%%%%%%%%%%%%%%%%%%%%%%%%%%%%%%%%%%%%%%%%%%%%%%%%%%%%
%%%%%%%%%%%%%%%%%%%%%%%%%%%%%%%%%%%%%%%%%%%%%%%%%%%%%%%%%%%%%%%%%%%%%%%%
\title{Comb associative operads}
\keywords{Operad; trees; Rewrite rule; Presentation; Enumeration.}
\subjclass[2010]{\Todo{}}
\date{\today}
\author{Cyrille Chenavier \and Christophe Cordero \and Samuele Giraudo}
\address{\scriptsize Université Paris-Est, LIGM (UMR $8049$), CNRS,
    ENPC, ESIEE Paris, UPEM, F-$77454$, Marne-la-Vallée, France}
\email{cyrille.chenavier@u-pem.fr}
\email{christophe.cordero@u-pem.fr}
\email{samuele.giraudo@u-pem.fr}

%%%%%%%%%%%%%%%%%%%%%%%%%%%%%%%%%%%%%%%%%%%%%%%%%%%%%%%%%%%%%%%%%%%%%%%%
%%%%%%%%%%%%%%%%%%%%%%%%%%%%%%%%%%%%%%%%%%%%%%%%%%%%%%%%%%%%%%%%%%%%%%%%
%%%%%%%%%%%%%%%%%%%%%%%%%%%%%%%%%%%%%%%%%%%%%%%%%%%%%%%%%%%%%%%%%%%%%%%%
% Author: Samuele Giraudo
% Creation: mar. 2018
% Modifications: mar. 2018

% Tools macros.
\newcommand{\Todo}[1]{\textcolor{Col1}{{\tt TODO: $[$#1$]$}}}
\newcommand{\Hide}[1]{\textcolor{Col4}{\tt [hidden]}}

% Format macros.
\newcommand{\Def}[1]{\textcolor{Col3}{\em #1}}
\newcommand{\OEIS}[1]{\href{http://oeis.org/#1}{{\bf #1}}}
\newcommand{\Arxiv}[1]{\href{https://arxiv.org/abs/#1}{\tt arXiv:#1}}

% Tikz for vertical centering.
\tikzstyle{Centering}=[{baseline={([yshift=-0.5ex]current
    bounding box.center)}}]



\newcommand{\N}{\mathbb{N}}
\newcommand{\Z}{\mathbb{Z}}
\newcommand{\Q}{\mathbb{Q}}
\newcommand{\K}{\mathbb{K}}

\newcommand{\Oca}{\mathcal{O}}
\newcommand{\Rfr}{\mathfrak{r}}
\newcommand{\Sfr}{\mathfrak{s}}
\newcommand{\Tfr}{\mathfrak{t}}

\newcommand{\Zero}{\mathtt{0}}
\newcommand{\Two}{\mathtt{2}}
\newcommand{\HilbertSeries}{\mathcal{H}}
\newcommand{\GeneratingSet}{\mathfrak{G}}
\newcommand{\FreeOperad}{\mathrm{F}}
\newcommand{\Mag}{\mathsf{Mag}}
\newcommand{\As}{\mathsf{As}}
\newcommand{\CAs}[1]{\mathsf{CAs}^{(#1)}}
\newcommand{\PrefixWord}{\mathrm{p}}
\newcommand{\Deg}{\mathrm{deg}}
\newcommand{\LC}[1]{\text{LC}^{(#1)}}
\newcommand{\RC}[1]{\text{RC}^{(#1)}}

\DeclareMathOperator{\Product}{\star}
\DeclareMathOperator{\Congr}{\equiv}
\DeclareMathOperator{\Rew}{\to}
\DeclareMathOperator{\RewRT}{\overset{*}{\Rew}}
\DeclareMathOperator{\RewT}{\overset{+}{\Rew}}
\DeclareMathOperator{\RewContext}{\Rightarrow}
\DeclareMathOperator{\RewContextRT}{\overset{*}{\RewContext}}
\DeclareMathOperator{\RewContextT}{\overset{+}{\RewContext}}

\newcommand{\CongrCAs}[1]{\Congr^{(#1)}}

% Tree drawings.
\tikzstyle{Node}=[circle,draw=Col1!80,fill=Col1!8,inner sep=1pt,
minimum size=2mm,thick,font=\scriptsize]
\tikzstyle{Edge}=[draw=Col2!80,cap=round,thick]
\tikzstyle{Leaf}=[rectangle,draw=ColBlack!70,fill=ColBlack!16,
inner sep=0pt,minimum size=1mm,thick]
\tikzstyle{NodeST}=[font=\footnotesize]

%%%%%%%%%%%%%%%%%%%%%%%%%%%%%%%%%%%%%%%%%%%%%%%%%%%%%%%%%%%%%%%%%%%%%%%%
%%%%%%%%%%%%%%%%%%%%%%%%%%%%%%%%%%%%%%%%%%%%%%%%%%%%%%%%%%%%%%%%%%%%%%%%
%%%%%%%%%%%%%%%%%%%%%%%%%%%%%%%%%%%%%%%%%%%%%%%%%%%%%%%%%%%%%%%%%%%%%%%%
\begin{document}

%%%%%%%%%%%%%%%%%%%%%%%%%%%%%%%%%%%%%%%%%%%%%%%%%%%%%%%%%%%%%%%%%%%%%%%%
%%%%%%%%%%%%%%%%%%%%%%%%%%%%%%%%%%%%%%%%%%%%%%%%%%%%%%%%%%%%%%%%%%%%%%%%
%%%%%%%%%%%%%%%%%%%%%%%%%%%%%%%%%%%%%%%%%%%%%%%%%%%%%%%%%%%%%%%%%%%%%%%%
\begin{abstract}
    The associative operad is the quotient of the magmatic operad by
    the operad congruence identifying the two binary trees of degree
    $2$. We introduce here a generalization of the associative operad
    depending on a nonnegative integer $d$, called $d$-comb associative
    operad, as the quotient of the magmatic operad by the operad
    congruence identifying the left and the right comb binary trees of
    degree $d$. We study the case $d = 3$ and provide an orientation
    of its space of relations by using rewrite systems on trees and
    the Buchberger algorithm for operads to obtain a convergent
    rewrite system.
\end{abstract}

\maketitle

\tableofcontents

%%%%%%%%%%%%%%%%%%%%%%%%%%%%%%%%%%%%%%%%%%%%%%%%%%%%%%%%%%%%%%%%%%%%%%%%
%%%%%%%%%%%%%%%%%%%%%%%%%%%%%%%%%%%%%%%%%%%%%%%%%%%%%%%%%%%%%%%%%%%%%%%%
%%%%%%%%%%%%%%%%%%%%%%%%%%%%%%%%%%%%%%%%%%%%%%%%%%%%%%%%%%%%%%%%%%%%%%%%
\section*{Introduction}
Associative algebras are spaces endowed with a binary product $\Product$
satisfying among others the associativity law
\begin{math}
    (x_1 \Product x_2) \Product x_3 = x_1 \Product (x_2 \Product x_3)
\end{math}.
It is well-known that the associative algebras are representations of 
the associative (nonsymmetric) operad $\As$. This operad can be seen as 
the quotient of the magmatic operad $\Mag$ (the free operad of binary
trees on the binary generator~$\Product$) by the operad congruence
$\Congr$ satisfying
\begin{equation} \label{equ:congruence_as}
    \begin{tikzpicture}[xscale=.24,yscale=.24,Centering]
        \node(0)at(0.00,-3.33){};
        \node(2)at(2.00,-3.33){};
        \node(4)at(4.00,-1.67){};
        \node[NodeST](1)at(1.00,-1.67)
            {\begin{math}\Product\end{math}};
        \node[NodeST](3)at(3.00,0.00)
            {\begin{math}\Product\end{math}};
        \draw[Edge](0)--(1);
        \draw[Edge](1)--(3);
        \draw[Edge](2)--(1);
        \draw[Edge](4)--(3);
        \node(r)at(3.00,1.5){};
        \draw[Edge](r)--(3);
    \end{tikzpicture}
    \Congr
    \begin{tikzpicture}[xscale=.24,yscale=.24,Centering]
        \node(0)at(0.00,-1.67){};
        \node(2)at(2.00,-3.33){};
        \node(4)at(4.00,-3.33){};
        \node[NodeST](1)at(1.00,0.00)
                {\begin{math}\Product\end{math}};
        \node[NodeST](3)at(3.00,-1.67)
                {\begin{math}\Product\end{math}};
        \draw[Edge](0)--(1);
        \draw[Edge](2)--(3);
        \draw[Edge](3)--(1);
        \draw[Edge](4)--(3);
        \node(r)at(1.00,1.5){};
        \draw[Edge](r)--(1);
    \end{tikzpicture}\,.
\end{equation}
These two binary trees are the syntax trees of the expressions appearing
in the above associativity law.

In a more combinatorial context and regardless of the theory of operads,
the Tamari order is a partial order on the set of the binary trees
having a fixed number of internal nodes $d$. This order is generated by
the covering relation consisting in rewriting a tree $\Tfr$ into a tree
$\Tfr'$ by replacing a subtree of $\Tfr$ of the form of the left member
of~\eqref{equ:congruence_as} into a tree of the form of the right member
of~\eqref{equ:congruence_as}. This transformation is known in a computer
science context as the right rotation operation~\cite{Knu98} and
intervenes in algorithms involving binary search trees~\cite{AVL62}. The
partial order hence generated by the right rotation operation is known
as the Tamari order~\cite{Tam62} and has a lot of combinatorial and
algebraic properties (see for instance~\cite{HT72,Cha06}).

A first connection between the associative operad and the Tamari order
is based upon the fact that the orientation of~\eqref{equ:congruence_as}
from left to right provides a convergent orientation (a terminating and 
confluent rewrite relation) of the congruence $\Congr$. The normal 
forms of the rewrite relation induced by the rewrite rule obtained by 
orienting~\eqref{equ:congruence_as} from left to right are right comb 
binary trees and are hence in one-to-one correspondence with the 
elements of~$\As$.

This work is intended to be a first strike in the study of the eventual
links between the Tamari order and some quotients of the operad $\Mag$.
In the long run, we would like to study quotients $\Mag/_{\Congr}$ of
$\Mag$ where $\Congr$ is an operad congruence generated by equivalence
classes of trees of a fixed degree. In particular, we would like to know
if $\Congr$ is generated by equivalence classes of trees forming
intervals of the Tamari order leads to algebraic properties for
$\Mag/_{\Congr}$ (like the description of orientations of its space of
relations, nice bases and Hilbert series).

We focus here on one of these quotients $\CAs{3}$ which is the operad
describing the category of the algebras equipped with a binary product
$\Product$ and subjected to the relation
\begin{math}
    ((x_1 \Product x_2) \Product x_3) \Product x_4
    =
    x_1 \Product (x_2 \Product (x_3 \Product x_4))
\end{math}.
This is a kind of associativity law in higher degree $d = 3$. This
operad is generated by an equivalence class of trees which is not an
interval for the Tamari order. As preliminary computer experiments show,
$\CAs{3}$ has oscillating first dimensions
(see~\eqref{equ:dimensions_CAs_3}), what is rather unusual among all
known operads. In this paper, we provide an orientation of the space of
relations of $\CAs{3}$. For this, we use rewrite systems on
trees~\cite{BN98} and the Buchberger algorithm for operads~\cite{DK10}.

This text is presented as follows. Section~\ref{sec:operad_Mag} contains
preliminaries about the magmatic operad and rewrite relations on trees.
In Section~\ref{sec:CAs_d}, we define the operad $\CAs{3}$ as a
particular case of a more general construction of generalizations
$\CAs{d}$, $d \geq 1$, of $\As$. Finally, Section~\ref{sec:CAs_3}
contains the orientation of the space of relations of
$\CAs{3}$ (Theorem~\ref{thm:convergent_rewrite_rule_CAs_3}). As
consequences, we obtain for $\CAs{3}$
the description of one of its Poincaré-Birkhoff-Witt bases
(Proposition~\ref{prop:PBW_basis_CAs_3}) and the description of
its Hilbert series (Proposition~\ref{prop:Hilbert_series_CAs_3}).


%%%%%%%%%%%%%%%%%%%%%%%%%%%%%%%%%%%%%%%%%%%%%%%%%%%%%%%%%%%%%%%%%%%%%%%%
%%%%%%%%%%%%%%%%%%%%%%%%%%%%%%%%%%%%%%%%%%%%%%%%%%%%%%%%%%%%%%%%%%%%%%%%
%%%%%%%%%%%%%%%%%%%%%%%%%%%%%%%%%%%%%%%%%%%%%%%%%%%%%%%%%%%%%%%%%%%%%%%%
\section{The magmatic operad, quotients, and rewrite relations}
\label{sec:operad_Mag}
We consider nonsymmetric set-theoretic operads. Let $\Oca$ be such an
operad. We denote respectively by $\circ_i$ and $\circ$ the partial and
complete compositions of $\Oca$. For any $n \geq 1$, $\Oca(n)$ is the
set of the elements $x$ of $\Oca$ of arity $|x| = n$. We denote
by $\Mag$ the magmatic operad, that is the free operad over one binary
generator $\Product$, and we represent the elements of $\Mag$ by binary
trees. The \Def{arity} $|\Tfr|$ (resp. \Def{degree} $\Deg(\Tfr)$) of a
binary tree $\Tfr$ is its number of leaves (resp. internal nodes).
Given a binary tree $\Tfr$, we denote by $ \PrefixWord(\Tfr)$ the
\Def{prefix word} of $\Tfr$, that is the word on $\{\Zero, \Two\}$
obtained by a left to right depth-first traversal of $\Tfr$ and by
writing $\Zero$ (resp. $\Two$) when a leaf (resp. an internal node) is
encountered. The set of all words on $\{\Zero, \Two\}$ is endowed with
the lexicographic order $\leq$ induced by $\Zero < \Two$.

If $\Rew$ is a rewrite rule on $\Mag$ such that $\Sfr \Rew \Sfr'$ 
implies $|\Sfr| = |\Sfr'|$, we denote by $\RewContext$ the
\Def{rewrite relation induced} by $\Rew$. Formally we have
\begin{math}
    \Tfr\circ_i\left(\Sfr\circ\left[\Rfr_1,\dots,\Rfr_{n}\right]\right)
    \RewContext
    \Tfr\circ_i\left(\Sfr'\circ\left[\Rfr_1,\dots,\Rfr_{n}\right]\right)
\end{math},
if $\Sfr \Rew \Sfr'$ where $n = |\Sfr|$, and $\Tfr$, $\Rfr_1$, \dots,
$\Rfr_n$ are binary trees. In other words, one has
$\Tfr \RewContext \Tfr'$ if it is possible to obtain $\Tfr'$ from $\Tfr$
by replacing a subtree $\Sfr$ of $\Tfr$ by $\Sfr'$ whenever
$\Sfr \Rew \Sfr'$. We use here the standard terminology
(\Def{terminating}, \Def{confluent}, \Def{convergent}, \Def{branching
pair}, \Def{joinable}, \Def{normal form}, {\em etc.}) about rewrite
relations and rewrite systems~\cite{BN98}.

Given an operad $\Oca \simeq \Mag/_{\Congr}$ where $\Congr$ is an operad
congruence of $\Mag$, we say that $\Rew$ is an \Def{orientation} of
$\Congr$ if the reflexive, transitive, and symmetric closure of
$\RewContext$ is $\Congr$. We say that $\Rew$ is a \Def{convergent
orientation} if $\RewContext$ is convergent. When $\Rew$ is a convergent
orientation of $\Congr$, the set of all normal forms of $\RewContext$ is
a \Def{Poincaré-Birkhoff-Witt basis} of the operad $\Oca$ and its
elements are exactly the binary trees avoiding, as subtrees, the trees
appearing as left members in $\Rew$.

We shall use the following criterion to prove that a rewrite relation on
$\Mag$ is terminating.

\begin{Lemma}\label{lem:prefix_word_termination}
    Let $\Rew$ be a rewrite rule on $\Mag$. If for any
    $\Tfr, \Tfr' \in \Mag$ such that $\Tfr \Rew \Tfr'$ one has
    $\PrefixWord(\Tfr) > \PrefixWord(\Tfr')$, then the rewrite relation
    induced by $\Rew$ is terminating.
\end{Lemma}

Moreover, we shall use the following result appearing in~\cite{Gir16}
specialized on rewrite relation on $\Mag$ to prove that a terminating
rewrite relation is convergent.


\begin{Lemma} \label{lem:degree_confluence}
    Let $\Rew$ be a rewrite rule on $\Mag$ wherein all trees $\Tfr$ and 
    $\Tfr'$ such that $\Tfr \Rew \Tfr'$ have degrees at most $\ell$. 
    Then, if the rewrite relation $\RewContext$ induced by $\Rew$ is
    terminating and all its branching pairs of degrees at most
    $2\ell - 1$ are joinable, $\RewContext$ is convergent.
\end{Lemma}

%%%%%%%%%%%%%%%%%%%%%%%%%%%%%%%%%%%%%%%%%%%%%%%%%%%%%%%%%%%%%%%%%%%%%%%%
%%%%%%%%%%%%%%%%%%%%%%%%%%%%%%%%%%%%%%%%%%%%%%%%%%%%%%%%%%%%%%%%%%%%%%%%
%%%%%%%%%%%%%%%%%%%%%%%%%%%%%%%%%%%%%%%%%%%%%%%%%%%%%%%%%%%%%%%%%%%%%%%%
\section{Generalizations of the associative operad} \label{sec:CAs_d}
It is known that the rewrite rule $\Rew$
orienting~\eqref{equ:congruence_as} from left to right is a convergent
orientation of (\ref{equ:congruence_as}). Then, a Poincaré-Birkhoff-Witt
basis of $\As$ is the set of all right comb binary trees.

Let us now define for any $d \geq 1$ the \Def{$\gamma$-comb associative
operad} $\CAs{\gamma}$ as the quotient operad $\Mag/_{\CongrCAs{\gamma}}$ where
$\CongrCAs{\gamma}$ is the smallest operad congruence of $\Mag$
satisfying
\begin{equation} \label{equ:congruence_Asd}
    \underbrace{(\dots (\Product \circ_1 \Product) \circ_1 \dots)
        \circ_1 \Product}
    _{\gamma \mbox{ \footnotesize operands}}
    \enspace \CongrCAs{\gamma} \enspace
    \underbrace{\Product \circ_2
        (\dots \circ_2 (\Product \circ_2 \Product) \dots)}
    _{\gamma \mbox{ \footnotesize operands}}.
\end{equation}
In words, \eqref{equ:congruence_Asd} says that the left and the right
comb binary trees of degree $\gamma$ are equivalent for $\CongrCAs{\gamma}$.
Notice that $\CongrCAs{1}$ is trivial so that $\CAs{1} = \Mag$ and that
$\CongrCAs{2}$ is the operad congruence defined
by~\eqref{equ:congruence_as} so that $\CAs{2} = \As$.

In the sequel we denote by $\LC{\gamma}$ and $\RC{\gamma}$ the left comb and the
right comb of degree $\gamma$, respectively. Hence, $\CongrCAs{\gamma}$ is the
congruence relation induced by $\LC{\gamma}\CongrCAs{\gamma}\RC{\gamma}$.

We denote by CAs the set of $\gamma$-comb associative operads where $\gamma$ belongs
the set of strictly positive integers:
\[\text{CAs}=\left\{\CAs{\gamma}\mid\gamma\geq 1\right\}.\]
Our purpose is to show that CAs admits a lattice structure. For that, we provide
a description of operads morphisms between the elements of CAs.

\begin{Proposition}\label{prop:morphisms_CAs_d}
  
  Let $\gamma$ and $\gamma'$ be two strictly positive integers. There exists at most
  one operads morphism $\varphi:\CAs{\gamma'}\to\CAs{\gamma}$. This morphism
  exists if and only if $\emph{LC}\left(\gamma'\right)\CongrCAs{\gamma}\emph{RC}\left(\gamma'\right)$ and it is surjective in this
  case. Moreover, $\varphi$ is injective if and only if $\gamma=\gamma'$, that is if and only
  if $\varphi$ is the identity morphism.
  
  \end{Proposition}

\begin{proof}
  The operad $\CAs{\gamma'}$ is generated by one binary generator $t^{(\gamma')}$, so that
  $\varphi$ is entirely determined by $\varphi\left(t^{(\gamma')}\right)$. Moreover,
  $\varphi\left(t^{(\gamma')}\right)$ has to be of arity $2$ in $\CAs{\gamma}$, so that
  $\varphi\left(t^{(\gamma')}\right)$ has to be equal to $t^{(\gamma)}$. Hence, if $\varphi$
  exists, it is the unique operads morphism from $\CAs{\gamma'}$ and $\CAs{\gamma}$,
  and in this case, $t^{(\gamma)}$ being in the image of $\varphi$, the latter is
  surjective. Moreover, $\varphi$ is well defined if and only if
  $\varphi\left(\LC{\gamma'}\right)$ and $\varphi\left(\RC{\gamma'}\right)$ are equal in
  $\CAs{\gamma}$, that is if and only if $\LC{\gamma'}\CongrCAs{\gamma}\RC{\gamma'}$. In particular,
  if $\gamma'$ is strictly smaller than $\gamma$, then there does not exist any morphism
  $\varphi:\CAs{\gamma'}\to\CAs{\gamma}$. Moreover, if $\gamma'$ is strictly smaller
  than $\gamma$, the cardinality of $\CAs{\gamma'}\left(\gamma\right)$ is equal to the cardinality
  of $\CAs{\gamma}(\gamma)$ plus one. Hence, if there exists an injective morphism
  $\varphi:\CAs{\gamma'}\to\CAs{\gamma}$, we must have $\gamma=\gamma'$, and in this case,
  $\varphi$ is the identity morphism.
\end{proof}

We define the binary relation $\preceq$ on CAs as follows: we have $\CAs{\gamma}\preceq\CAs{\gamma'}$
if and only if there exists a morphism $\varphi:\CAs{\gamma'}\to\CAs{\gamma}$.

\begin{Proposition}\label{prop:poset_CAs}
  The binary relation $\preceq$ is a partially order on \emph{CAs}, that is
  $\left(\emph{CAs},\ \preceq\right)$ is a poset.
  \end{Proposition}

\begin{proof}
  The binary relation $\preceq$ is reflexive since there exists the identity morphism
  of $\CAs{\gamma}$ for every strictly positive integer $\gamma$. It is transitive since if
  there exist two morphisms  $\varphi:\CAs{\gamma}\to\CAs{\gamma'}$ and
  $\psi:\CAs{\gamma''}\to\CAs{\gamma'}$, then $\psi\circ\varphi$ is a morphism from
  $\CAs{\gamma''}$ to $\CAs{\gamma}$. Now, let us assume that there exist morphisms
  $\varphi:\CAs{\gamma'}\to\CAs{\gamma}$ and $\psi:\CAs{\gamma}\to\CAs{\gamma'}$.
  In particular, $\psi\circ\varphi$ and $\varphi\circ\psi$ are endomorphisms of $\CAs{\gamma'}$
  and $\CAs{\gamma}$, respectively. From Proposition~\ref{prop:morphisms_CAs_d}, these two
  morphisms are identity morphisms, so that $\varphi$ and $\psi$ are injective. From
  Proposition~\ref{prop:morphisms_CAs_d}, $\gamma$ and $\gamma'$ are equal, which proves that
  $\preceq$ is anti-symmetric, hence a partial order.
  \end{proof}

In order to show that the poset $\left(\text{CAs},\ \preceq\right)$ extends into a lattice,
that is two elements of CAs admit lower-bounds and upper-bounds, we relate
$\left(\text{CAs},\ \preceq\right)$ to the lattice of integers
$\left(\N,\ \mid,\ \text{gcd},\ \text{lcm}\right)$, where $\mid$ denotes the division
relation, gcd denotes the \emph{greatest common divisor} and lcm the \emph{least common multiple}
operators, respectively. In order to obtain this relationship, we introduce the notion of
\Def{left rank} of an element $t\in\Mag$: this is the length $\text{lr}\left(t\right)$ of
the left branch beginning at the root of $t$. The left rank of $t$ is represented as follows:
\[\textbf{PICTURE}\]

\begin{Lemma}\label{lem:left_rank_and_CongrCAs}
  Let $\gamma$ be a strictly positive integer and let $t$ and $t'$ be two elements of $\Mag$. If
  $t\CongrCAs{\gamma}t'$, then $\emph{lr}\left(t\right)$ and $\emph{lr}\left(t'\right)$ are equal modulo $\gamma-1$: $\emph{lr}\left(t\right)\equiv\emph{lr}\left(t'\right) [\gamma-1]$.
\end{Lemma}

\begin{proof}
  Let $\Rew$ be the rewrite relation on $\Mag$ induced by the rewrite rule
  $\LC{\gamma}\Rew\RC{\gamma}$. In the rest of the proof we fix two elements $t$ and $t'$ of $\Mag$.

  First, we show that $t\Rew t'$ implies that $\text{lr}\left(t'\right)-\text{lr}\left(t\right)$
  is divisible by $\gamma-1$ \textbf{TODO}.

  Assume that $t\overset{*}{\Rew}t'$, so that there exist elements $t_1,\ \cdots,\ t_n$
  of $\Mag$ such that $t_1=t$, $t_n=t'$ and for every $i\in\{1,\cdots,n-1\}$, we have
  $t_i\Rew t_{i+1}$. The integer $\text{lr}\left(t_{i+1}\right)-\text{lr}\left(t_i\right)$
  is divisible by $\gamma-1$ from the first part of the proof, so that
  $\text{lr}\left(t'\right)-\text{lr}\left(t\right)$ is divisible by $\gamma-1$.

  Assume that $t\overset{*}{\leftrightarrow}t'$, so that there exist elements $t_1,\ \cdots,\ t_n$
  of $\Mag$ such that $t_1=t$, $t_n=t'$ and for every $i\in\{1,\cdots,n-1\}$, we have
  $t_i\overset{*}{\Rew}t_{i+1}$ or $t_{i+1}\overset{*}{\Rew}t_i$. From the second part of the proof,
  $\text{lr}\left(t_{i+1}\right)-\text{lr}\left(t_i\right)$ is divisible by $\gamma-1$, so that
  $\text{lr}\left(t'\right)-\text{lr}\left(t\right)$ is divisible by $\gamma-1$.
  
  The congruence relation $\CongrCAs{\gamma}$ is the equivalence relation induced by $\Rew$, that
  is we have $t\CongrCAs{\gamma}t'$ if and only if $t\overset{*}{\leftrightarrow}t'$. Hence, from the
  third part of the proof, $t\CongrCAs{\gamma}t'$ implies that
  $\text{lr}\left(t'\right)-\text{lr}\left(t\right)$ is divisible by $\gamma-1$.
  \end{proof}

\begin{Proposition} \label{prop:division_CAs}
  Let $\gamma$ and $\gamma'$ be two strictly positive integers. There exists a morphism
  $\varphi:\CAs{\gamma'}\to\CAs{\gamma}$ if and only if $\left(\gamma-1\right)\mid\left(\gamma'-1\right)$.
\end{Proposition}
\begin{proof}
  From Proposition~\ref{prop:morphisms_CAs_d}, it is sufficient to show that $\LC{\gamma'}\CongrCAs{\gamma}\RC{\gamma'}$
  if and only if $\left(\gamma-1\right)\mid\left(\gamma'-1\right)$. If $\LC{\gamma'}\CongrCAs{d}\RC{\gamma'}$, from
  Lemma~\ref{lem:left_rank_and_CongrCAs}, $\text{lr}\left(\LC{\gamma'}\right)-\text{lr}\left(\RC{\gamma'}\right)=\gamma'-1$
  is divisible by $\gamma-1$, which shows the direct implication. Conversely, if
  $\left(\gamma-1\right)\mid\left(\gamma'-1\right)$, the rewrite rule $\LC{\gamma}\Rew\RC{\gamma}$ induces the
  following sequence of rewriting steps:
  \[\textbf{Rewriting sequence from left comb of degree d' to right comb of degree d'}\]
  Hence, we have $\LC{\gamma'}\CongrCAs{\gamma}\RC{\gamma'}$.
   \end{proof}

Proposition~\ref{prop:division_CAs} implies the following:
\begin{Theorem}\label{thm:lattice_CAs}
  The poset $\left(\emph{CAs},\ \preceq\right)$ extends into a lattice
  $\left(\emph{CAs},\ \preceq,\ \wedge,\ \vee\right)$, where the lower-
  bound $\wedge$ and the upper-bound $\vee$ are defined as follows:
  \[\CAs{\gamma}\wedge\CAs{\gamma'}=\CAs{\text{gcd}\left(\gamma-1,\ \gamma'-1\right)}\
  \text{and}\ \CAs{\gamma}\vee\CAs{\gamma'}=\CAs{\text{lcm}\left(\gamma-1,\ \gamma'-1\right)}.\]
  \end{Theorem}



%%%%%%%%%%%%%%%%%%%%%%%%%%%%%%%%%%%%%%%%%%%%%%%%%%%%%%%%%%%%%%%%%%%%%%%%
%%%%%%%%%%%%%%%%%%%%%%%%%%%%%%%%%%%%%%%%%%%%%%%%%%%%%%%%%%%%%%%%%%%%%%%%
%%%%%%%%%%%%%%%%%%%%%%%%%%%%%%%%%%%%%%%%%%%%%%%%%%%%%%%%%%%%%%%%%%%%%%%%
\section{The \texorpdfstring{$3$}{3}-comb associative operad}
\label{sec:CAs_3}
We now focus on the study of the operad $\CAs{3}$. By definition, this
operad is the quotient of $\Mag$ by the operad congruence spanned by the
relation
\begin{equation} \label{equ:rew_1}
    \begin{tikzpicture}[xscale=.22,yscale=.23,Centering]
        \node(0)at(0.00,-5.25){};
        \node(2)at(2.00,-5.25){};
        \node(4)at(4.00,-3.50){};
        \node(6)at(6.00,-1.75){};
        \node[NodeST](1)at(1.00,-3.50){\begin{math}\Product\end{math}};
        \node[NodeST](3)at(3.00,-1.75){\begin{math}\Product\end{math}};
        \node[NodeST](5)at(5.00,0.00){\begin{math}\Product\end{math}};
        \draw[Edge](0)--(1);
        \draw[Edge](1)--(3);
        \draw[Edge](2)--(1);
        \draw[Edge](3)--(5);
        \draw[Edge](4)--(3);
        \draw[Edge](6)--(5);
        \node(r)at(5.00,1.5){};
        \draw[Edge](r)--(5);
    \end{tikzpicture}
    \enspace \Rew \enspace
    \begin{tikzpicture}[xscale=.22,yscale=.23,Centering]
        \node(0)at(0.00,-1.75){};
        \node(2)at(2.00,-3.50){};
        \node(4)at(4.00,-5.25){};
        \node(6)at(6.00,-5.25){};
        \node[NodeST](1)at(1.00,0.00){\begin{math}\Product\end{math}};
        \node[NodeST](3)at(3.00,-1.75){\begin{math}\Product\end{math}};
        \node[NodeST](5)at(5.00,-3.50){\begin{math}\Product\end{math}};
        \draw[Edge](0)--(1);
        \draw[Edge](2)--(3);
        \draw[Edge](3)--(1);
        \draw[Edge](4)--(5);
        \draw[Edge](5)--(3);
        \draw[Edge](6)--(5);
        \node(r)at(1.00,1.5){};
        \draw[Edge](r)--(1);
    \end{tikzpicture}\,.
\end{equation}
This rewrite rule is compatible with the lexicographic order on prefix 
words presented at the beginning of Section~\ref{sec:operad_Mag} in the 
sense that the prefix word of the left member of~\eqref{equ:rew_1} is 
lexicographically greater than the prefix word of the right one.

However, the rewrite relation $\RewContext$ induced by $\Rew$ is not
confluent. Indeed, we have
\begin{equation} \label{equ:branching_pair_CAs_3}
    \begin{tikzpicture}[xscale=.22,yscale=.22,Centering]
        \node(0)at(0.00,-7.20){};
        \node(2)at(2.00,-7.20){};
        \node(4)at(4.00,-5.40){};
        \node(6)at(6.00,-3.60){};
        \node(8)at(8.00,-1.80){};
        \node[NodeST](1)at(1.00,-5.40){\begin{math}\Product\end{math}};
        \node[NodeST](3)at(3.00,-3.60){\begin{math}\Product\end{math}};
        \node[NodeST](5)at(5.00,-1.80){\begin{math}\Product\end{math}};
        \node[NodeST](7)at(7.00,0.00){\begin{math}\Product\end{math}};
        \draw[Edge](0)--(1);
        \draw[Edge](1)--(3);
        \draw[Edge](2)--(1);
        \draw[Edge](3)--(5);
        \draw[Edge](4)--(3);
        \draw[Edge](5)--(7);
        \draw[Edge](6)--(5);
        \draw[Edge](8)--(7);
        \node(r)at(7.00,1.35){};
        \draw[Edge](r)--(7);
    \end{tikzpicture}
    \enspace \RewContext \enspace
    \begin{tikzpicture}[xscale=.22,yscale=.20,Centering]
        \node(0)at(0.00,-4.50){};
        \node(2)at(2.00,-4.50){};
        \node(4)at(4.00,-4.50){};
        \node(6)at(6.00,-6.75){};
        \node(8)at(8.00,-6.75){};
        \node[NodeST](1)at(1.00,-2.25){\begin{math}\Product\end{math}};
        \node[NodeST](3)at(3.00,0.00){\begin{math}\Product\end{math}};
        \node[NodeST](5)at(5.00,-2.25){\begin{math}\Product\end{math}};
        \node[NodeST](7)at(7.00,-4.50){\begin{math}\Product\end{math}};
        \draw[Edge](0)--(1);
        \draw[Edge](1)--(3);
        \draw[Edge](2)--(1);
        \draw[Edge](4)--(5);
        \draw[Edge](5)--(3);
        \draw[Edge](6)--(7);
        \draw[Edge](7)--(5);
        \draw[Edge](8)--(7);
        \node(r)at(3.00,1.74){};
        \draw[Edge](r)--(3);
    \end{tikzpicture}
    \qquad \mbox{and} \qquad
    \begin{tikzpicture}[xscale=.22,yscale=.22,Centering]
        \node(0)at(0.00,-7.20){};
        \node(2)at(2.00,-7.20){};
        \node(4)at(4.00,-5.40){};
        \node(6)at(6.00,-3.60){};
        \node(8)at(8.00,-1.80){};
        \node[NodeST](1)at(1.00,-5.40){\begin{math}\Product\end{math}};
        \node[NodeST](3)at(3.00,-3.60){\begin{math}\Product\end{math}};
        \node[NodeST](5)at(5.00,-1.80){\begin{math}\Product\end{math}};
        \node[NodeST](7)at(7.00,0.00){\begin{math}\Product\end{math}};
        \draw[Edge](0)--(1);
        \draw[Edge](1)--(3);
        \draw[Edge](2)--(1);
        \draw[Edge](3)--(5);
        \draw[Edge](4)--(3);
        \draw[Edge](5)--(7);
        \draw[Edge](6)--(5);
        \draw[Edge](8)--(7);
        \node(r)at(7.00,1.35){};
        \draw[Edge](r)--(7);
    \end{tikzpicture}
    \enspace \RewContext \enspace
    \begin{tikzpicture}[xscale=.22,yscale=.22,Centering]
        \node(0)at(0.00,-3.60){};
        \node(2)at(2.00,-5.40){};
        \node(4)at(4.00,-7.20){};
        \node(6)at(6.00,-7.20){};
        \node(8)at(8.00,-1.80){};
        \node[NodeST](1)at(1.00,-1.80){\begin{math}\Product\end{math}};
        \node[NodeST](3)at(3.00,-3.60){\begin{math}\Product\end{math}};
        \node[NodeST](5)at(5.00,-5.40){\begin{math}\Product\end{math}};
        \node[NodeST](7)at(7.00,0.00){\begin{math}\Product\end{math}};
        \draw[Edge](0)--(1);
        \draw[Edge](1)--(7);
        \draw[Edge](2)--(3);
        \draw[Edge](3)--(1);
        \draw[Edge](4)--(5);
        \draw[Edge](5)--(3);
        \draw[Edge](6)--(5);
        \draw[Edge](8)--(7);
        \node(r)at(7.00,1.5){};
        \draw[Edge](r)--(7);
    \end{tikzpicture}\,,
\end{equation}
and the two right members of~\eqref{equ:branching_pair_CAs_3} form a
branching pair which is not joinable.

In order to transform the rewrite relation induced by~\eqref{equ:rew_1} 
into a convergent one, we apply the Buchberger algorithm for
operads~\cite[Section 3.7]{DK10} with respect to the lexicographic order
on prefix words. Following this algorithm, we need to put the right
members of~\eqref{equ:branching_pair_CAs_3} in relation by $\Rew$. To
respect the lexicographic property of the prefix words, this leads to
the new relation
\begin{equation} \label{equ:rew_2}
    \begin{tikzpicture}[xscale=.22,yscale=.22,Centering]
        \node(0)at(0.00,-3.60){};
        \node(2)at(2.00,-5.40){};
        \node(4)at(4.00,-7.20){};
        \node(6)at(6.00,-7.20){};
        \node(8)at(8.00,-1.80){};
        \node[NodeST](1)at(1.00,-1.80){\begin{math}\Product\end{math}};
        \node[NodeST](3)at(3.00,-3.60){\begin{math}\Product\end{math}};
        \node[NodeST](5)at(5.00,-5.40){\begin{math}\Product\end{math}};
        \node[NodeST](7)at(7.00,0.00){\begin{math}\Product\end{math}};
        \draw[Edge](0)--(1);
        \draw[Edge](1)--(7);
        \draw[Edge](2)--(3);
        \draw[Edge](3)--(1);
        \draw[Edge](4)--(5);
        \draw[Edge](5)--(3);
        \draw[Edge](6)--(5);
        \draw[Edge](8)--(7);
        \node(r)at(7.00,1.5){};
        \draw[Edge](r)--(7);
    \end{tikzpicture}
    \enspace \Rew \enspace
    \begin{tikzpicture}[xscale=.22,yscale=.20,Centering]
        \node(0)at(0.00,-4.50){};
        \node(2)at(2.00,-4.50){};
        \node(4)at(4.00,-4.50){};
        \node(6)at(6.00,-6.75){};
        \node(8)at(8.00,-6.75){};
        \node[NodeST](1)at(1.00,-2.25){\begin{math}\Product\end{math}};
        \node[NodeST](3)at(3.00,0.00){\begin{math}\Product\end{math}};
        \node[NodeST](5)at(5.00,-2.25){\begin{math}\Product\end{math}};
        \node[NodeST](7)at(7.00,-4.50){\begin{math}\Product\end{math}};
        \draw[Edge](0)--(1);
        \draw[Edge](1)--(3);
        \draw[Edge](2)--(1);
        \draw[Edge](4)--(5);
        \draw[Edge](5)--(3);
        \draw[Edge](6)--(7);
        \draw[Edge](7)--(5);
        \draw[Edge](8)--(7);
        \node(r)at(3.00,1.74){};
        \draw[Edge](r)--(3);
    \end{tikzpicture}\,.
\end{equation}
The Buchberger algorithm applied on binary trees of degrees $5$, $6$,
and $7$ provides the new relations \\
\begin{minipage}{7cm}
\begin{equation} \label{equ:rew_3}
    \begin{tikzpicture}[xscale=.23,yscale=.21,Centering]
        \node(0)at(0.00,-1.83){};
        \node(10)at(10.00,-5.50){};
        \node(2)at(2.00,-3.67){};
        \node(4)at(4.00,-7.33){};
        \node(6)at(6.00,-9.17){};
        \node(8)at(8.00,-9.17){};
        \node[NodeST](1)at(1.00,0.00){\begin{math}\Product\end{math}};
        \node[NodeST](3)at(3.00,-1.83){\begin{math}\Product\end{math}};
        \node[NodeST](5)at(5.00,-5.50){\begin{math}\Product\end{math}};
        \node[NodeST](7)at(7.00,-7.33){\begin{math}\Product\end{math}};
        \node[NodeST](9)at(9.00,-3.67){\begin{math}\Product\end{math}};
        \draw[Edge](0)--(1);
        \draw[Edge](10)--(9);
        \draw[Edge](2)--(3);
        \draw[Edge](3)--(1);
        \draw[Edge](4)--(5);
        \draw[Edge](5)--(9);
        \draw[Edge](6)--(7);
        \draw[Edge](7)--(5);
        \draw[Edge](8)--(7);
        \draw[Edge](9)--(3);
        \node(r)at(1.00,1.75){};
        \draw[Edge](r)--(1);
    \end{tikzpicture}
    \enspace \Rew \enspace
    \begin{tikzpicture}[xscale=.22,yscale=.24,Centering]
        \node(0)at(0.00,-1.83){};
        \node(10)at(10.00,-9.17){};
        \node(2)at(2.00,-3.67){};
        \node(4)at(4.00,-5.50){};
        \node(6)at(6.00,-7.33){};
        \node(8)at(8.00,-9.17){};
        \node[NodeST](1)at(1.00,0.00){\begin{math}\Product\end{math}};
        \node[NodeST](3)at(3.00,-1.83){\begin{math}\Product\end{math}};
        \node[NodeST](5)at(5.00,-3.67){\begin{math}\Product\end{math}};
        \node[NodeST](7)at(7.00,-5.50){\begin{math}\Product\end{math}};
        \node[NodeST](9)at(9.00,-7.33){\begin{math}\Product\end{math}};
        \draw[Edge](0)--(1);
        \draw[Edge](10)--(9);
        \draw[Edge](2)--(3);
        \draw[Edge](3)--(1);
        \draw[Edge](4)--(5);
        \draw[Edge](5)--(3);
        \draw[Edge](6)--(7);
        \draw[Edge](7)--(5);
        \draw[Edge](8)--(9);
        \draw[Edge](9)--(7);
        \node(r)at(1.00,1.5){};
        \draw[Edge](r)--(1);
    \end{tikzpicture}
\end{equation}
\end{minipage},
\begin{minipage}{7cm}
\begin{equation}\label{equ:rew_4}
     \begin{tikzpicture}[xscale=.2,yscale=.19,Centering]
        \node(0)at(0.00,-2.20){};
        \node(10)at(10.00,-6.60){};
        \node(2)at(2.00,-6.60){};
        \node(4)at(4.00,-8.80){};
        \node(6)at(6.00,-8.80){};
        \node(8)at(8.00,-6.60){};
        \node[NodeST](1)at(1.00,0.00){\begin{math}\Product\end{math}};
        \node[NodeST](3)at(3.00,-4.40){\begin{math}\Product\end{math}};
        \node[NodeST](5)at(5.00,-6.60){\begin{math}\Product\end{math}};
        \node[NodeST](7)at(7.00,-2.20){\begin{math}\Product\end{math}};
        \node[NodeST](9)at(9.00,-4.40){\begin{math}\Product\end{math}};
        \draw[Edge](0)--(1);
        \draw[Edge](10)--(9);
        \draw[Edge](2)--(3);
        \draw[Edge](3)--(7);
        \draw[Edge](4)--(5);
        \draw[Edge](5)--(3);
        \draw[Edge](6)--(5);
        \draw[Edge](7)--(1);
        \draw[Edge](8)--(9);
        \draw[Edge](9)--(7);
        \node(r)at(1.00,2){};
        \draw[Edge](r)--(1);
    \end{tikzpicture}
    \enspace \Rew \enspace
    \begin{tikzpicture}[xscale=.22,yscale=.24,Centering]
        \node(0)at(0.00,-1.83){};
        \node(10)at(10.00,-9.17){};
        \node(2)at(2.00,-3.67){};
        \node(4)at(4.00,-5.50){};
        \node(6)at(6.00,-7.33){};
        \node(8)at(8.00,-9.17){};
        \node[NodeST](1)at(1.00,0.00){\begin{math}\Product\end{math}};
        \node[NodeST](3)at(3.00,-1.83){\begin{math}\Product\end{math}};
        \node[NodeST](5)at(5.00,-3.67){\begin{math}\Product\end{math}};
        \node[NodeST](7)at(7.00,-5.50){\begin{math}\Product\end{math}};
        \node[NodeST](9)at(9.00,-7.33){\begin{math}\Product\end{math}};
        \draw[Edge](0)--(1);
        \draw[Edge](10)--(9);
        \draw[Edge](2)--(3);
        \draw[Edge](3)--(1);
        \draw[Edge](4)--(5);
        \draw[Edge](5)--(3);
        \draw[Edge](6)--(7);
        \draw[Edge](7)--(5);
        \draw[Edge](8)--(9);
        \draw[Edge](9)--(7);
        \node(r)at(1.00,1.75){};
        \draw[Edge](r)--(1);
    \end{tikzpicture}
\end{equation}
\end{minipage}, \\
\begin{minipage}{7cm}
\begin{equation} \label{equ:rew_5}
    \begin{tikzpicture}[xscale=.2,yscale=.2,Centering]
        \node(0)at(0.00,-2.17){};
        \node(10)at(10.00,-10.83){};
        \node(12)at(12.00,-10.83){};
        \node(2)at(2.00,-4.33){};
        \node(4)at(4.00,-8.67){};
        \node(6)at(6.00,-8.67){};
        \node(8)at(8.00,-8.67){};
        \node[NodeST](1)at(1.00,0.00){\begin{math}\Product\end{math}};
        \node[NodeST](11)at(11.00,-8.67)
            {\begin{math}\Product\end{math}};
        \node[NodeST](3)at(3.00,-2.17){\begin{math}\Product\end{math}};
        \node[NodeST](5)at(5.00,-6.50){\begin{math}\Product\end{math}};
        \node[NodeST](7)at(7.00,-4.33){\begin{math}\Product\end{math}};
        \node[NodeST](9)at(9.00,-6.50){\begin{math}\Product\end{math}};
        \draw[Edge](0)--(1);
        \draw[Edge](10)--(11);
        \draw[Edge](11)--(9);
        \draw[Edge](12)--(11);
        \draw[Edge](2)--(3);
        \draw[Edge](3)--(1);
        \draw[Edge](4)--(5);
        \draw[Edge](5)--(7);
        \draw[Edge](6)--(5);
        \draw[Edge](7)--(3);
        \draw[Edge](8)--(9);
        \draw[Edge](9)--(7);
        \node(r)at(1.00,1.75){};
        \draw[Edge](r)--(1);
    \end{tikzpicture}
    \Rew
    \begin{tikzpicture}[xscale=.21,yscale=.22,Centering]
        \node(0)at(0.00,-1.86){};
        \node(10)at(10.00,-11.14){};
        \node(12)at(12.00,-9.29){};
        \node(2)at(2.00,-3.71){};
        \node(4)at(4.00,-5.57){};
        \node(6)at(6.00,-7.43){};
        \node(8)at(8.00,-11.14){};
        \node[NodeST](1)at(1.00,0.00){\begin{math}\Product\end{math}};
        \node[NodeST](11)at(11.00,-7.43)
            {\begin{math}\Product\end{math}};
        \node[NodeST](3)at(3.00,-1.86){\begin{math}\Product\end{math}};
        \node[NodeST](5)at(5.00,-3.71){\begin{math}\Product\end{math}};
        \node[NodeST](7)at(7.00,-5.57){\begin{math}\Product\end{math}};
        \node[NodeST](9)at(9.00,-9.29){\begin{math}\Product\end{math}};
        \draw[Edge](0)--(1);
        \draw[Edge](10)--(9);
        \draw[Edge](11)--(7);
        \draw[Edge](12)--(11);
        \draw[Edge](2)--(3);
        \draw[Edge](3)--(1);
        \draw[Edge](4)--(5);
        \draw[Edge](5)--(3);
        \draw[Edge](6)--(7);
        \draw[Edge](7)--(5);
        \draw[Edge](8)--(9);
        \draw[Edge](9)--(11);
        \node(r)at(1.00,1.75){};
        \draw[Edge](r)--(1);
    \end{tikzpicture}
\end{equation}
\end{minipage},
\begin{minipage}{7cm}
\begin{equation} \label{equ:rew_6}
    \begin{tikzpicture}[xscale=.21,yscale=.21,Centering]
        \node(0)at(0.00,-2.17){};
        \node(10)at(10.00,-10.83){};
        \node(12)at(12.00,-10.83){};
        \node(2)at(2.00,-6.50){};
        \node(4)at(4.00,-6.50){};
        \node(6)at(6.00,-6.50){};
        \node(8)at(8.00,-8.67){};
        \node[NodeST](1)at(1.00,0.00){\begin{math}\Product\end{math}};
        \node[NodeST](11)at(11.00,-8.67)
            {\begin{math}\Product\end{math}};
        \node[NodeST](3)at(3.00,-4.33){\begin{math}\Product\end{math}};
        \node[NodeST](5)at(5.00,-2.17){\begin{math}\Product\end{math}};
        \node[NodeST](7)at(7.00,-4.33){\begin{math}\Product\end{math}};
        \node[NodeST](9)at(9.00,-6.50){\begin{math}\Product\end{math}};
        \draw[Edge](0)--(1);
        \draw[Edge](10)--(11);
        \draw[Edge](11)--(9);
        \draw[Edge](12)--(11);
        \draw[Edge](2)--(3);
        \draw[Edge](3)--(5);
        \draw[Edge](4)--(3);
        \draw[Edge](5)--(1);
        \draw[Edge](6)--(7);
        \draw[Edge](7)--(5);
        \draw[Edge](8)--(9);
        \draw[Edge](9)--(7);
        \node(r)at(1.00,1.75){};
        \draw[Edge](r)--(1);
    \end{tikzpicture}
    \Rew
    \begin{tikzpicture}[xscale=.22,yscale=.21,Centering]
        \node(0)at(0.00,-2.17){};
        \node(10)at(10.00,-10.83){};
        \node(12)at(12.00,-10.83){};
        \node(2)at(2.00,-4.33){};
        \node(4)at(4.00,-6.50){};
        \node(6)at(6.00,-10.83){};
        \node(8)at(8.00,-10.83){};
        \node[NodeST](1)at(1.00,0.00){\begin{math}\Product\end{math}};
        \node[NodeST](11)at(11.00,-8.67)
            {\begin{math}\Product\end{math}};
        \node[NodeST](3)at(3.00,-2.17){\begin{math}\Product\end{math}};
        \node[NodeST](5)at(5.00,-4.33){\begin{math}\Product\end{math}};
        \node[NodeST](7)at(7.00,-8.67){\begin{math}\Product\end{math}};
        \node[NodeST](9)at(9.00,-6.50){\begin{math}\Product\end{math}};
        \draw[Edge](0)--(1);
        \draw[Edge](10)--(11);
        \draw[Edge](11)--(9);
        \draw[Edge](12)--(11);
        \draw[Edge](2)--(3);
        \draw[Edge](3)--(1);
        \draw[Edge](4)--(5);
        \draw[Edge](5)--(3);
        \draw[Edge](6)--(7);
        \draw[Edge](7)--(9);
        \draw[Edge](8)--(7);
        \draw[Edge](9)--(5);
        \node(r)at(1.00,1.75){};
        \draw[Edge](r)--(1);
    \end{tikzpicture}\hspace{-8pt}
\end{equation}
\end{minipage}, \\
\begin{minipage}{7cm}
\begin{equation} \label{equ:rew_7}
    \begin{tikzpicture}[xscale=.19,yscale=.17,Centering]
        \node(0)at(0.00,-5.20){};
        \node(10)at(10.00,-7.80){};
        \node(12)at(12.00,-7.80){};
        \node(2)at(2.00,-10.40){};
        \node(4)at(4.00,-10.40){};
        \node(6)at(6.00,-7.80){};
        \node(8)at(8.00,-5.20){};
        \node[NodeST](1)at(1.00,-2.60){\begin{math}\Product\end{math}};
        \node[NodeST](11)at(11.00,-5.20)
            {\begin{math}\Product\end{math}};
        \node[NodeST](3)at(3.00,-7.80){\begin{math}\Product\end{math}};
        \node[NodeST](5)at(5.00,-5.20){\begin{math}\Product\end{math}};
        \node[NodeST](7)at(7.00,0.00){\begin{math}\Product\end{math}};
        \node[NodeST](9)at(9.00,-2.60){\begin{math}\Product\end{math}};
        \draw[Edge](0)--(1);
        \draw[Edge](1)--(7);
        \draw[Edge](10)--(11);
        \draw[Edge](11)--(9);
        \draw[Edge](12)--(11);
        \draw[Edge](2)--(3);
        \draw[Edge](3)--(5);
        \draw[Edge](4)--(3);
        \draw[Edge](5)--(1);
        \draw[Edge](6)--(5);
        \draw[Edge](8)--(9);
        \draw[Edge](9)--(7);
        \node(r)at(7.00,2){};
        \draw[Edge](r)--(7);
    \end{tikzpicture}
    \enspace \Rew \enspace
        \begin{tikzpicture}[xscale=.2,yscale=.19,Centering]
        \node(0)at(0.00,-4.33){};
        \node(10)at(10.00,-10.83){};
        \node(12)at(12.00,-10.83){};
        \node(2)at(2.00,-4.33){};
        \node(4)at(4.00,-4.33){};
        \node(6)at(6.00,-6.50){};
        \node(8)at(8.00,-8.67){};
        \node[NodeST](1)at(1.00,-2.17){\begin{math}\Product\end{math}};
        \node[NodeST](11)at(11.00,-8.67)
            {\begin{math}\Product\end{math}};
        \node[NodeST](3)at(3.00,0.00){\begin{math}\Product\end{math}};
        \node[NodeST](5)at(5.00,-2.17){\begin{math}\Product\end{math}};
        \node[NodeST](7)at(7.00,-4.33){\begin{math}\Product\end{math}};
        \node[NodeST](9)at(9.00,-6.50){\begin{math}\Product\end{math}};
        \draw[Edge](0)--(1);
        \draw[Edge](1)--(3);
        \draw[Edge](10)--(11);
        \draw[Edge](11)--(9);
        \draw[Edge](12)--(11);
        \draw[Edge](2)--(1);
        \draw[Edge](4)--(5);
        \draw[Edge](5)--(3);
        \draw[Edge](6)--(7);
        \draw[Edge](7)--(5);
        \draw[Edge](8)--(9);
        \draw[Edge](9)--(7);
        \node(r)at(3.00,1.75){};
        \draw[Edge](r)--(3);
    \end{tikzpicture}\hspace{-10pt}
\end{equation}
\end{minipage},
\begin{minipage}{7cm}
\begin{equation} \label{equ:rew_8}
    \begin{tikzpicture}[xscale=.2,yscale=.19,Centering]
        \node(0)at(0.00,-2.14){};
        \node(10)at(10.00,-12.86){};
        \node(12)at(12.00,-12.86){};
        \node(14)at(14.00,-12.86){};
        \node(2)at(2.00,-4.29){};
        \node(4)at(4.00,-6.43){};
        \node(6)at(6.00,-8.57){};
        \node(8)at(8.00,-12.86){};
        \node[NodeST](1)at(1.00,0.00){\begin{math}\Product\end{math}};
        \node[NodeST](11)at(11.00,-8.57)
            {\begin{math}\Product\end{math}};
        \node[NodeST](13)at(13.00,-10.71)
            {\begin{math}\Product\end{math}};
        \node[NodeST](3)at(3.00,-2.14){\begin{math}\Product\end{math}};
        \node[NodeST](5)at(5.00,-4.29){\begin{math}\Product\end{math}};
        \node[NodeST](7)at(7.00,-6.43){\begin{math}\Product\end{math}};
        \node[NodeST](9)at(9.00,-10.71){\begin{math}\Product\end{math}};
        \draw[Edge](0)--(1);
        \draw[Edge](10)--(9);
        \draw[Edge](11)--(7);
        \draw[Edge](12)--(13);
        \draw[Edge](13)--(11);
        \draw[Edge](14)--(13);
        \draw[Edge](2)--(3);
        \draw[Edge](3)--(1);
        \draw[Edge](4)--(5);
        \draw[Edge](5)--(3);
        \draw[Edge](6)--(7);
        \draw[Edge](7)--(5);
        \draw[Edge](8)--(9);
        \draw[Edge](9)--(11);
        \node(r)at(1.00,2){};
        \draw[Edge](r)--(1);
    \end{tikzpicture}
    \hspace{-10pt}\Rew
    \begin{tikzpicture}[xscale=.22,yscale=.2,Centering]
        \node(0)at(0.00,-1.88){};
        \node(10)at(10.00,-13.12){};
        \node(12)at(12.00,-13.12){};
        \node(14)at(14.00,-11.25){};
        \node(2)at(2.00,-3.75){};
        \node(4)at(4.00,-5.62){};
        \node(6)at(6.00,-7.50){};
        \node(8)at(8.00,-9.38){};
        \node[NodeST](1)at(1.00,0.00){\begin{math}\Product\end{math}};
        \node[NodeST](11)at(11.00,-11.25)
            {\begin{math}\Product\end{math}};
        \node[NodeST](13)at(13.00,-9.38)
            {\begin{math}\Product\end{math}};
        \node[NodeST](3)at(3.00,-1.88){\begin{math}\Product\end{math}};
        \node[NodeST](5)at(5.00,-3.75){\begin{math}\Product\end{math}};
        \node[NodeST](7)at(7.00,-5.62){\begin{math}\Product\end{math}};
        \node[NodeST](9)at(9.00,-7.50){\begin{math}\Product\end{math}};
        \draw[Edge](0)--(1);
        \draw[Edge](10)--(11);
        \draw[Edge](11)--(13);
        \draw[Edge](12)--(11);
        \draw[Edge](13)--(9);
        \draw[Edge](14)--(13);
        \draw[Edge](2)--(3);
        \draw[Edge](3)--(1);
        \draw[Edge](4)--(5);
        \draw[Edge](5)--(3);
        \draw[Edge](6)--(7);
        \draw[Edge](7)--(5);
        \draw[Edge](8)--(9);
        \draw[Edge](9)--(7);
        \node(r)at(1.00,2){};
        \draw[Edge](r)--(1);
    \end{tikzpicture}\hspace{-20pt}
\end{equation}
\end{minipage}, \\
\begin{minipage}{7cm}
\begin{equation} \label{equ:rew_9}
    \begin{tikzpicture}[xscale=0.18,yscale=.2,Centering]
        \node(0)at(0.00,-2.14){};
        \node(10)at(10.00,-12.86){};
        \node(12)at(12.00,-12.86){};
        \node(14)at(14.00,-10.71){};
        \node(2)at(2.00,-4.29){};
        \node(4)at(4.00,-6.43){};
        \node(6)at(6.00,-10.71){};
        \node(8)at(8.00,-10.71){};
        \node[NodeST](1)at(1.00,0.00){\begin{math}\Product\end{math}};
        \node[NodeST](11)at(11.00,-10.71)
            {\begin{math}\Product\end{math}};
        \node[NodeST](13)at(13.00,-8.57)
            {\begin{math}\Product\end{math}};
        \node[NodeST](3)at(3.00,-2.14){\begin{math}\Product\end{math}};
        \node[NodeST](5)at(5.00,-4.29){\begin{math}\Product\end{math}};
        \node[NodeST](7)at(7.00,-8.57){\begin{math}\Product\end{math}};
        \node[NodeST](9)at(9.00,-6.43){\begin{math}\Product\end{math}};
        \draw[Edge](0)--(1);
        \draw[Edge](10)--(11);
        \draw[Edge](11)--(13);
        \draw[Edge](12)--(11);
        \draw[Edge](13)--(9);
        \draw[Edge](14)--(13);
        \draw[Edge](2)--(3);
        \draw[Edge](3)--(1);
        \draw[Edge](4)--(5);
        \draw[Edge](5)--(3);
        \draw[Edge](6)--(7);
        \draw[Edge](7)--(9);
        \draw[Edge](8)--(7);
        \draw[Edge](9)--(5);
        \node(r)at(1.00,2){};
        \draw[Edge](r)--(1);
    \end{tikzpicture}
    \hspace{-5pt} \Rew \hspace{-5pt}
    \begin{tikzpicture}[xscale=.22,yscale=.22,Centering]
        \node(0)at(0.00,-1.88){};
        \node(10)at(10.00,-11.25){};
        \node(12)at(12.00,-13.12){};
        \node(14)at(14.00,-13.12){};
        \node(2)at(2.00,-3.75){};
        \node(4)at(4.00,-5.62){};
        \node(6)at(6.00,-7.50){};
        \node(8)at(8.00,-9.38){};
        \node[NodeST](1)at(1.00,0.00){\begin{math}\Product\end{math}};
        \node[NodeST](11)at(11.00,-9.38)
            {\begin{math}\Product\end{math}};
        \node[NodeST](13)at(13.00,-11.25)
            {\begin{math}\Product\end{math}};
        \node[NodeST](3)at(3.00,-1.88){\begin{math}\Product\end{math}};
        \node[NodeST](5)at(5.00,-3.75){\begin{math}\Product\end{math}};
        \node[NodeST](7)at(7.00,-5.62){\begin{math}\Product\end{math}};
        \node[NodeST](9)at(9.00,-7.50){\begin{math}\Product\end{math}};
        \draw[Edge](0)--(1);
        \draw[Edge](10)--(11);
        \draw[Edge](11)--(9);
        \draw[Edge](12)--(13);
        \draw[Edge](13)--(11);
        \draw[Edge](14)--(13);
        \draw[Edge](2)--(3);
        \draw[Edge](3)--(1);
        \draw[Edge](4)--(5);
        \draw[Edge](5)--(3);
        \draw[Edge](6)--(7);
        \draw[Edge](7)--(5);
        \draw[Edge](8)--(9);
        \draw[Edge](9)--(7);
        \node(r)at(1.00,1.5){};
        \draw[Edge](r)--(1);
    \end{tikzpicture}\hspace{-25pt}
\end{equation}
\end{minipage},
\begin{minipage}{7cm}
\begin{equation} \label{equ:rew_10}
    \hspace{-2pt}\begin{tikzpicture}[xscale=.2,yscale=.17,Centering]
        \node(0)at(0.00,-5.00){};
        \node(10)at(10.00,-12.50){};
        \node(12)at(12.00,-12.50){};
        \node(14)at(14.00,-12.50){};
        \node(2)at(2.00,-5.00){};
        \node(4)at(4.00,-5.00){};
        \node(6)at(6.00,-7.50){};
        \node(8)at(8.00,-12.50){};
        \node[NodeST](1)at(1.00,-2.50){\begin{math}\Product\end{math}};
        \node[NodeST](11)at(11.00,-7.50)
            {\begin{math}\Product\end{math}};
        \node[NodeST](13)at(13.00,-10.00)
            {\begin{math}\Product\end{math}};
        \node[NodeST](3)at(3.00,0.00){\begin{math}\Product\end{math}};
        \node[NodeST](5)at(5.00,-2.50){\begin{math}\Product\end{math}};
        \node[NodeST](7)at(7.00,-5.00){\begin{math}\Product\end{math}};
        \node[NodeST](9)at(9.00,-10.00){\begin{math}\Product\end{math}};
        \draw[Edge](0)--(1);
        \draw[Edge](1)--(3);
        \draw[Edge](10)--(9);
        \draw[Edge](11)--(7);
        \draw[Edge](12)--(13);
        \draw[Edge](13)--(11);
        \draw[Edge](14)--(13);
        \draw[Edge](2)--(1);
        \draw[Edge](4)--(5);
        \draw[Edge](5)--(3);
        \draw[Edge](6)--(7);
        \draw[Edge](7)--(5);
        \draw[Edge](8)--(9);
        \draw[Edge](9)--(11);
        \node(r)at(3.00,2){};
        \draw[Edge](r)--(3);
    \end{tikzpicture}
    \hspace{-10pt} \Rew \hspace{5pt}
    \begin{tikzpicture}[xscale=.21,yscale=.2,Centering]
        \node(0)at(0.00,-4.29){};
        \node(10)at(10.00,-12.86){};
        \node(12)at(12.00,-12.86){};
        \node(14)at(14.00,-10.71){};
        \node(2)at(2.00,-4.29){};
        \node(4)at(4.00,-4.29){};
        \node(6)at(6.00,-6.43){};
        \node(8)at(8.00,-8.57){};
        \node[NodeST](1)at(1.00,-2.14){\begin{math}\Product\end{math}};
        \node[NodeST](11)at(11.00,-10.71)
            {\begin{math}\Product\end{math}};
        \node[NodeST](13)at(13.00,-8.57)
            {\begin{math}\Product\end{math}};
        \node[NodeST](3)at(3.00,0.00){\begin{math}\Product\end{math}};
        \node[NodeST](5)at(5.00,-2.14){\begin{math}\Product\end{math}};
        \node[NodeST](7)at(7.00,-4.29){\begin{math}\Product\end{math}};
        \node[NodeST](9)at(9.00,-6.43){\begin{math}\Product\end{math}};
        \draw[Edge](0)--(1);
        \draw[Edge](1)--(3);
        \draw[Edge](10)--(11);
        \draw[Edge](11)--(13);
        \draw[Edge](12)--(11);
        \draw[Edge](13)--(9);
        \draw[Edge](14)--(13);
        \draw[Edge](2)--(1);
        \draw[Edge](4)--(5);
        \draw[Edge](5)--(3);
        \draw[Edge](6)--(7);
        \draw[Edge](7)--(5);
        \draw[Edge](8)--(9);
        \draw[Edge](9)--(7);
        \node(r)at(3.00,2){};
        \draw[Edge](r)--(3);
    \end{tikzpicture} \hspace{-25pt}
\end{equation}
\end{minipage}, \\
\begin{minipage}{8.8cm}
\begin{equation} \label{equ:rew_11}
    \begin{tikzpicture}[xscale=.22,yscale=.18,Centering]
        \node(0)at(0.00,-5.00){};
        \node(10)at(10.00,-10.00){};
        \node(12)at(12.00,-12.50){};
        \node(14)at(14.00,-12.50){};
        \node(2)at(2.00,-7.50){};
        \node(4)at(4.00,-7.50){};
        \node(6)at(6.00,-5.00){};
        \node(8)at(8.00,-7.50){};
        \node[NodeST](1)at(1.00,-2.50){\begin{math}\Product\end{math}};
        \node[NodeST](11)at(11.00,-7.50)
            {\begin{math}\Product\end{math}};
        \node[NodeST](13)at(13.00,-10.00)
            {\begin{math}\Product\end{math}};
        \node[NodeST](3)at(3.00,-5.00){\begin{math}\Product\end{math}};
        \node[NodeST](5)at(5.00,0.00){\begin{math}\Product\end{math}};
        \node[NodeST](7)at(7.00,-2.50){\begin{math}\Product\end{math}};
        \node[NodeST](9)at(9.00,-5.00){\begin{math}\Product\end{math}};
        \draw[Edge](0)--(1);
        \draw[Edge](1)--(5);
        \draw[Edge](10)--(11);
        \draw[Edge](11)--(9);
        \draw[Edge](12)--(13);
        \draw[Edge](13)--(11);
        \draw[Edge](14)--(13);
        \draw[Edge](2)--(3);
        \draw[Edge](3)--(1);
        \draw[Edge](4)--(3);
        \draw[Edge](6)--(7);
        \draw[Edge](7)--(5);
        \draw[Edge](8)--(9);
        \draw[Edge](9)--(7);
        \node(r)at(5.00,2){};
        \draw[Edge](r)--(5);
    \end{tikzpicture}
    \enspace \Rew \enspace
    \begin{tikzpicture}[xscale=.21,yscale=.19,Centering]
        \node(0)at(0.00,-4.29){};
        \node(10)at(10.00,-12.86){};
        \node(12)at(12.00,-12.86){};
        \node(14)at(14.00,-10.71){};
        \node(2)at(2.00,-4.29){};
        \node(4)at(4.00,-4.29){};
        \node(6)at(6.00,-6.43){};
        \node(8)at(8.00,-8.57){};
        \node[NodeST](1)at(1.00,-2.14){\begin{math}\Product\end{math}};
        \node[NodeST](11)at(11.00,-10.71)
            {\begin{math}\Product\end{math}};
        \node[NodeST](13)at(13.00,-8.57)
            {\begin{math}\Product\end{math}};
        \node[NodeST](3)at(3.00,0.00){\begin{math}\Product\end{math}};
        \node[NodeST](5)at(5.00,-2.14){\begin{math}\Product\end{math}};
        \node[NodeST](7)at(7.00,-4.29){\begin{math}\Product\end{math}};
        \node[NodeST](9)at(9.00,-6.43){\begin{math}\Product\end{math}};
        \draw[Edge](0)--(1);
        \draw[Edge](1)--(3);
        \draw[Edge](10)--(11);
        \draw[Edge](11)--(13);
        \draw[Edge](12)--(11);
        \draw[Edge](13)--(9);
        \draw[Edge](14)--(13);
        \draw[Edge](2)--(1);
        \draw[Edge](4)--(5);
        \draw[Edge](5)--(3);
        \draw[Edge](6)--(7);
        \draw[Edge](7)--(5);
        \draw[Edge](8)--(9);
        \draw[Edge](9)--(7);
        \node(r)at(3.00,2){};
        \draw[Edge](r)--(3);
    \end{tikzpicture}
\end{equation}
\end{minipage}.


\noindent
We claim that the rewrite relation $\RewContext$ induced by
rewrite rule $\Rew$ satisfying~\eqref{equ:rew_1}, \eqref{equ:rew_2}, 
\eqref{equ:rew_3}---\eqref{equ:rew_11} is convergent. First, for every 
relation $\Tfr \Rew \Tfr'$, we have
$\PrefixWord(\Tfr) > \PrefixWord(\Tfr')$. Therefore, by
Lemma~\ref{lem:prefix_word_termination}, $\RewContext$ is terminating.
Moreover, the greatest degree of a tree appearing in $\Rew$ is~$7$ so
that, from Lemma~\ref{lem:degree_confluence}, to show that $\RewContext$
is convergent, it is enough to prove that each tree of degree at most
$13$ admits exactly one normal form. Equivalently, this amounts to
show that the number of normal forms of trees of arity $n$ is equal
to $\#\CAs{3}(n)$. By computer exploration, we get the same sequence
\begin{equation} \label{equ:dimensions_CAs_3}
    1, 1, 2, 4, 8, 14, 20, 19, 16, 14, 14, 15, 16, 17
\end{equation}
for $\#\CAs{3}(n)$ and for the numbers of normal forms of arity $n$,
when $ 1 \leq n \leq 14$. Hence, we get our following main result.

\begin{Theorem} \label{thm:convergent_rewrite_rule_CAs_3}
    The rewrite rule $\Rew$ satisfying~\eqref{equ:rew_1},
    \eqref{equ:rew_2}, \eqref{equ:rew_3}---\eqref{equ:rew_11} is a
    convergent orientation of the congruence $\CongrCAs{3}$
    of~$\CAs{3}$.
\end{Theorem}

The rewrite rule $\Rew$ has, arity by arity, the cardinalities
\begin{equation}
    0, 0, 0, 1, 1, 2, 3, 4, 0, \dots~.
\end{equation}
We obtain from Theorem~\ref{thm:convergent_rewrite_rule_CAs_3} also
the following consequences.

\begin{Proposition} \label{prop:PBW_basis_CAs_3}
    The set of the trees avoiding as subtrees the ones appearing as
    left members of $\Rew$ is a Poincaré-Birkhoff-Witt basis
    of~$\CAs{3}$.
\end{Proposition}

From Proposition~\ref{prop:PBW_basis_CAs_3}, and by using a result
of~\cite{Gir18} describing a system of equations for the generating
series of syntax trees avoiding some sets of subtrees, we obtain the
following result.

\begin{Proposition} \label{prop:Hilbert_series_CAs_3}
    The Hilbert series of $\CAs{3}$ is
    \begin{equation} \label{equ:Hilbert_series_CAs_3}
        \HilbertSeries_{\CAs{3}}(t) = \frac{t}{(1 - t)^2}
        \left(1 - t + t^2 + t^3 + 2t^4 + 2t^5 - 7t^7 - 2t^8 + t^9 +
        2t^{10} + t^{11}\right).
    \end{equation}
\end{Proposition}

For $n \leq 10$, the dimensions of $\CAs{3}(n)$ are provided by
Sequence~\eqref{equ:dimensions_CAs_3} and for all $n \geq 11$, the
Taylor expansion of~\eqref{equ:Hilbert_series_CAs_3} shows that
$\# \CAs{3}(n) = n + 3$.

%%%%%%%%%%%%%%%%%%%%%%%%%%%%%%%%%%%%%%%%%%%%%%%%%%%%%%%%%%%%%%%%%%%%%%%%
%%%%%%%%%%%%%%%%%%%%%%%%%%%%%%%%%%%%%%%%%%%%%%%%%%%%%%%%%%%%%%%%%%%%%%%%
%%%%%%%%%%%%%%%%%%%%%%%%%%%%%%%%%%%%%%%%%%%%%%%%%%%%%%%%%%%%%%%%%%%%%%%%
\section*{Perspectives}
Our first axis of perspectives consists in collecting properties about
the operads $\CAs{d}$. A natural question consists in finding all the
morphisms between the operads $\CAs{d}$. Some surjective morphisms are
described by Proposition~\ref{prop:quotients_CAs_d} and we can hope to a
full description of these, as well as some possible injections.
Moreover, we can try to obtain a convergent orientation of
$\CongrCAs{d}$ and general expressions of the Hilbert series of
$\CAs{d}$ when $d \geq 4$. By computer exploration, we have the sequence
\begin{equation}
    1, 1, 2, 5, 13, 35, 96, 264, 724, 1973, 5355, 14390
\end{equation}
for the first dimensions for $\CAs{4}$. By applying the Buchberger
algorithm on trees of degrees until $10$, we obtain that a convergent
orientation of $\CongrCAs{4}$ has, arity by arity, the sequence
\begin{math}
    0, 0, 0, 0, 1, 1, 0, 3, 4, 5, 18, 22
\end{math}
for its first cardinalities. Moreover, for $\CAs{5}$, we get the
sequence
\begin{equation}
    1, 1, 2, 5, 14, 41, 124, 384, 1210, 3861, 12440
\end{equation}
of dimensions and the first cardinalities
\begin{math}
    0, 0, 0, 0, 0, 1, 1, 0, 0, 4, 5
\end{math}
for any convergent orientation of $\CongrCAs{5}$. Finally, for
$\CAs{6}$, we get the sequence
\begin{equation}
    1, 1, 2, 5, 14, 42, 131, 420, 1375, 4576, 15431
\end{equation}
of dimensions and the first cardinalities
\begin{math}
    0, 0, 0, 0, 0, 0, 1, 1, 0, 0, 0
\end{math}
for any convergent orientation of $\CongrCAs{6}$. We can notice that
only $\CAs{3}$ seems to have oscillating first dimensions.

A second axis concerns a complete understanding of $\CAs{3}$. We can
try to construct an explicit basis of this operad.
Proposition~\ref{prop:PBW_basis_CAs_3} describes a basis in terms of
trees avoiding some patterns but, we can hope to find a simpler
description. This includes the description of a family of combinatorial
objects forming a basis of $\CAs{3}$ and an adequate definition of a
partial composition map $\circ_i$ on these. Moreover, a natural
question is to explore the suboperads $\CAs{3}$ in the category of
vector spaces.

In a last axis, we can consider further generalizations of $\As$ being
quotients of $\Mag$ by congruences defined by identifying certain binary
trees of a same fixed degree. A possible question is, as presented in
the introduction, to investigate if combinatorial properties of the
trees belonging to a same equivalence class imply algebraic properties
on the obtained operads.

%%%%%%%%%%%%%%%%%%%%%%%%%%%%%%%%%%%%%%%%%%%%%%%%%%%%%%%%%%%%%%%%%%%%%%%%
%%%%%%%%%%%%%%%%%%%%%%%%%%%%%%%%%%%%%%%%%%%%%%%%%%%%%%%%%%%%%%%%%%%%%%%%
%%%%%%%%%%%%%%%%%%%%%%%%%%%%%%%%%%%%%%%%%%%%%%%%%%%%%%%%%%%%%%%%%%%%%%%%
\bibliographystyle{alpha}
\bibliography{Bibliography}

\end{document}
