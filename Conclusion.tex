%%%%%%%%%%%%%%%%%%%%%%%%%%%%%%%%%%%%%%%%%%%%%%%%%%%%%%%%%%%%%%%%%%%%%%%%
%%%%%%%%%%%%%%%%%%%%%%%%%%%%%%%%%%%%%%%%%%%%%%%%%%%%%%%%%%%%%%%%%%%%%%%%
%%%%%%%%%%%%%%%%%%%%%%%%%%%%%%%%%%%%%%%%%%%%%%%%%%%%%%%%%%%%%%%%%%%%%%%%
\section*{Perspectives}

\Todo{S: parler aussi des généralisation avec $2\Mag$ (arbres de Schröder)}

Our first axis of perspectives consists in finding convergent
presentations for $\CAs{d}$ operads in order to describe algebraic and
combinatorial properties of these operads, \emph{e.g.} find explicit bases,
compute Hilbert series, provide realizations, \emph{ect}... By computer
exploration, we have already some informations in this direction: we have
the sequence
\begin{equation}
    1, 1, 2, 5, 13, 35, 96, 264, 724, 1973, 5335, 14390, 38872, 105141, 
    284929, 774254
\end{equation}
for the first dimensions for $\CAs{4}$. By applying the Buchberger
algorithm on trees of degrees until $10$, we obtain that a convergent
orientation of $\CongrCAs{4}$ has, arity by arity, the sequence
\begin{math}
    0, 0, 0, 0, 1, 1, 0, 3, 4, 5, 18, 22
\end{math}
for its first cardinalities. Moreover, for $\CAs{5}$, we get the
sequence
\begin{equation}
1, 1, 2, 5, 14, 41, 124, 384, 1210, 3861, 12440, 40392, 131997, 
433782, 1432696, 4752857
\end{equation}
of dimensions and the first cardinalities
\begin{math}
    0, 0, 0, 0, 0, 1, 1, 0, 0, 4, 5
\end{math}
for any convergent orientation of $\CongrCAs{5}$. Finally, for
$\CAs{6}$, we get the sequence
\begin{equation}
   1, 1, 2, 5, 14, 42, 131, 420, 1375, 4576, 15431, 52598, 
   180895, 626862, 2186504, 7670138
\end{equation}
of dimensions and the first cardinalities
\begin{math}
    0, 0, 0, 0, 0, 0, 1, 1, 0, 0, 0
\end{math}
for any convergent orientation of $\CongrCAs{6}$. We can notice that
only $\CAs{3}$ seems to have oscillating first dimensions. 
\medbreak

Following ideas existing for word rewriring~\cite{GGM15}, a second axis
consists in allowing new generators in the orientations of $\CongrCAs{d}$
in order to obtain finite convergent presentations for $d\geq 4$. More
generally, we also would like to use this method for other magmatic
quotients, such as the operad $\Mag^{\{3, 4\}}$ that we did not study
entirely in Section~\ref{sec:MAg_3}.
\medbreak

A third axis for studying completion of magmatic operads is to use
the lattice structure introduced in Section~\ref{sec:Magmatic_operads}.
More precisely, assuming that we have completed the magmatic operads
$\Oca_1$, $\Oca_2$ as well as the lower-bound $\Oca_2\InfQMag \Oca_2$,
is there a procedure for completing the upper-bound
$\Oca_1\SupQMag \Oca_2$? Of course, the same question also makes sense
for the lattice of comb associative operads introduced in
Section~\ref{sec:CAs_d}.
\medbreak

\Todo{C : Plus développer les aspects treillis de Tamari ?}
In a last axis, we can consider further generalizations of $\As$ being
quotients of $\Mag$ by congruences defined by identifying certain binary
trees of a same fixed degree. A possible question is, as presented in
the introduction, to investigate if combinatorial properties of the
trees belonging to a same equivalence class imply algebraic properties
on the obtained operads.
\medbreak

\Todo{S: Placer ces résultats expérimentaux dans la partie sur CAs}

A second axis concerns a complete understanding of $\CAs{3}$. We can
try to construct an explicit basis of this operad.
Proposition~\ref{prop:PBW_basis_CAs_3} describes a basis in terms of
trees avoiding some patterns but, we can hope to find a simpler
description. This includes the description of a family of combinatorial
objects forming a basis of $\CAs{3}$ and an adequate definition of a
partial composition map $\circ_i$ on these. Moreover, a natural
question is to explore the suboperads $\CAs{3}$ in the category of
vector spaces.
\medbreak

