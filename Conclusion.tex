%%%%%%%%%%%%%%%%%%%%%%%%%%%%%%%%%%%%%%%%%%%%%%%%%%%%%%%%%%%%%%%%%%%%%%%%
%%%%%%%%%%%%%%%%%%%%%%%%%%%%%%%%%%%%%%%%%%%%%%%%%%%%%%%%%%%%%%%%%%%%%%%%
%%%%%%%%%%%%%%%%%%%%%%%%%%%%%%%%%%%%%%%%%%%%%%%%%%%%%%%%%%%%%%%%%%%%%%%%
\section*{Conclusion and perspectives}
In this paper, we have considered some quotients of the magmatic operad
in both the linear and the set-theoretic frameworks. We focused mainly
our study on comb associative operads and collected properties by
using computer exploration and rewrite systems on trees. There are many
ways to extend this work. Here follow some few further research
directions.
\medbreak

A first research direction consists in finding convergent orientations
of presentations for (all or most of) the $\CAs{\gamma}$ operads in
order to describe algebraic and combinatorial properties of these
operads (as describing explicit bases, computing Hilbert
series, and providing combinatorial realizations). This has been
done only in the case $\gamma = 3$. For some other cases, we only have
conjectural and experimental data (see Section~\ref{?}).
\Todo{S: compléter avec la bonne partie une fois qu'on aura ajouté ces
données}.
\medbreak

Following ideas existing for word rewriting theory~\cite{GGM15}, a
second axis consists in allowing new generators in the orientations of
the operad congruence $\CongrCAs{\gamma}$ in order to obtain finite
convergent orientations for their presentations when $\gamma \geq 4$.
Indeed, the Buchberger semi-algorithm works by adding rewrite rules to a
set of rewrite rules to obtain a convergent rewrite system. An orthogonal
procedure consists rather in adding new generators (new labels for
internal nodes in the trees) in order to obtain convergent rewrite
systems. More generally, we also would like to use these ideas for other
magmatic quotients, such as the operad $\Mag^{\{3, 4\}}$ that we did not
study entirely in Section~\ref{sec:MAg_3}.
\medbreak

A third axis consists in studying if the completion of orientations of
presentations of quotients of the magmatic operad maintains links with
the lattice structure introduced in Section~\ref{sec:Magmatic_operads}.
More precisely, assuming that we have completed the orientations of
the presentations of the quotients $\Oca_1$ and $\Oca_2$ of $\Mag$,
as well as the one of the lower-bound $\Oca_2 \InfQMag \Oca_2$, the
question consists in designing an algorithm for computing a completion
of an orientation of the presentation of the upper-bound
$\Oca_1 \SupQMag \Oca_2$. Of course, the same question also makes sense
for the lattice of comb associative operads introduced in
Section~\ref{sec:CAs_d}.
\medbreak

Let us address now a perspective fitting more in a combinatorial
context. As mentioned in the introduction of this article, we suspect
that some combinatorial properties of quotients $\Mag/_{\Congr}$ of
$\Mag$ derive from properties of the equivalence relation generating the
operad congruence $\Congr$. More precisely, we would like to investigate
if, when this equivalence relation is a set of Tamari intervals (or is
closed by interval, or satisfies some other classical properties coming
from poset theory), one harvests a nice description of the Hilbert
series and of a combinatorial realization of $\Mag/_{\Congr}$.
\medbreak

A last research axis rely on the study on the $2$-magmatic operad
$\TwoMag$, that is, the free operad generated by two binary elements.
The analog of the associative operad in this context is the operad
$\TwoAs$ (see~\cite{LR06}), defined as the quotient of $\TwoMag$ by
the congruence saying that the two generators are associative. This
operad has a nice combinatorial realization in terms of alternating
bicolored Schröder trees. The question consists here in generalizing
our main results for the quotients of $\TwoMag$ and the generalizations
of $\TwoAs$ (definition of analog of comb associative operads and
study of its presentations).
\medbreak

\Todo{S: attention, on emploie souvent l'expression "presentation" et
"completion of presentation" mais on a jamais vraiment défini ce qu'est
une presentation et ce qu'est une orientation d'une complétion. C'est
facile, mais il faudrait le faire (préliminaires). Il faudrait d'ailleurs
utiliser plutôt l'expression "convergent orientation of a presentation"
plutôt que "convergent orientation".}
