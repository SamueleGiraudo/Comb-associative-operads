%%%%%%%%%%%%%%%%%%%%%%%%%%%%%%%%%%%%%%%%%%%%%%%%%%%%%%%%%%%%%%%%%%%%%%%%
%%%%%%%%%%%%%%%%%%%%%%%%%%%%%%%%%%%%%%%%%%%%%%%%%%%%%%%%%%%%%%%%%%%%%%%%
%%%%%%%%%%%%%%%%%%%%%%%%%%%%%%%%%%%%%%%%%%%%%%%%%%%%%%%%%%%%%%%%%%%%%%%%
\section*{Perspectives}

\Todo{S: parler aussi des généralisation avec $2\Mag$ (arbres de Schröder)}

Our first axis of perspectives consists in finding convergent
presentations for $\CAs{d}$ operads in order to describe algebraic and
combinatorial properties of these operads, \emph{e.g.} find explicit bases,
compute Hilbert series, provide realizations, \emph{ect}...



Following ideas existing for word rewriring~\cite{GGM15}, a second axis
consists in allowing new generators in the orientations of $\CongrCAs{d}$
in order to obtain finite convergent presentations for $d\geq 4$. More
generally, we also would like to use this method for other magmatic
quotients, such as the operad $\Mag^{\{3, 4\}}$ that we did not study
entirely in Section~\ref{sec:MAg_3}.
\medbreak

A third axis for studying completion of magmatic operads is to use
the lattice structure introduced in Section~\ref{sec:Magmatic_operads}.
More precisely, assuming that we have completed the magmatic operads
$\Oca_1$, $\Oca_2$ as well as the lower-bound $\Oca_2\InfQMag \Oca_2$,
is there a procedure for completing the upper-bound
$\Oca_1\SupQMag \Oca_2$? Of course, the same question also makes sense
for the lattice of comb associative operads introduced in
Section~\ref{sec:CAs_d}.
\medbreak

\Todo{C : Plus développer les aspects treillis de Tamari ?}
In a last axis, we can consider further generalizations of $\As$ being
quotients of $\Mag$ by congruences defined by identifying certain binary
trees of a same fixed degree. A possible question is, as presented in
the introduction, to investigate if combinatorial properties of the
trees belonging to a same equivalence class imply algebraic properties
on the obtained operads.
\medbreak

\Todo{S: Placer ces résultats expérimentaux dans la partie sur CAs}

A second axis concerns a complete understanding of $\CAs{3}$. We can
try to construct an explicit basis of this operad.
Proposition~\ref{prop:PBW_basis_CAs_3} describes a basis in terms of
trees avoiding some patterns but, we can hope to find a simpler
description. This includes the description of a family of combinatorial
objects forming a basis of $\CAs{3}$ and an adequate definition of a
partial composition map $\circ_i$ on these. Moreover, a natural
question is to explore the suboperads $\CAs{3}$ in the category of
vector spaces.
\medbreak
