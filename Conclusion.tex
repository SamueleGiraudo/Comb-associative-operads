%%%%%%%%%%%%%%%%%%%%%%%%%%%%%%%%%%%%%%%%%%%%%%%%%%%%%%%%%%%%%%%%%%%%%%%%
%%%%%%%%%%%%%%%%%%%%%%%%%%%%%%%%%%%%%%%%%%%%%%%%%%%%%%%%%%%%%%%%%%%%%%%%
%%%%%%%%%%%%%%%%%%%%%%%%%%%%%%%%%%%%%%%%%%%%%%%%%%%%%%%%%%%%%%%%%%%%%%%%
\section*{Perspectives}
Our first axis of perspectives consists in collecting properties about
the operads $\CAs{d}$. A natural question consists in finding all the
morphisms between the operads $\CAs{d}$. Some surjective morphisms are
described by Proposition~\ref{prop:quotients_CAs_d} and we can hope to a
full description of these, as well as some possible injections.
Moreover, we can try to obtain a convergent orientation of
$\CongrCAs{d}$ and general expressions of the Hilbert series of
$\CAs{d}$ when $d \geq 4$. By computer exploration, we have the sequence
\begin{equation}
    1, 1, 2, 5, 13, 35, 96, 264, 724, 1973, 5355, 14390
\end{equation}
for the first dimensions for $\CAs{4}$. By applying the Buchberger
algorithm on trees of degrees until $10$, we obtain that a convergent
orientation of $\CongrCAs{4}$ has, arity by arity, the sequence
\begin{math}
    0, 0, 0, 0, 1, 1, 0, 3, 4, 5, 18, 22
\end{math}
for its first cardinalities. Moreover, for $\CAs{5}$, we get the
sequence
\begin{equation}
    1, 1, 2, 5, 14, 41, 124, 384, 1210, 3861, 12440
\end{equation}
of dimensions and the first cardinalities
\begin{math}
    0, 0, 0, 0, 0, 1, 1, 0, 0, 4, 5
\end{math}
for any convergent orientation of $\CongrCAs{5}$. Finally, for
$\CAs{6}$, we get the sequence
\begin{equation}
    1, 1, 2, 5, 14, 42, 131, 420, 1375, 4576, 15431
\end{equation}
of dimensions and the first cardinalities
\begin{math}
    0, 0, 0, 0, 0, 0, 1, 1, 0, 0, 0
\end{math}
for any convergent orientation of $\CongrCAs{6}$. We can notice that
only $\CAs{3}$ seems to have oscillating first dimensions.
\medbreak

A second axis concerns a complete understanding of $\CAs{3}$. We can
try to construct an explicit basis of this operad.
Proposition~\ref{prop:PBW_basis_CAs_3} describes a basis in terms of
trees avoiding some patterns but, we can hope to find a simpler
description. This includes the description of a family of combinatorial
objects forming a basis of $\CAs{3}$ and an adequate definition of a
partial composition map $\circ_i$ on these. Moreover, a natural
question is to explore the suboperads $\CAs{3}$ in the category of
vector spaces.
\medbreak

In a last axis, we can consider further generalizations of $\As$ being
quotients of $\Mag$ by congruences defined by identifying certain binary
trees of a same fixed degree. A possible question is, as presented in
the introduction, to investigate if combinatorial properties of the
trees belonging to a same equivalence class imply algebraic properties
on the obtained operads.
\medbreak
