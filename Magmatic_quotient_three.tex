\section{Magmatic quotients of degree \texorpdfstring{$3$}{3}}
\label{sec:MAg_3}

%%%%%%%%%%%%%%%%%%%%%%% Draw accolade %%%%%%%%%%%%%%%%%%%%%%
%%%%%%%%%%%%%%%%%%%%%%%% Vincent V. %%%%%%%%%%%%%%%%%%%%%%%%

\makeatletter
\newcommand{\zLengthNAngle}[4]{%
  \coordinate (dum) at (#3);  \coordinate (ble) at (#4);
  \pgfmathanglebetweenpoints{\pgfpointanchor{dum}{center}}{
  \pgfpointanchor{ble}{center}}
  \global\let#2\pgfmathresult  
  \pgfpointdiff{\pgfpointanchor{dum}{center}}{
  \pgfpointanchor{ble}{center}}
  \pgfmathparse{veclen(\pgf@x,\pgf@y)/28.45274}  
  \global\let#1\pgfmathresult
}
\newcommand{\accolade}[7][]{
  \zLengthNAngle\alength\aangle{#5}{#6}
  \begin{scope}[shift={($(#5)+(90+\aangle:#2)$)},rotate=\aangle]
    \draw[#1] (-#4,0) arc(180:90:#3) -- (.5*\alength-#3,#3)arc(-90:0:#3)
    node[anchor=\aangle-90,scale=.6,#1] {#7}
      arc(-180:-90:#3) -- (\alength-#3+#4,#3) arc(90:0:#3);
  \end{scope}
}

%%%%%%%%%%%%%%%%%%%%% Tree of degree 3 %%%%%%%%%%%%%%%%%%%%%%%%%

\newcommand{\TreeA}{
\begin{tikzpicture}[xscale=.22,yscale=.23,Centering]
    \node(0)at(0.00,-5.25){};
    \node(2)at(2.00,-5.25){};
    \node(4)at(4.00,-3.50){};
    \node(6)at(6.00,-1.75){};
    \node[NodeST](1)at(1.00,-3.50){\begin{math}\Product\end{math}};
    \node[NodeST](3)at(3.00,-1.75){\begin{math}\Product\end{math}};
    \node[NodeST](5)at(5.00,0.00){\begin{math}\Product\end{math}};
    \draw[Edge](0)--(1);
    \draw[Edge](1)--(3);
    \draw[Edge](2)--(1);
    \draw[Edge](3)--(5);
    \draw[Edge](4)--(3);
    \draw[Edge](6)--(5);
    \node(r)at(5.00,1.31){};
    \draw[Edge](r)--(5);
\end{tikzpicture}}

\newcommand{\TreeB}{
\begin{tikzpicture}[xscale=.22,yscale=.23,Centering]
    \node(0)at(0.00,-3.50){};
    \node(2)at(2.00,-5.25){};
    \node(4)at(4.00,-5.25){};
    \node(6)at(6.00,-1.75){};
    \node[NodeST](1)at(1.00,-1.75){\begin{math}\Product\end{math}};
    \node[NodeST](3)at(3.00,-3.50){\begin{math}\Product\end{math}};
    \node[NodeST](5)at(5.00,0.00){\begin{math}\Product\end{math}};
    \draw[Edge](0)--(1);
    \draw[Edge](1)--(5);
    \draw[Edge](2)--(3);
    \draw[Edge](3)--(1);
    \draw[Edge](4)--(3);
    \draw[Edge](6)--(5);
    \node(r)at(5.00,1.31){};
    \draw[Edge](r)--(5);
\end{tikzpicture}}

\newcommand{\TreeC}{
\begin{tikzpicture}[xscale=.22,yscale=.23,Centering]
    \node(0)at(0.00,-4.67){};
    \node(2)at(2.00,-4.67){};
    \node(4)at(4.00,-4.67){};
    \node(6)at(6.00,-4.67){};
    \node[NodeST](1)at(1.00,-2.33){\begin{math}\Product\end{math}};
    \node[NodeST](3)at(3.00,0.00){\begin{math}\Product\end{math}};
    \node[NodeST](5)at(5.00,-2.33){\begin{math}\Product\end{math}};
    \draw[Edge](0)--(1);
    \draw[Edge](1)--(3);
    \draw[Edge](2)--(1);
    \draw[Edge](4)--(5);
    \draw[Edge](5)--(3);
    \draw[Edge](6)--(5);
    \node(r)at(3.00,1.75){};
    \draw[Edge](r)--(3);
\end{tikzpicture}}

\newcommand{\TreeD}{
\begin{tikzpicture}[xscale=.22,yscale=.23,Centering]
    \node(0)at(0.00,-1.75){};
    \node(2)at(2.00,-5.25){};
    \node(4)at(4.00,-5.25){};
    \node(6)at(6.00,-3.50){};
    \node[NodeST](1)at(1.00,0.00){\begin{math}\Product\end{math}};
    \node[NodeST](3)at(3.00,-3.50){\begin{math}\Product\end{math}};
    \node[NodeST](5)at(5.00,-1.75){\begin{math}\Product\end{math}};
    \draw[Edge](0)--(1);
    \draw[Edge](2)--(3);
    \draw[Edge](3)--(5);
    \draw[Edge](4)--(3);
    \draw[Edge](5)--(1);
    \draw[Edge](6)--(5);
    \node(r)at(1.00,1.31){};
    \draw[Edge](r)--(1);
\end{tikzpicture}}

\newcommand{\TreeE}{
\begin{tikzpicture}[xscale=.22,yscale=.23,Centering]
    \node(0)at(0.00,-1.75){};
    \node(2)at(2.00,-3.50){};
    \node(4)at(4.00,-5.25){};
    \node(6)at(6.00,-5.25){};
    \node[NodeST](1)at(1.00,0.00){\begin{math}\Product\end{math}};
    \node[NodeST](3)at(3.00,-1.75){\begin{math}\Product\end{math}};
    \node[NodeST](5)at(5.00,-3.50){\begin{math}\Product\end{math}};
    \draw[Edge](0)--(1);
    \draw[Edge](2)--(3);
    \draw[Edge](3)--(1);
    \draw[Edge](4)--(5);
    \draw[Edge](5)--(3);
    \draw[Edge](6)--(5);
    \node(r)at(1.00,1.31){};
    \draw[Edge](r)--(1);
\end{tikzpicture}}

%%%%%%%%%%%%%%%%% Normal forms of Mag(3,5) %%%%%%%%%%%%%%%%%%%%%%%%%

\newcommand{\Lightning}[1]{
\mathfrak{c}^{(#1)}_{\begin{tikzpicture}[scale=.1, Centering]  
    \draw (0,0)--(-1,-2);
    \draw (-1,-2)--(0,-2);
    \draw (0,-2)--(-1,-4);
\end{tikzpicture}}}

%%%%%%%%%%%%%%%%%%%%%%%%%%%%%%%%%%%%%%%%%%%%%%%%%%%%%%%%%%%%%%%%%%%%%

We now explore all magmatic simple quotients of degree $3$. We denote 
by $A_k$ the $k$-th tree of degree $3$ for lexicographic order. Thus 
\begin{equation}
A_1 := \TreeA,\, A_2 := \TreeB,\, A_3 := \TreeC,\, A_4 := \TreeD, 
\text{ and } A_5 := \TreeE.
\end{equation}
We denote by $\Mag(i,j)$ the operad quotient of $\Mag$ by the relation 
$A_i \Congr A_j$. We already studied the operad $\Mag(1,5) = \CAs{3}$ 
in section~\ref{sec:CAs_3}. Some of those operads are anti-isomorphic.
We said that $\Oca_1$ and $\Oca_2$ are anti-isomorphic if there exist a function (called anti-isomorphism)
$\phi : \Oca_1 \rightarrow \Oca_2$ such that 
\begin{equation}
 \phi(1) = 1 \text{ and } \phi(x \circ_i y) = 
 \phi(x) \circ_{|x|-i+1} \phi(y), \text{ for } x,y \in \Oca_1.
\end{equation}
We recall some basics properties of anti-isomorphic operads.
\begin{Proposition} \label{prop:Recall_anti_isomorphic}
If $\phi$ is an anti-isomorphism between $\Oca_1$ and $\Oca_2$ then
\begin{itemize}
\item $\HilbertSeries_{\Oca_1}(t) = \HilbertSeries_{\Oca_2}(t)$;
\item if $\left(\Oca, \left\{\circ_i\right\}_{i \geq 1}\right)$ is a realisation of $\Oca_1$ then 
$\left(\Oca, \left\{\overline{\circ_i}\right\}_{i \geq 1}\right)$ is a 
realisation of $\Oca_2$, where $\overline{\circ_i}$ is the operation
 defined by
\begin{equation}
x \, \overline{\circ_i} \, y := x \circ_{|x|-i+1} y;
\end{equation}
\item if $\{ x_i \rightarrow y_i \}_{1 \leq i \leq n}$ is a convergent
 orientation of $\Oca_1$ then $\{ \phi\left(x_i\right) \rightarrow 
 \phi\left(y_i\right) \}_{1 \leq i \leq n}$ is a convergent orientation
  of $\Oca_2$.
\end{itemize}
\end{Proposition}


\subsection{Operads \texorpdfstring{$\Mag(1,2)$}{Mag(1,2)} and 
\texorpdfstring{$\Mag(4,5)$}{Mag(4,5)}} \label{sub:Mag_1_2}
The reader can check that the rewriting rule $A_2 \Rew A_1$ is a 
convergent orientation of $\Mag(1,2)$. The operads $\Mag(1,2)$ and 
$\Mag(4,5)$ are anti-isomorphic, hence the rewriting rule $A_4 \Rew A_5$ 
is a convergent orientation of $\Mag(4,5)$. 

Like proposition~\eqref{prop:Hilbert_series_CAs_3} we compute the 
following results thanks to~\cite{Gir18}.
\begin{Proposition} \label{prop:Hilbert_series_Mag_1_2}
The Hilbert series of $\Mag(1,2)$ and $\Mag(4,5)$ is
\begin{equation} \label{equ:Hilbert_series_Mag_1_2}
\HilbertSeries_{\Mag(1,2)}(t) = \HilbertSeries_{\Mag(4,5)}(t) = 
t\frac{t-1}{2t -1}.
\end{equation}
\end{Proposition}

A Taylor expansion of series~\eqref{equ:Hilbert_series_Mag_1_2} shows 
that 
\begin{Proposition} \label{prop:Close_formula_Mag_1_2}
For all $n \geq 2,$
\begin{equation} \label{equ:Close_formula_Mag_1_2}
\# \Mag(1,2)(1) = \# \Mag(4,5)(1) = 1 \text{ and } \# \Mag(1,2)(n) = 
\# \Mag(4,5)(n) = 2^{n-2}.
\end{equation}
\end{Proposition}

Many combinatorial objects are graduate sets with the powers of two as 
cardinals. We choose to present a realisation of $\Mag(1,2)$ based on 
integer compositions. We denote by 
\begin{equation}
\mathcal{C} := \bigsqcup\limits_{n \geq 1} \mathcal{C}(n) = 
\bigsqcup\limits_{n \geq 1}\left\{ (a_1 , \dots, a_p), p \geq 0 
\text{ and } 1+\sum\limits_{k = 1}^{p} a_k = n \right\}
\end{equation}
the graduate set of integer compositions. Given an integer composition 
$a = (a_1 , \dots, a_p)$, we denote by $s_{(i,j)}(a)$ the number 
$1 + \sum\limits_{k = i}^{j} a_k$.

Given an integer $i \geq 1$, we define the binary operation 
$\circ_i^{(1,2)} : \mathcal{C}(n) \times \mathcal{C}(n') \rightarrow \mathcal{C}(n + n' -1)$ for 
$n \geq i$ and $n' \geq 1$ by
\begin{equation}
\left(a_1 , \dots, a_p \right) \circ_i^{(1,2)} 
\left(b_1, \dots ,b_{q}\right) := \left\{
 \begin{split}
 & \left(a_1, \dots ,a_{p}, b_1, \dots ,b_{q}\right) & 
    \text{if } i = n\\ 
 & \left(a_1, \dots, a_{k}, a_{k+1} + n'-1, a_{k+2}, \dots ,a_{p}\right)
  & \text{otherwise}
 \end{split}
\right.
\end{equation}
where $k \geq 0$ be such that $s_{(1,k)}(a) \leq i < s_{(1,k+1)}(a)$.

\begin{Proposition} \label{prop:Realisation_Mag_1_2}
The operad 
$\left(\mathcal{C}, \left\{\circ_i^{(1,2)}\right\}_{i \geq 1}\right)$ 
is a realisation of $\Mag(1,2)$.
\end{Proposition}
\begin{proof}
It's equivalent to show that there are isomorphic. The set of normal 
forms of arity $n$ for the rewriting rule $A_2 \Rew A_1$ is 
\begin{equation} \label{equ:Normal_forms_A2}
\left\{ \LComb{k_1 \, ; \, \dots \, ; \, k_p} \text{ for } 
\left(k_1, \dots, k_p\right) \in \mathcal{C}(n) \right\}
\end{equation}
where $\LComb{k_1 \, ; \, \dots \, ; \, k_p}$ is 
$\RComb{p} \circ \left[\LComb{k_1-1}; \dots ; \LComb{k_p-1} 
; | \, \right]$.
Thus the function 
$\phi : \Mag(1,2)(n) \rightarrow \mathcal{C}(n)$ defined by
\begin{equation} \label{equ:Morphism_A2}
\phi\left( \LComb{k_1 \, ; \, \dots \, ; \, k_p} \right) := 
\left(k_1 ; \dots ; k_p  \right)
\end{equation}
is a bijection. Let's show that $\phi$ is an operad morphism. Given 
\begin{equation}
a := \LComb{a_1 \, ; \, \dots \, ; \, a_p} \in \Mag(1,2)(n) \text{ and }
b := \LComb{b_1 \, ; \, \dots \, ; \, b_q} \in \Mag(1,2)(n'),
\end{equation}
we have that $a \circ_n b$ equals $\LComb{a_1 \, ; \, \dots \, ; \, a_p
 \, ; \, b_1 \, ; \, \dots \, ; \, b_q}$, thus 
 $\phi\left(a \circ_n b\right) = \phi\left(a\right) \circ_n^{(1,2)} 
 \phi\left(b\right)$. Let $1 \leq i < n$ and $k$ be such that 
 $s_{(1, k)}(a) \leq i < s_{(1, k+1)}(a)$ then $a \circ_i b$ equals 
\begin{equation} \label{equ:Rewriting_1_Mag_1_2}
\RComb{p} \circ \left[\LComb{a_1-1}; \dots ; \LComb{a_k-1}; 
\LComb{a_{k+1}-1}
\circ_{i-s_{(1,k)}(a)} b  ; \LComb{a_{k+2}-1} ; \dots; \LComb{a_p-1} ; 
| \, \right]
\end{equation}
that can be rewrite in one step into
\begin{equation}
\RComb{p} \circ \left[\LComb{a_1-1}; \dots ; \LComb{a_k-1}; \left( 
\LComb{a_{k+1}-1} \circ_{i-s_{(1,k)}(a)}  \LComb{b_1} \right) 
\circ_{i-s_{(1,k)}(a)+b_1} \LComb{b_2 \, ; \, \dots \, ; \, b_q} ; 
\LComb{a_{k+2}-1} ; \dots; \LComb{a_p-1} ; | \, \right]
\end{equation}
that can be rewrite in $b_1$ steps into
\begin{equation} \label{equ:Rewriting_2_Mag_1_2}
\RComb{p} \circ \left[\LComb{a_1-1}; \dots ; \LComb{a_k-1}; 
\LComb{a_{k+1}-1+b_1} \circ_{i-s_{(1,k)}(a)+b_1} 
\LComb{b_2 \, ; \, \dots \, ; \, b_q}  ; \LComb{a_{k+2}-1} ; \dots; 
\LComb{a_p-1} ; | \, \right]
\end{equation}
by iterating $q-1$ times the rewriting process 
$\eqref{equ:Rewriting_1_Mag_1_2} \rightarrow 
\eqref{equ:Rewriting_2_Mag_1_2}$ we get that 
\begin{equation}
a \circ_i b \rightarrow \LComb{a_1 \, ; \, \dots \, ; \, a_{k} \, ; \, 
a_{k+1} + n'-1 \, ; \, a_{k+2} \, ; \, \dots \, ; \, a_p}
\end{equation}
thus $\phi(a \circ_i b) = \phi(a) \circ_i^{(1,2)} \phi(b)$.
\end{proof}

We deduce from proposition~\ref{prop:Recall_anti_isomorphic} that the
 operad $\left(\mathcal{C}, 
 \left\{\overline{\circ_i^{(1,2)}}\right\}_{i \geq 1}\right)$ 
is a realisation of $\Mag(4,5)$.

\subsection{Operads \texorpdfstring{$\Mag(1,3)$}{Mag(1,3)} and 
\texorpdfstring{$\Mag(3,5)$}{Mag(3,5)}}
The reader can check that the rewriting rule $A_3 \Rew A_1$ is a 
convergent orientation of $\Mag(1,3)$. Symmetrically, we have that the 
rewriting rule $A_3 \Rew A_5$ is a convergent orientation of 
$\Mag(3,5)$.

Thanks to~\cite{Gir18}, we get that the Hilbert series of $\Mag(1,3)$ 
and $\Mag(3,5)$ are equals to~\eqref{equ:Hilbert_series_Mag_1_2}. Thus 
for $n \geq 1,$ $\# \Mag(1,3)(n)$ and $\# \Mag(3,5)(n)$ are equals 
to~\eqref{prop:Close_formula_Mag_1_2}. 

Like in subsection~\ref{sub:Mag_1_2}, we choose a realisation based on 
integer compositions. Given an integer $i \geq 1$, we define the 
binary operation 
$\circ_i^{(1,3)} : \mathcal{C}(n) \times \mathcal{C}(n')\rightarrow 
\mathcal{C}(n + n' -1)$ for 
$n \geq i$ and $n' \geq 1$ by
\begin{equation}
\left(a_1 , \dots, a_p\right) \circ_i^{(1,3)} 
\left(b_1, \dots ,b_{q}\right) := \left\{
 \begin{split}
 & \left(a_1, \dots ,a_{i-1}, b_1 + s_{(i,p)}(a),b_2,\dots,b_{q}\right) 
 & \text{if } i \leq p + 1\\ 
 & \left(a_1, \dots, a_{k-1}, a_{k} + n'-1, a_{k+1},\dots,a_{p}\right)
  & \text{otherwise}
 \end{split}
\right.
\end{equation}
where $k \geq 0$ be such that 
$k+1 + s_{(k+1,p)}(a) < i \leq k + s_{(k,p)}(a)$.

\begin{Proposition} \label{prop:Realisation_Mag_1_3}
The operads 
$\left(\mathcal{C}, \left\{\circ_i^{(1,3)}\right\}_{i \geq 1}\right)$ 
and
$\left(\mathcal{C}, \left\{\overline{\circ_i^{(1,3)}}\right\}_{i \geq 1}
\right)$ are respectively realisations of $\Mag(1,3)$ and $\Mag(3,5)$.
\end{Proposition}
\begin{proof}
The set of normal forms of arity $n$ for the rewriting rule 
$A_3 \Rew A_1$ is
\begin{equation}
\left\{ \Lightning{k_1 \, ; \, \dots \, ; \, k_p} \text{ for } 
\left(k_1, \dots, k_p\right) \in \mathcal{C}(n) \right\}
\end{equation}
where $\Lightning{k_1 \, ; \, \dots \, ; \, k_p}$ is 
$\LComb{k_1} \circ_2 \left( \LComb{k_2} \circ_2 \left( \cdots \left( 
\LComb{k_{p-1}} \circ_2 \LComb{k_p} \right) \cdots \right) \right)$. 
Thus we can show that the function 
$\phi : \Mag(1,3)(n) \rightarrow \mathcal{C}(n)$ defined by
\begin{equation}
\phi\left( \Lightning{k_1 \, ; \, \dots \, ; \, k_p} \right) := 
\left(k_1 ; \dots ; k_p \right)
\end{equation}
is an operad isomorphism from the operad $\Mag(1,3)$ to 
$\left(\mathcal{C}, \left\{\circ_i^{(1,3)}\right\}_{i \geq 1}\right)$. 
\end{proof}

\subsection{Operads \texorpdfstring{$\Mag(1,4)$}{Mag(1,4)} and 
\texorpdfstring{$\Mag(2,5)$}{Mag(2,5)}}
The rewriting rule $A_4 \Rew A_1$ is a convergent orientation of 
$\Mag(1,4)$. Symmetrically the rewriting rule $A_2 \Rew A_5$ is a 
convergent orientation of $\Mag(2,5)$.

Thanks to~\cite{Gir18}, we get that the Hilbert series of $\Mag(1,4)$
 and $\Mag(2,5)$ are equals to~\eqref{equ:Hilbert_series_Mag_1_2}. 
Thus for $n \geq 1,$ $\# \Mag(1,4)(n)$ and $\# \Mag(2,5)(n)$ are equals 
to~\eqref{prop:Close_formula_Mag_1_2}. 

Given an integer $i \geq 1$, we define the binary operation 
$\circ_i^{(2,5)} : \mathcal{C}(n) \times \mathcal{C}(n')
\rightarrow \mathcal{C}(n + n' -1)$ for $n \geq i$ and $n' \geq 1$ by
\begin{equation}
a \circ_i^{(2,5)} b := \left\{
    \begin{split}
    & (a_1, \dots, a_{k}, b_1, \dots, b_{q}, a_{k+1}, \dots ,
    a_{p}) & \text{if } i = s_k\\ 
    & (a_1, \dots, a_{k}, i - s_{k}, b_1, \dots, b_{q-1}, 
    b_{q} + s_{k+1} - i, a_{k+2}, \dots ,a_{p}) & \text{otherwise}
    \end{split}
  \right.
\end{equation}
where $k \geq 0$ be such that $s_k \leq i < s_{k+1}$. 

\begin{Proposition} \label{prop:Realisation_Mag_1_4}
The operads 
$\left(\mathcal{C}, \left\{\overline{\circ_i^{(2,5)}}\right\}_{i \geq 1}
\right)$ and 
$\left(\mathcal{C}, \left\{\circ_i^{(2,5)}\right\}_{i \geq 1}\right)$ 
are respectively realisation of $\Mag(1,4)$ and $\Mag(2,5)$.
\end{Proposition}

\begin{proof}
The set of normal forms of arity $n$ for the rewriting rule 
$A_2 \Rew A_5$ is~\eqref{equ:Normal_forms_A2}. Thus we can show that
 the function~\eqref{equ:Morphism_A2}
is an operad isomorphism from the operad $\Mag(2,5)$ to 
$\left(\mathcal{C}, \left\{\circ_i^{(2,5)}\right\}_{i \geq 1}\right)$. 
\end{proof}

\subsection{Operads \texorpdfstring{$\Mag(2,3)$}{Mag(2,3)} and 
\texorpdfstring{$\Mag(3,4)$}{Mag(3,4)}}
We did not found a finite, terminating and confluent orientation  
for these operads. However, thanks to computer exploration, we 
conjecture that the following rules form a terminating and confluent 
orientation of $\Mag(3,4)$:
\begin{equation}\label{eq:Rules_Mag_3_4}
\begin{split}
A_4 \Rew A_3, 
\begin{tikzpicture}[xscale=.22,yscale=.23,Centering]
    \node(0)at(0.00,-8.25){};
    \node(10)at(10.00,-8.25){};
    \node(2)at(2.00,-8.25){};
    \node(4)at(4.00,-5.50){};
    \node(6)at(6.00,-5.50){};
    \node(8)at(8.00,-8.25){};
    \node[NodeST](1)at(1.00,-5.50){\begin{math}\Product\end{math}};
    \node[NodeST](3)at(3.00,-2.75){\begin{math}\Product\end{math}};
    \node[NodeST](5)at(5.00,0.00){\begin{math}\Product\end{math}};
    \node[NodeST](7)at(7.00,-2.75){\begin{math}\Product\end{math}};
    \node[NodeST](9)at(9.00,-5.50){\begin{math}\Product\end{math}};
    \draw[Edge](0)--(1);
    \draw[Edge](1)--(3);
    \draw[Edge](10)--(9);
    \draw[Edge](2)--(1);
    \draw[Edge](3)--(5);
    \draw[Edge](4)--(3);
    \draw[Edge](6)--(7);
    \draw[Edge](7)--(5);
    \draw[Edge](8)--(9);
    \draw[Edge](9)--(7);
    \node(r)at(5.00,2.06){};
    \draw[Edge](r)--(5);
\end{tikzpicture}  \Rew 
\begin{tikzpicture}[xscale=.22,yscale=.23,Centering]
    \node(0)at(0.00,-5.50){};
    \node(10)at(10.00,-8.25){};
    \node(2)at(2.00,-8.25){};
    \node(4)at(4.00,-8.25){};
    \node(6)at(6.00,-5.50){};
    \node(8)at(8.00,-8.25){};
    \node[NodeST](1)at(1.00,-2.75){\begin{math}\Product\end{math}};
    \node[NodeST](3)at(3.00,-5.50){\begin{math}\Product\end{math}};
    \node[NodeST](5)at(5.00,0.00){\begin{math}\Product\end{math}};
    \node[NodeST](7)at(7.00,-2.75){\begin{math}\Product\end{math}};
    \node[NodeST](9)at(9.00,-5.50){\begin{math}\Product\end{math}};
    \draw[Edge](0)--(1);
    \draw[Edge](1)--(5);
    \draw[Edge](10)--(9);
    \draw[Edge](2)--(3);
    \draw[Edge](3)--(1);
    \draw[Edge](4)--(3);
    \draw[Edge](6)--(7);
    \draw[Edge](7)--(5);
    \draw[Edge](8)--(9);
    \draw[Edge](9)--(7);
    \node(r)at(5.00,2.06){};
    \draw[Edge](r)--(5);
\end{tikzpicture},
\begin{tikzpicture}[xscale=.22,yscale=.23,Centering]
    \node(0)at(0.00,-10.40){};
    \node(10)at(10.00,-5.20){};
    \node(12)at(12.00,-5.20){};
    \node(2)at(2.00,-10.40){};
    \node(4)at(4.00,-7.80){};
    \node(6)at(6.00,-7.80){};
    \node(8)at(8.00,-7.80){};
    \node[NodeST](1)at(1.00,-7.80){\begin{math}\Product\end{math}};
    \node[NodeST](11)at(11.00,-2.60){\begin{math}\Product\end{math}};
    \node[NodeST](3)at(3.00,-5.20){\begin{math}\Product\end{math}};
    \node[NodeST](5)at(5.00,-2.60){\begin{math}\Product\end{math}};
    \node[NodeST](7)at(7.00,-5.20){\begin{math}\Product\end{math}};
    \node[NodeST](9)at(9.00,0.00){\begin{math}\Product\end{math}};
    \draw[Edge](0)--(1);
    \draw[Edge](1)--(3);
    \draw[Edge](10)--(11);
    \draw[Edge](11)--(9);
    \draw[Edge](12)--(11);
    \draw[Edge](2)--(1);
    \draw[Edge](3)--(5);
    \draw[Edge](4)--(3);
    \draw[Edge](5)--(9);
    \draw[Edge](6)--(7);
    \draw[Edge](7)--(5);
    \draw[Edge](8)--(7);
    \node(r)at(9.00,1.95){};
    \draw[Edge](r)--(9);
\end{tikzpicture}  \Rew 
\begin{tikzpicture}[xscale=.22,yscale=.23,Centering]
    \node(0)at(0.00,-7.80){};
    \node(10)at(10.00,-5.20){};
    \node(12)at(12.00,-5.20){};
    \node(2)at(2.00,-7.80){};
    \node(4)at(4.00,-7.80){};
    \node(6)at(6.00,-10.40){};
    \node(8)at(8.00,-10.40){};
    \node[NodeST](1)at(1.00,-5.20){\begin{math}\Product\end{math}};
    \node[NodeST](11)at(11.00,-2.60){\begin{math}\Product\end{math}};
    \node[NodeST](3)at(3.00,-2.60){\begin{math}\Product\end{math}};
    \node[NodeST](5)at(5.00,-5.20){\begin{math}\Product\end{math}};
    \node[NodeST](7)at(7.00,-7.80){\begin{math}\Product\end{math}};
    \node[NodeST](9)at(9.00,0.00){\begin{math}\Product\end{math}};
    \draw[Edge](0)--(1);
    \draw[Edge](1)--(3);
    \draw[Edge](10)--(11);
    \draw[Edge](11)--(9);
    \draw[Edge](12)--(11);
    \draw[Edge](2)--(1);
    \draw[Edge](3)--(9);
    \draw[Edge](4)--(5);
    \draw[Edge](5)--(3);
    \draw[Edge](6)--(7);
    \draw[Edge](7)--(5);
    \draw[Edge](8)--(7);
    \node(r)at(9.00,1.95){};
    \draw[Edge](r)--(9);
\end{tikzpicture},\\
\begin{tikzpicture}[xscale=.22,yscale=.23,Centering]
    \node(0)at(0.00,-8.67){};
    \node(10)at(10.00,-4.33){};
    \node(12)at(12.00,-4.33){};
    \node(2)at(2.00,-10.83){};
    \node(4)at(4.00,-10.83){};
    \node(6)at(6.00,-6.50){};
    \node(8)at(8.00,-4.33){};
    \node[NodeST](1)at(1.00,-6.50){\begin{math}\Product\end{math}};
    \node[NodeST](11)at(11.00,-2.17){\begin{math}\Product\end{math}};
    \node[NodeST](3)at(3.00,-8.67){\begin{math}\Product\end{math}};
    \node[NodeST](5)at(4.00,-5.33){\begin{math}\Product\end{math}};
    \node[NodeST](7)at(7.00,-2.17){\begin{math}\Product\end{math}};
    \node[NodeST](9)at(9.00,0.00){\begin{math}\Product\end{math}};
    \draw[Edge](0)--(1);
    \draw[Edge](1)--(5);
    \draw[Edge](10)--(11);
    \draw[Edge](11)--(9);
    \draw[Edge](12)--(11);
    \draw[Edge](2)--(3);
    \draw[Edge](3)--(1);
    \draw[Edge](4)--(3);
    \draw[Edge, dashed](5)--(7);
    \draw[Edge](6)--(5);
    \draw[Edge](7)--(9);
    \draw[Edge](8)--(7);
    \node(r)at(9.00,1.62){};
    \draw[Edge](r)--(9);    
    \accolade{0.5pt}{0.3}{0.2}{3.70,-5.33}{6.70,-1.90}{\LARGE $k$}
\end{tikzpicture} \Rew
\begin{tikzpicture}[xscale=.22,yscale=.23,Centering]
    \node(0)at(0.00,-7.80){};
    \node(10)at(10.00,-5.20){};
    \node(12)at(12.00,-5.20){};
    \node(2)at(2.00,-7.80){};
    \node(4)at(4.00,-7.80){};
    \node(6)at(6.00,-10.40){};
    \node(8)at(8.00,-10.40){};
    \node[NodeST](1)at(1.00,-5.20){\begin{math}\Product\end{math}};
    \node[NodeST](11)at(11.00,-2.60){\begin{math}\Product\end{math}};
    \node[NodeST](3)at(3.00,-2.60){\begin{math}\Product\end{math}};
    \node[NodeST](5)at(5.00,-5.20){\begin{math}\Product\end{math}};
    \node[NodeST](7)at(7.00,-7.80){\begin{math}\Product\end{math}};
    \node[NodeST](9)at(9.00,0.00){\begin{math}\Product\end{math}};
    \draw[Edge](0)--(1);
    \draw[Edge](1)--(3);
    \draw[Edge](10)--(11);
    \draw[Edge](11)--(9);
    \draw[Edge](12)--(11);
    \draw[Edge](2)--(1);
    \draw[Edge](3)--(9);
    \draw[Edge](4)--(5);
    \draw[Edge](5)--(3);
    \draw[Edge](6)--(7);
    \draw[Edge,dashed](7)--(5);
    \draw[Edge](8)--(7);
    \node(r)at(9.00,1.95){};
    \draw[Edge](r)--(9);
    \accolade{0.5pt}{0.3}{0.2}{5.20,-5.10}{7.20,-7.90}{\LARGE $k$}
\end{tikzpicture} \text{ for } k \geq 1.
\end{split}
\end{equation}

One of our perspective is to allowed new generators in the rewriting 
process to found a finite, terminating and confluent 
orientation of $\Mag(3,4)$.

From~\eqref{eq:Rules_Mag_3_4}, we also conjecture that the Hilbert 
series of $\Mag(2,3)$ and $\Mag(3,4)$ equals
\begin{equation}
\HilbertSeries_{\Mag(3,4)}(t) = \frac{t}{(t-1)^3}
\left( t^6 -t^4 -2 t^2 +2t -1 \right).
\end{equation}
And thus by Taylor expansion, we have the sequence 
\begin{equation}
1, 1, 2, 4
\end{equation}
for the first dimensions of $\Mag(2,3)$ and $\Mag(3,4)$, and then for 
$n \geq 5,$
\begin{equation}
\Mag(2,3)(n) =\Mag(3,4)(n) = \frac{n(n+1)}{2} - 7.
\end{equation}

\subsection{Operad \texorpdfstring{$\Mag(2,4)$}{Mag(2,4)}}
The rewriting rules $A_2 \Rew A_4$ and $A_4 \Rew A_2$ are both 
convergent orientation of $\Mag(2,4)$. Its Hilbert series is equal 
to~\eqref{equ:Hilbert_series_Mag_1_2}.

Given an integer $i \geq 1$, we define the 
binary operation 
$\circ_i^{(2,4)} : \mathcal{C}(n) \times \mathcal{C}(n')\rightarrow
 \mathcal{C}(n + n' -1)$ for 
$n \geq i$ and $n' \geq 1$ by
\begin{equation}
a \circ_i^{(2,4)} b := \left\{
    \begin{split}
    & (a_1, \dots, a_{k}, b_1, \dots, b_{q-1}, b_{q} + a_{k+1}, 
    a_{k+2}, \dots ,a_{p}) & \text{if } i = s_k\\ 
    & (a_1, \dots, a_{k}, i - s_{k}, b_1, \dots, b_{q-1}, b_{q}
     + s_{k+1} - i, a_{k+2}, \dots ,a_{p}) & \text{otherwise}
    \end{split}
  \right.
\end{equation}
where $k \geq 0$  be such that $s_k \leq i < s_{k+1}$. 

\begin{Proposition} \label{prop:Realisation_Mag_2_4}
The operads 
$\left(\mathcal{C}, \left\{\overline{\circ_i^{(2,4)}}\right\}_{i \geq 1}
\right)$ is a realisation of $\Mag(2,4)$.
\end{Proposition}

\begin{proof}
The set of normal forms of arity $n$ for the rewriting rule 
$A_2 \Rew A_4$ is~\eqref{equ:Normal_forms_A2}. Thus we can show that
 the function~\eqref{equ:Morphism_A2}
is an operad isomorphism from the operad $\Mag(2,4)$ to 
$\left(\mathcal{C}, \left\{\circ_i^{(2,4)}\right\}_{i \geq 1}\right)$. 
\end{proof}