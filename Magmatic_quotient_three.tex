%%%%%%%%%%%%%%%%%%%%%%%%%%%%%%%%%%%%%%%%%%%%%%%%%%%%%%%%%%%%%%%%%%%%%%%%
%%%%%%%%%%%%%%%%%%%%%%%%%%%%%%%%%%%%%%%%%%%%%%%%%%%%%%%%%%%%%%%%%%%%%%%%
%%%%%%%%%%%%%%%%%%%%%%%%%%%%%%%%%%%%%%%%%%%%%%%%%%%%%%%%%%%%%%%%%%%%%%%%
\section{Equating two cubic trees} \label{sec:MAg_3}
In this section, we explore all the quotients of $\Mag$ obtained by
equating two trees of degree $3$. We denote by $\Afr_k$ the $k$-th tree
of degree $3$ for the lexicographic order, that is
\begin{center}
    \begin{tabular}{ccccc}
        \quad $\TreeA$ \quad & \quad $\TreeB$ \quad
        & \quad $\TreeC$ \quad & \quad $\TreeD$ \quad
        & \quad $\TreeE$ \quad \\
        $\Afr_1$ & $\Afr_2$ & $\Afr_3$ & $\Afr_4$ & $\Afr_5$
    \end{tabular}.
\end{center}
We denote by $\Mag^{\{i, j\}}$ the quotient operad $\Mag/_{\Congr}$,
where $\Congr$ is the operad congruence generated by
$\Afr_i \Congr \Afr_j$. We have already studied the operad
$\Mag^{\{1, 5\}} = \CAs{3}$ in Section~\ref{subsubsec:CAs_3}.
\medbreak

%%%%%%%%%%%%%%%%%%%%%%%%%%%%%%%%%%%%%%%%%%%%%%%%%%%%%%%%%%%%%%%%%%%%%%%%
%%%%%%%%%%%%%%%%%%%%%%%%%%%%%%%%%%%%%%%%%%%%%%%%%%%%%%%%%%%%%%%%%%%%%%%%
\subsection{Anti-isomorphism and combinatorial realizations}
Some of the quotients $\Mag^{\{i, j\}}$ are anti-isomorphic.
Indeed, the map $\phi : \Mag \to \Mag$ sending any binary tree $\Tfr$
to the binary tree obtained by exchanging recursively the left and
right subtrees of $\Tfr$ is an anti-isomorphism of $\Mag$. For this
reason, the $\binom{5}{2} = 10$ quotients $\Mag^{\{i, j\}}$ of $\Mag$
fit into the six equivalence classes
\begin{multline}
    \left\{\Mag^{\{1, 2\}}, \Mag^{\{4, 5\}}\right\},
    \left\{\Mag^{\{1, 3\}}, \Mag^{\{3, 5\}}\right\},
    \left\{\Mag^{\{1, 4\}}, \Mag^{\{2, 5\}}\right\}, \\
    \left\{\CAs{3}\right\},
    \left\{\Mag^{\{2, 3\}}, \Mag^{\{3, 4\}}\right\},
    \left\{\Mag^{\{2, 4\}}\right\}
\end{multline}
of anti-isomorphic operads. Anti-isomorphic operads are linked by the
following Proposition.
\medbreak

\Todo{Fusionner les deux propositions et en faire un lemme, écrire aucune
preuve et dire qu'elles sony easy}
\begin{Lemma} \label{prop:recall_anti_isomorphic}
  Let $\Oca_1$ and $\Oca_2$ be two anti-isomorphic operads and let
  $\phi$ be an anti-isomorphism between $\Oca_1$ and $\Oca_2$. Then,
    \begin{enumerate}[label={(\it\roman*)}]
        \item \label{item:recall_anti_isomorphic_1}
        $\HilbertSeries_{\Oca_1}(t) = \HilbertSeries_{\Oca_2}(t)$;
        \item \label{item:recall_anti_isomorphic_2}
        if $\Rew^{(1)}$ is a convergent presentation of
        $\Oca_1$,
        then the set of rewrite rules $\Rew^{(2)}$ satisfying
        $\phi(x) \Rew^{(2)} \phi(y)$
        for any $x, y \in \Oca_1$ whenever
        $x \Rew^{(1)} y$,
        is a convergent presentation of~$\Oca_2$.
    \end{enumerate}
\end{Lemma}
\begin{proof}
  Point~\ref{item:recall_anti_isomorphic_1} is due to the fact that for
  any $n \geq 1$, the restriction of $\phi$ from $\Oca_1(n)$ to
  $\Oca_2(n)$ is a bijection.
  \smallbreak

    To establish~\ref{item:recall_anti_isomorphic_2},
    \Todo{C : preuve ou référence ?}
    \Todo{S: je suis presque d'avis de mettre ça en lemme et de ne pas
    mettre de démo vu que c'est un résultat facile}
\end{proof}
\medbreak

Given an operad $\Oca$ with partial compositions $\circ_i$, we consider
the partial compositions $\bar{\circ}_i$ defined by
\begin{math}
  x \, \bar{\circ}_i \, y := x \circ_{|x| - i + 1} y
\end{math}
for all $x, y \in \Oca$ and $i \in [|x|]$.

\begin{Proposition} \label{prop:realization_anti_isomorphic}
    Let $\Oca_1$ and $\Oca_2$ be two anti-isomorphic operads. If
    $\left(\Oca, \circ_i\right)$ is a combinatorial realization of
    $\Oca_1$, then
    \begin{math}
        \left(\Oca, \bar{\circ}_i\right)
    \end{math}
    is a combinatorial realization of $\Oca_2$.
\end{Proposition}
\begin{proof}
  If $\phi_1$ is isomorphism from $(\Oca, \circ_i)$ to
  $\Oca_1$ and $\phi$ an anti-isomorphism from $\Oca_1$ to $\Oca_2$,
  then $\phi \circ \phi_1$ is an isomorphism from
  $(\Oca, \bar{\circ}_i)$ to $\Oca_2$.
\end{proof}
\medbreak

\Todo{S: pourquoi avoir séparé ces deux résultats?}
\Todo{S: sinon, je crois que ça devrait vraiment être plutot des lemmes}

%%%%%%%%%%%%%%%%%%%%%%%%%%%%%%%%%%%%%%%%%%%%%%%%%%%%%%%%%%%%%%%%%%%%%%%%
%%%%%%%%%%%%%%%%%%%%%%%%%%%%%%%%%%%%%%%%%%%%%%%%%%%%%%%%%%%%%%%%%%%%%%%%
\subsection{Quotients on integer compositions}
Four among the six equivalence classes of the quotients
$\Mag^{\{i, j\}}$ of $\Mag$ can be realized in terms of operads on
integer compositions. Let us review these.
\medbreak

\Todo{S: pas encore eu le temps de relir en détail.}

%%%%%%%%%%%%%%%%%%%%%%%%%%%%%%%%%%%%%%%%%%%%%%%%%%%%%%%%%%%%%%%%%%%%%%%%
\subsubsection{Operads $\Mag^{\{1, 2\}}$ and $\Mag^{\{4, 5\}}$}
\label{subsubsec:Mag_1_2}
The reader can check that the rewrite rule $\Afr_2 \Rew \Afr_1$ is a
convergent presentation of $\Mag^{\{1, 2\}}$. The operads
$\Mag^{\{1, 2\}}$ and $\Mag^{\{4, 5\}}$ are anti-isomorphic, so that the
rewrite rule $\Afr_4 \Rew \Afr_5$ is a convergent presentation
of~$\Mag^{\{4, 5\}}$.
\medbreak

Like Proposition~\ref{prop:Hilbert_series_CAs_3}, we compute the
following result thanks to~\cite{Gir18}.

\begin{Theorem} \label{thm:Hilbert_series_Mag_1_2}
    The Hilbert series of $\Mag^{\{1, 2\}}$ and $\Mag^{\{4, 5\}}$ are
    \begin{equation} \label{equ:Hilbert_series_Mag_1_2}
        \HilbertSeries_{\Mag^{\{1, 2\}}}(t)
        = \HilbertSeries_{\Mag^{\{4, 5\}}}(t) =
        t \frac{1 - t}{1 - 2t}.
    \end{equation}
\end{Theorem}
\medbreak

A Taylor expansion of series~\eqref{equ:Hilbert_series_Mag_1_2} shows
the following.
\medbreak

\begin{Proposition} \label{prop:Close_formula_Mag_1_2}
    For all $n \geq 2,$
    \begin{equation} \label{equ:Close_formula_Mag_1_2}
        \# \Mag^{\{1, 2\}}(n) = \# \Mag^{\{4, 5\}}(n) = 2^{n-2}.
    \end{equation}
\end{Proposition}
\medbreak

Many graduate sets of combinatorial objects are enumerated by powers of
$2$. We choose to present a combinatorial realization of
$\Mag^{\{1, 2\}}$ based on
integer compositions. Recall that an \Def{integer composition} is a
finite sequence of positive integers. If
$a := \left(a_1, \dots, a_p\right)$ is an integer composition, we denote
by $s_{i, j}(a)$ the number $1 + \sum_{i \leq k \leq j} a_k$. The
\Def{arity} of $a$ is $s_{1, p}(a)$. Observe that the empty integer
composition $\epsilon$ is the unique object of arity $1$. The graded set
of all integer compositions is denoted by $\Compositions$.
\medbreak

Given an integer $i \geq 1$, we define the binary operation
\begin{math}
    \circ_i^{(1,2)} : \Compositions(n) \times \Compositions(m)
    \to\Compositions(n + m - 1)
\end{math}
for $n \geq i$ and $m \geq 1$ by
\begin{equation}
    \left(a_1, \dots, a_p\right) \circ_i^{(1,2)}
    \left(b_1, \dots, b_{q}\right)
    :=
    \begin{cases}
        \left(a_1, \dots ,a_{p}, b_1, \dots ,b_{q}\right) &
        \mbox{if } i = n, \\
        \left(a_1, \dots, a_{k}, a_{k+1} + m - 1, a_{k + 2}, \dots,
            a_p\right)
            & \mbox{otherwise},
    \end{cases}
\end{equation}
where $k \geq 0$ is such that $s_{1, k}(a) \leq i < s_{1, k + 1}(a)$.
\medbreak

\begin{Proposition} \label{prop:Realisation_Mag_1_2}
    The operad $\left(\Compositions, \circ_i^{(1,2)}\right)$ is a
    combinatorial realization of $\Mag^{\{1, 2\}}$.
\end{Proposition}
\begin{proof}
  It is equivalent to show that the operads
  $\left(\Compositions, \circ_i^{(1,2)}\right)$ and $\Mag^{\{1, 2\}}$
  are
  isomorphic. The set of normal forms of arity $n$ for the rewrite rule
  $\Afr_2 \Rew \Afr_1$ is
\begin{equation} \label{equ:Normal_forms_A2}
\left\{ \LComb{k_1 \, ; \, \dots \, ; \, k_p} :
\left(k_1, \dots, k_p\right) \in \Compositions(n) \right\}
\end{equation}
where $\LComb{k_1 \, ; \, \dots \, ; \, k_p}$ is
$\RComb{p} \circ \left[\LComb{k_1-1}; \dots ; \LComb{k_p-1}
; | \, \right]$.
Thus, the function
$\phi : \Mag^{\{1, 2\}}(n) \Rew \Compositions(n)$ defined by
\begin{equation} \label{equ:Morphism_A2}
\phi\left( \LComb{k_1 \, ; \, \dots \, ; \, k_p} \right) :=
\left(k_1 ; \dots ; k_p  \right)
\end{equation}
is a bijection. Let us show that $\phi$ is an operad morphism. Given
\begin{equation}
  a := \LComb{a_1 \, ; \, \dots \, ; \, a_p} \in \Mag^{\{1, 2\}}(n)
  \text{ and }
  b := \LComb{b_1 \, ; \, \dots \, ; \, b_q} \in \Mag^{\{1, 2\}}(n'),
\end{equation}
$a \circ_n b$ equals $\LComb{a_1 \, ; \, \dots \, ; \, a_p
 \, ; \, b_1 \, ; \, \dots \, ; \, b_q}$, so that
 $\phi\left(a \circ_n b\right) = \phi\left(a\right) \circ_n^{(1,2)}
 \phi\left(b\right)$. Let $1 \leq i < n$ and let $k$ be such that
 $s_{(1, k)}(a) \leq i < s_{(1, k+1)}(a)$, so that $a \circ_i b$ equals
\begin{equation} \label{equ:Rewriting_1_Mag_1_2}
\RComb{p} \circ \left[\LComb{a_1-1}; \dots ; \LComb{a_k-1};
\LComb{a_{k+1}-1}
\circ_{i-s_{(1,k)}(a)} b  ; \LComb{a_{k+2}-1} ; \dots; \LComb{a_p-1} ;
| \, \right]
\end{equation}
is rewritable into
\begin{equation}
\RComb{p} \circ \left[\LComb{a_1-1}; \dots ; \LComb{a_k-1}; \left(
\LComb{a_{k+1}-1} \circ_{i-s_{(1,k)}(a)}  \LComb{b_1} \right)
\circ_{i-s_{(1,k)}(a)+b_1} \LComb{b_2 \, ; \, \dots \, ; \, b_q} ;
\LComb{a_{k+2}-1} ; \dots; \LComb{a_p-1} ; | \, \right]
\end{equation}
which is rewritable into
\begin{equation} \label{equ:Rewriting_2_Mag_1_2}
\RComb{p} \circ \left[\LComb{a_1-1}; \dots ; \LComb{a_k-1};
\LComb{a_{k+1}-1+b_1} \circ_{i-s_{(1,k)}(a)+b_1}
\LComb{b_2 \, ; \, \dots \, ; \, b_q}  ; \LComb{a_{k+2}-1} ; \dots;
\LComb{a_p-1} ; | \, \right]
\end{equation}
 in $b_1$ steps. By iterating $q-1$ times the process
$\eqref{equ:Rewriting_1_Mag_1_2} \rightarrow
\eqref{equ:Rewriting_2_Mag_1_2}$, we get
\begin{equation}
a \circ_i b \rightarrow \LComb{a_1 \, ; \, \dots \, ; \, a_{k} \, ; \,
a_{k+1} + n'-1 \, ; \, a_{k+2} \, ; \, \dots \, ; \, a_p},
\end{equation}
so that $\phi(a \circ_i b) = \phi(a) \circ_i^{(1,2)} \phi(b)$, that is
$\phi$ is an operad morphism.
\end{proof}
\medbreak

From proposition~\ref{prop:recall_anti_isomorphic}, we deduce that
$\left(\Compositions, \bar{\circ}_i^{(1,2)}\right)$ is a combinatorial
realization of $\Mag^{\{4, 5\}}$.
\medbreak

%%%%%%%%%%%%%%%%%%%%%%%%%%%%%%%%%%%%%%%%%%%%%%%%%%%%%%%%%%%%%%%%%%%%%%%%
\subsubsection{Operads $\Mag^{\{1, 3\}}$ and $\Mag^{\{3, 5\}}$}
The reader can check that the rewrite rule $\Afr_3 \Rew \Afr_1$ is a
convergent presentation of $\Mag^{\{1, 3\}}$. Symmetrically, the rewrite
rule $\Afr_3 \Rew \Afr_5$ is a convergent presentation of
$\Mag^{\{3, 5\}}$.
\medbreak

Thanks to~\cite{Gir18}, the Hilbert series of $\Mag^{\{1, 3\}}$ and
$\Mag^{\{3, 5\}}$ are equals to~\eqref{equ:Hilbert_series_Mag_1_2}. Thus,
for $n \geq 1,$ $\# \Mag^{\{1, 3\}}(n)$ and $\# \Mag^{\{3, 5\}}(n)$ are
equal to~\eqref{prop:Close_formula_Mag_1_2}.
\medbreak

Like in Section~\ref{subsubsec:Mag_1_2}, we choose a combinatorial
 realization based on
integer compositions. Given an integer $i \geq 1$, we define the
binary operation
$\circ_i^{(1,3)} : \Compositions(n) \times \Compositions(n')\rightarrow
\Compositions(n + n' -1)$ for
$n \geq i$ and $n' \geq 1$ by
\begin{equation}
    \left(a_1, \dots, a_p\right) \circ_i^{(1,3)}
    \left(b_1, \dots, b_{q}\right)
    :=
    \begin{cases}
        \left(a_1, \dots ,a_{i-1}, b_1 + s_{(i,p)}(a),
        b_2,\dots,b_{q}\right) &
        \mbox{if } i \leq p + 1, \\
        \left(a_1, \dots, a_{k-1}, a_{k} + n'-1, a_{k+1},
        \dots,a_{p}\right)
            & \mbox{otherwise},
    \end{cases}
\end{equation}
where $k \geq 0$ is such that
$k+1 + s_{(k+1,p)}(a) < i \leq k + s_{(k,p)}(a)$.
\medbreak

\begin{Proposition} \label{prop:Realisation_Mag_1_3}
The operads
$\left(\Compositions, \circ_i^{(1,3)}\right)$
and $\left(\Compositions, \bar{\circ}_i^{(1,3)}\right)$
are combinatorial realizations of $\Mag^{\{1, 3\}}$ and $\Mag^{\{3, 5\}}$,
respectively.
\end{Proposition}
\begin{proof}
The set of normal forms of arity $n$ for the rewrite rule
$\Afr_3 \Rew \Afr_1$ is
\begin{equation}
\left\{ \Lightning{k_1 \, ; \, \dots \, ; \, k_p} :
\left(k_1, \dots, k_p\right) \in \Compositions(n) \right\}
\end{equation}
where $\Lightning{k_1 \, ; \, \dots \, ; \, k_p}$ is
$\LComb{k_1} \circ_2 \left( \LComb{k_2} \circ_2 \left( \cdots \left(
\LComb{k_{p-1}} \circ_2 \LComb{k_p} \right) \cdots \right) \right)$.
Thus, we can show that the function
$\phi : \Mag^{\{1, 3\}}(n) \rightarrow \Compositions(n)$ defined by
\begin{equation}
\phi\left( \Lightning{k_1 \, ; \, \dots \, ; \, k_p} \right) :=
\left(k_1 ; \dots ; k_p \right)
\end{equation}
is an operad isomorphism from the operad $\Mag^{\{1, 3\}}$ to
$\left(\Compositions, \circ_i^{(1,3)}\right)$.
\end{proof}
\medbreak

%%%%%%%%%%%%%%%%%%%%%%%%%%%%%%%%%%%%%%%%%%%%%%%%%%%%%%%%%%%%%%%%%%%%%%%%
\subsubsection{Operads $\Mag^{\{1, 4\}}$ and $\Mag^{\{2, 5\}}$}
The rewrite rule $\Afr_4 \Rew \Afr_1$ is a convergent presentation of
$\Mag^{\{1, 4\}}$. Symmetrically, the rewrite rule $\Afr_2 \Rew \Afr_5$
is a convergent p of $\Mag^{\{2, 5\}}$.
\medbreak

Thanks to~\cite{Gir18}, the Hilbert series of $\Mag^{\{1, 4\}}$ and
$\Mag^{\{2, 5\}}$ are equal to~\eqref{equ:Hilbert_series_Mag_1_2}. Thus,
for $n \geq 1,$ $\# \Mag^{\{1, 4\}}(n)$ and $\# \Mag^{\{2, 5\}}(n)$ are
equal to~\eqref{prop:Close_formula_Mag_1_2}.
\medbreak

Given an integer $i \geq 1$, we define the binary operation
$\circ_i^{(2,5)} : \Compositions(n) \times \Compositions(n')
\rightarrow \Compositions(n + n' -1)$ for $n \geq i$ and $n' \geq 1$ by
\begin{equation}
    a \circ_i^{(2,5)} b :=
    \begin{cases}
        (a_1, \dots, a_{k}, b_1, \dots, b_{q}, a_{k+1}, \dots ,
        a_{p}) & \mbox{if } i = s_k, \\
        (a_1, \dots, a_{k}, i - s_{k}, b_1, \dots, b_{q-1},
        b_{q} + s_{k+1} - i, a_{k+2}, \dots ,a_{p}) & \mbox{otherwise},
    \end{cases}
\end{equation}
where $k \geq 0$ be such that $s_k \leq i < s_{k+1}$.
\medbreak

\begin{Proposition} \label{prop:Realisation_Mag_1_4}
The operad
$\left(\Compositions, \bar{\circ}_i^{(2,5)}\right)$ and
$\left(\Compositions, \circ_i^{(2,5)}\right)$
are combinatorial realizations of $\Mag^{\{1, 4\}}$ and
$\Mag^{\{2, 5\}}$, respectively.
\end{Proposition}

\begin{proof}
The set of normal forms of arity $n$ for the rewrite rule
$\Afr_2 \Rew \Afr_5$ is~\eqref{equ:Normal_forms_A2}. Thus, we can show
that the function~\eqref{equ:Morphism_A2} is an operad isomorphism from
$\Mag^{\{2, 5\}}$ to
$\left(\Compositions, \circ_i^{(2,5)}\right)$.
\end{proof}
\medbreak

%%%%%%%%%%%%%%%%%%%%%%%%%%%%%%%%%%%%%%%%%%%%%%%%%%%%%%%%%%%%%%%%%%%%%%%%
\subsubsection{Operad $\Mag^{\{2, 4\}}$}
The rewrite rules $\Afr_2 \Rew \Afr_4$ and $\Afr_4 \Rew \Afr_2$ are both
convergent presentations of $\Mag^{\{2, 4\}}$. The Hilbert series of
this operad is equal to~\eqref{equ:Hilbert_series_Mag_1_2}.
\medbreak

Given an integer $i \geq 1$, we define the
binary operation
$\circ_i^{(2,4)} : \Compositions(n) \times \Compositions(m)\rightarrow
 \Compositions(n + m - 1)$ for
$n \geq i$ and $m \geq 1$ by
\begin{equation}
    a \circ_i^{(2,4)} b :=
    \begin{cases}
    (a_1, \dots, a_{k}, b_1, \dots, b_{q-1}, b_{q} + a_{k+1},
    a_{k+2}, \dots ,a_{p}) & \mbox{if } i = s_k, \\
    (a_1, \dots, a_{k}, i - s_{k}, b_1, \dots, b_{q-1}, b_{q}
     + s_{k+1} - i, a_{k+2}, \dots ,a_{p}) & \mbox{otherwise},
    \end{cases}
\end{equation}
where $k \geq 0$  be such that $s_k \leq i < s_{k+1}$.
\medbreak

\begin{Proposition} \label{prop:Realisation_Mag_2_4}
The operad
$\left(\Compositions, \circ_i^{(2,4)}\right)$
is a combinatorial realization of $\Mag^{\{2, 4\}}$.
\end{Proposition}

\begin{proof}
The set of normal forms of arity $n$ for the rewrite rule
$\Afr_2 \Rew \Afr_4$ is~\eqref{equ:Normal_forms_A2}. Thus, we can show
that the function~\eqref{equ:Morphism_A2} is an operad isomorphism from
the operad $\Mag^{\{2, 4\}}$ to
$\left(\Compositions, \circ_i^{(2,4)}\right)$.
\end{proof}
\medbreak

%%%%%%%%%%%%%%%%%%%%%%%%%%%%%%%%%%%%%%%%%%%%%%%%%%%%%%%%%%%%%%%%%%%%%%%%
%%%%%%%%%%%%%%%%%%%%%%%%%%%%%%%%%%%%%%%%%%%%%%%%%%%%%%%%%%%%%%%%%%%%%%%%
\subsection{Quotients with complicated presentations}
We did not find finite convergent presentations for the operads
$\Mag^{\{2, 3\}}$ and $\Mag^{\{3, 4\}}$. However, thanks to
computer explorations, we
conjecture that the rewrite rules \Todo{S:remettre en forme}
\begin{equation}\label{eq:Rules_Mag_3_4}
\begin{split}
\Afr_4 \Rew \Afr_3,
\begin{tikzpicture}[xscale=.2,yscale=.17,Centering]
    \node(0)at(0.00,-8.25){};
    \node(10)at(10.00,-8.25){};
    \node(2)at(2.00,-8.25){};
    \node(4)at(4.00,-5.50){};
    \node(6)at(6.00,-5.50){};
    \node(8)at(8.00,-8.25){};
    \node[NodeST](1)at(1.00,-5.50){\begin{math}\Product\end{math}};
    \node[NodeST](3)at(3.00,-2.75){\begin{math}\Product\end{math}};
    \node[NodeST](5)at(5.00,0.00){\begin{math}\Product\end{math}};
    \node[NodeST](7)at(7.00,-2.75){\begin{math}\Product\end{math}};
    \node[NodeST](9)at(9.00,-5.50){\begin{math}\Product\end{math}};
    \draw[Edge](0)--(1);
    \draw[Edge](1)--(3);
    \draw[Edge](10)--(9);
    \draw[Edge](2)--(1);
    \draw[Edge](3)--(5);
    \draw[Edge](4)--(3);
    \draw[Edge](6)--(7);
    \draw[Edge](7)--(5);
    \draw[Edge](8)--(9);
    \draw[Edge](9)--(7);
    \node(r)at(5.00,2.06){};
    \draw[Edge](r)--(5);
\end{tikzpicture}
\Rew
\begin{tikzpicture}[xscale=.2,yscale=.17,Centering]
    \node(0)at(0.00,-5.50){};
    \node(10)at(10.00,-8.25){};
    \node(2)at(2.00,-8.25){};
    \node(4)at(4.00,-8.25){};
    \node(6)at(6.00,-5.50){};
    \node(8)at(8.00,-8.25){};
    \node[NodeST](1)at(1.00,-2.75){\begin{math}\Product\end{math}};
    \node[NodeST](3)at(3.00,-5.50){\begin{math}\Product\end{math}};
    \node[NodeST](5)at(5.00,0.00){\begin{math}\Product\end{math}};
    \node[NodeST](7)at(7.00,-2.75){\begin{math}\Product\end{math}};
    \node[NodeST](9)at(9.00,-5.50){\begin{math}\Product\end{math}};
    \draw[Edge](0)--(1);
    \draw[Edge](1)--(5);
    \draw[Edge](10)--(9);
    \draw[Edge](2)--(3);
    \draw[Edge](3)--(1);
    \draw[Edge](4)--(3);
    \draw[Edge](6)--(7);
    \draw[Edge](7)--(5);
    \draw[Edge](8)--(9);
    \draw[Edge](9)--(7);
    \node(r)at(5.00,2.06){};
    \draw[Edge](r)--(5);
\end{tikzpicture},
\begin{tikzpicture}[xscale=.2,yscale=.17,Centering]
    \node(0)at(0.00,-10.40){};
    \node(10)at(10.00,-5.20){};
    \node(12)at(12.00,-5.20){};
    \node(2)at(2.00,-10.40){};
    \node(4)at(4.00,-7.80){};
    \node(6)at(6.00,-7.80){};
    \node(8)at(8.00,-7.80){};
    \node[NodeST](1)at(1.00,-7.80){\begin{math}\Product\end{math}};
    \node[NodeST](11)at(11.00,-2.60){\begin{math}\Product\end{math}};
    \node[NodeST](3)at(3.00,-5.20){\begin{math}\Product\end{math}};
    \node[NodeST](5)at(5.00,-2.60){\begin{math}\Product\end{math}};
    \node[NodeST](7)at(7.00,-5.20){\begin{math}\Product\end{math}};
    \node[NodeST](9)at(9.00,0.00){\begin{math}\Product\end{math}};
    \draw[Edge](0)--(1);
    \draw[Edge](1)--(3);
    \draw[Edge](10)--(11);
    \draw[Edge](11)--(9);
    \draw[Edge](12)--(11);
    \draw[Edge](2)--(1);
    \draw[Edge](3)--(5);
    \draw[Edge](4)--(3);
    \draw[Edge](5)--(9);
    \draw[Edge](6)--(7);
    \draw[Edge](7)--(5);
    \draw[Edge](8)--(7);
    \node(r)at(9.00,1.95){};
    \draw[Edge](r)--(9);
\end{tikzpicture}
\Rew
\begin{tikzpicture}[xscale=.2,yscale=.17,Centering]
    \node(0)at(0.00,-7.80){};
    \node(10)at(10.00,-5.20){};
    \node(12)at(12.00,-5.20){};
    \node(2)at(2.00,-7.80){};
    \node(4)at(4.00,-7.80){};
    \node(6)at(6.00,-10.40){};
    \node(8)at(8.00,-10.40){};
    \node[NodeST](1)at(1.00,-5.20){\begin{math}\Product\end{math}};
    \node[NodeST](11)at(11.00,-2.60){\begin{math}\Product\end{math}};
    \node[NodeST](3)at(3.00,-2.60){\begin{math}\Product\end{math}};
    \node[NodeST](5)at(5.00,-5.20){\begin{math}\Product\end{math}};
    \node[NodeST](7)at(7.00,-7.80){\begin{math}\Product\end{math}};
    \node[NodeST](9)at(9.00,0.00){\begin{math}\Product\end{math}};
    \draw[Edge](0)--(1);
    \draw[Edge](1)--(3);
    \draw[Edge](10)--(11);
    \draw[Edge](11)--(9);
    \draw[Edge](12)--(11);
    \draw[Edge](2)--(1);
    \draw[Edge](3)--(9);
    \draw[Edge](4)--(5);
    \draw[Edge](5)--(3);
    \draw[Edge](6)--(7);
    \draw[Edge](7)--(5);
    \draw[Edge](8)--(7);
    \node(r)at(9.00,1.95){};
    \draw[Edge](r)--(9);
\end{tikzpicture},\\
\begin{tikzpicture}[xscale=.23,yscale=.18,Centering]
    \node(0)at(0.00,-8.67){};
    \node(10)at(10.00,-4.33){};
    \node(12)at(12.00,-4.33){};
    \node(2)at(2.00,-10.83){};
    \node(4)at(4.00,-10.83){};
    \node(6)at(6.00,-6.50){};
    \node(8)at(8.00,-4.33){};
    \node[NodeST](1)at(1.00,-6.50){\begin{math}\Product\end{math}};
    \node[NodeST](11)at(11.00,-2.17){\begin{math}\Product\end{math}};
    \node[NodeST](3)at(3.00,-8.67){\begin{math}\Product\end{math}};
    \node[NodeST](5)at(4.00,-5.33){\begin{math}\Product\end{math}};
    \node[NodeST](7)at(7.00,-2.17){\begin{math}\Product\end{math}};
    \node[NodeST](9)at(9.00,0.00){\begin{math}\Product\end{math}};
    \draw[Edge](0)--(1);
    \draw[Edge](1)--(5);
    \draw[Edge](10)--(11);
    \draw[Edge](11)--(9);
    \draw[Edge](12)--(11);
    \draw[Edge](2)--(3);
    \draw[Edge](3)--(1);
    \draw[Edge](4)--(3);
    \draw[Edge,dotted](5)edge[]node[font=\footnotesize]{
        \hspace*{.4cm}\begin{math}k\end{math}}(7);
    \draw[Edge](6)--(5);
    \draw[Edge](7)--(9);
    \draw[Edge](8)--(7);
    \node(r)at(9.00,1.62){};
    \draw[Edge](r)--(9);
\end{tikzpicture}
\Rew
\begin{tikzpicture}[xscale=.22,yscale=.18,Centering]
    \node(0)at(0.00,-7.80){};
    \node(10)at(10.00,-5.20){};
    \node(12)at(12.00,-5.20){};
    \node(2)at(2.00,-7.80){};
    \node(4)at(4.00,-7.80){};
    \node(6)at(6.00,-10.40){};
    \node(8)at(8.00,-10.40){};
    \node[NodeST](1)at(1.00,-5.20){\begin{math}\Product\end{math}};
    \node[NodeST](11)at(11.00,-2.60){\begin{math}\Product\end{math}};
    \node[NodeST](3)at(3.00,-2.60){\begin{math}\Product\end{math}};
    \node[NodeST](5)at(5.00,-5.20){\begin{math}\Product\end{math}};
    \node[NodeST](7)at(7.00,-7.80){\begin{math}\Product\end{math}};
    \node[NodeST](9)at(9.00,0.00){\begin{math}\Product\end{math}};
    \draw[Edge](0)--(1);
    \draw[Edge](1)--(3);
    \draw[Edge](10)--(11);
    \draw[Edge](11)--(9);
    \draw[Edge](12)--(11);
    \draw[Edge](2)--(1);
    \draw[Edge](3)--(9);
    \draw[Edge](4)--(5);
    \draw[Edge](5)--(3);
    \draw[Edge](6)--(7);
    \draw[Edge,dotted](7)edge[]node[font=\footnotesize]{
        \hspace*{.4cm}\begin{math}k\end{math}}(5);
    \draw[Edge](8)--(7);
    \node(r)at(9.00,1.95){};
    \draw[Edge](r)--(9);
\end{tikzpicture} \text{ for } k \geq 1.
\end{split}
\end{equation}
form a convergent presentation of $\Mag^{\{3, 4\}}$.
\medbreak

From~\eqref{eq:Rules_Mag_3_4}, we also conjecture that the Hilbert
series of $\Mag^{\{2, 3\}}$ and $\Mag^{\{3, 4\}}$ are
\begin{equation}
    \HilbertSeries_{\Mag^{\{2, 3\}}}(t) =
    \HilbertSeries_{\Mag^{\{3, 4\}}}(t) =
    \frac{t}{(1 - t)^3}
    \left(1 - 2t + 2t^2 + t^4 - t^6\right).
\end{equation}
By Taylor expansion, we have the sequence
\begin{equation}
    1, 1, 2, 4, 8, 14, 21, 29, 38, 48
\end{equation}
for the first dimensions of $\Mag^{\{2, 3\}}$ and $\Mag^{\{3, 4\}}$.
For $n \geq 5$,
\begin{equation}
    \#\Mag^{\{2, 3\}}(n) =\#\Mag^{\{3, 4\}}(n) = \frac{n(n+1)}{2} - 7.
\end{equation}
\medbreak
