% Author: Cyrille Chenavier, Christophe Cordero, Samuele Giraudo
% Creation: jan. 2019
% Modifications: jan. 2019

\documentclass[11pt,reqno]{amsart}

%%%%%%%%%%%%%%%%%%%%%%%%%%%%%%%%%%%%%%%%%%%%%%%%%%%%%%%%%%%%%%%%%%%%%%%%
%%%%%%%%%%%%%%%%%%%%%%%%%%%%%%%%%%%%%%%%%%%%%%%%%%%%%%%%%%%%%%%%%%%%%%%%
%%%%%%%%%%%%%%%%%%%%%%%%%%%%%%%%%%%%%%%%%%%%%%%%%%%%%%%%%%%%%%%%%%%%%%%%
\usepackage[utf8x]{inputenc}
\usepackage[english]{babel}
\usepackage{amsmath,amsfonts,amssymb,amsthm,shuffle}
\usepackage[T1]{fontenc}
\usepackage[math]{anttor}

% Layout.
\usepackage[top=3.5cm,bottom=3.5cm,left=3.4cm,right=3.4cm]{geometry}

% Colors of hyperlinks.
\usepackage[dvipsnames]{xcolor}

\usepackage[hyperindex=true,frenchlinks=true,colorlinks=true,
citecolor=Col2,linkcolor=Col3,urlcolor=Col4,linktocpage,
pagebackref=true]{hyperref}

% Tikz.
\usepackage{tikz}
\usetikzlibrary{shapes}
\usetikzlibrary{fit}
\usetikzlibrary{decorations.pathmorphing}

% Misc.
\usepackage{mathtools}
\usepackage{dsfont}
\usepackage{wasysym}
\usepackage{stmaryrd}
\usepackage{cite}
\usepackage{subfig}
\usepackage{multirow}
\usepackage{enumitem}
\usepackage{multicol}
\usepackage{cancel}
\usepackage{xspace}

%%%%%%%%%%%%%%%%%%%%%%%%%%%%%%%%%%%%%%%%%%%%%%%%%%%%%%%%%%%%%%%%%%%%%%%%
%%%%%%%%%%%%%%%%%%%%%%%%%%%%%%%%%%%%%%%%%%%%%%%%%%%%%%%%%%%%%%%%%%%%%%%%
%%%%%%%%%%%%%%%%%%%%%%%%%%%%%%%%%%%%%%%%%%%%%%%%%%%%%%%%%%%%%%%%%%%%%%%%
% Line space.
\linespread{1.15}
\renewcommand{\arraystretch}{1.4}

% Vertical space for equations.
\setlength{\abovedisplayskip}{5pt}
\setlength{\belowdisplayskip}{5pt}

% Alphabetic footnote marks.
\renewcommand{\thefootnote}{\alph{footnote}}

% To allow cutting equations in several pages.
\allowdisplaybreaks

% Numbering of equations.
\numberwithin{equation}{subsection}

% Depth of the table of contents.
\setcounter{tocdepth}{2}

% Indentation in the table of contents.
\makeatletter
\def\l@section{\@tocline{1}{3pt}{1pc}{5pc}{}}
\def\l@subsection{\@tocline{2}{2pt}{2pc}{5pc}{}}
\makeatother

% Environments definitions.
\newtheorem{Theorem}{Theorem}[subsection]
\newtheorem{Proposition}[Theorem]{Proposition}
\newtheorem{Lemma}[Theorem]{Lemma}

% Better comparison symbols.
\renewcommand{\leq}{\leqslant}
\renewcommand{\geq}{\geqslant}

%%%%%%%%%%%%%%%%%%%%%%%%%%%%%%%%%%%%%%%%%%%%%%%%%%%%%%%%%%%%%%%%%%%%%%%%
%%%%%%%%%%%%%%%%%%%%%%%%%%%%%%%%%%%%%%%%%%%%%%%%%%%%%%%%%%%%%%%%%%%%%%%%
%%%%%%%%%%%%%%%%%%%%%%%%%%%%%%%%%%%%%%%%%%%%%%%%%%%%%%%%%%%%%%%%%%%%%%%%
\title[Response to Referees]
    {Response to referee comments \\
    and \\
    changes report}
\keywords{}
\subjclass[2010]{}
\date{\today}
\author{Cyrille Chenavier \and Christophe Cordero \and Samuele Giraudo}
\address{\scriptsize Université Paris-Est, LIGM (UMR $8049$), CNRS,
    ENPC, ESIEE Paris, UPEM, F-$77454$, Marne-la-Vallée, France}
\email{cyrille.chenavier@u-pem.fr}
\email{christophe.cordero@u-pem.fr}
\email{samuele.giraudo@u-pem.fr}

%%%%%%%%%%%%%%%%%%%%%%%%%%%%%%%%%%%%%%%%%%%%%%%%%%%%%%%%%%%%%%%%%%%%%%%%
%%%%%%%%%%%%%%%%%%%%%%%%%%%%%%%%%%%%%%%%%%%%%%%%%%%%%%%%%%%%%%%%%%%%%%%%
%%%%%%%%%%%%%%%%%%%%%%%%%%%%%%%%%%%%%%%%%%%%%%%%%%%%%%%%%%%%%%%%%%%%%%%%
\begin{document}

\maketitle

First of all, we would like to thank the reviewer for its time and its
careful reading. Its suggestions improve the paper.
\medbreak

Please find here the modifications made to the manuscript
``Quotients of the magmatic operad: lattice structures and convergent
rewrite systems'' submitted to Experimental Mathematics, following
the comments of the reviewer.
\bigbreak

\begin{itemize}

%%%%%%%%%%%%%%%%%%%%%%%%%%%%%%%%%%%%%%%%%%%%%%%%%%%%%%%%%%%%%%%%%%%%%%%%
\item `` {\it
They were obviously obtained through an expert and
heavy use of computer software.
In my humble opinion, when doing so, one should cite carefully what
is being used (language, package, etc), just like articles are cited.
}''
\smallbreak

We do not use a particular package. We added
``All our computer programs are made from scratch in {\sc Caml}
and {\sc Python}'' in the introduction.
\medbreak

%%%%%%%%%%%%%%%%%%%%%%%%%%%%%%%%%%%%%%%%%%%%%%%%%%%%%%%%%%%%%%%%%%%%%%%%
\item `` {\it
The allusions in two places to "intervals in the Tamari order" are
rather strange. If you have any evidence or hint that this could be a
good condition, please tell the reader.
}''
\smallbreak

For the while, we have no general result connecting the quotients of
the magmatic operad and intervals of Tamari lattices but we suspect
that there is a link between the properties of the equated trees
with respect to the Tamari order and the description of the quotient
operad. This question was actually one of the starting points of this
work, and the research rather leads up to the presented results.
\medbreak

%%%%%%%%%%%%%%%%%%%%%%%%%%%%%%%%%%%%%%%%%%%%%%%%%%%%%%%%%%%%%%%%%%%%%%%%
\item `` {\it
The standard indexing for Catalan numbers is
cat(n)=binomial(2n,n)/(n+1). Please respect this convention.
}''
\smallbreak

We have changed this, as suggested by the reviewer. The convention is
now respected. Moreover, we have provided an updated version of
the statement of Lemma~3.2.1.
\medbreak

%%%%%%%%%%%%%%%%%%%%%%%%%%%%%%%%%%%%%%%%%%%%%%%%%%%%%%%%%%%%%%%%%%%%%%%%
\item `` {\it
In the proof of Lemma 1.3.1, say that p(t) > p(t') means t > t'
}''
\smallbreak

This precision has been added to the proof of the lemma.
\medbreak

%%%%%%%%%%%%%%%%%%%%%%%%%%%%%%%%%%%%%%%%%%%%%%%%%%%%%%%%%%%%%%%%%%%%%%%%
\item `` {\it
In the proof of Lemma 1.3.4, refer to a precise Theorem in [Gir16]
}''
\smallbreak

The precise reference has been added.
\medbreak

%%%%%%%%%%%%%%%%%%%%%%%%%%%%%%%%%%%%%%%%%%%%%%%%%%%%%%%%%%%%%%%%%%%%%%%%
\item ``{\it
  Page 9, is this completion or rewriting systems an operadic avatar of
  Knuth-Bendix procedure ?
}''
\smallbreak

In essence, this is the same thing but the terminology differs from
the context: for term rewriting ystems or word rewriting systems, the
classical terminology in the litterature is ``Knuth-Bendix procedure''
and for linear structures, such  such as polynomial algebras,
noncommutative algebras or operads, the classical terminology is
''Buchberger procedure".
\medbreak

%%%%%%%%%%%%%%%%%%%%%%%%%%%%%%%%%%%%%%%%%%%%%%%%%%%%%%%%%%%%%%%%%%%%%%%%
\item ''{\it
A reference for the name "Grassmann formula" would be welcome (page 12)
}''
\smallbreak

A reference has been added.
\medbreak

%%%%%%%%%%%%%%%%%%%%%%%%%%%%%%%%%%%%%%%%%%%%%%%%%%%%%%%%%%%%%%%%%%%%%%%%
\item ''{\it
In the proof of Prop. 3.2.5, why not use that the composite of a
morphism is a morphism ?
}''
\smallbreak

This is a good remark: we changed the text.
\medbreak

%%%%%%%%%%%%%%%%%%%%%%%%%%%%%%%%%%%%%%%%%%%%%%%%%%%%%%%%%%%%%%%%%%%%%%%%
\item ''{\it
What are the squared regions (page 17) ? I did not look at the colored
article.
}''
\smallbreak

This was a mistake because, indeed, there are no squared regions. we
change ``squared regions'' by ``dotted edges''.
\medbreak

%%%%%%%%%%%%%%%%%%%%%%%%%%%%%%%%%%%%%%%%%%%%%%%%%%%%%%%%%%%%%%%%%%%%%%%%
\item ''{\it
Maybe it would be a good idea to add the words in $\{0,2\}$ below the
relations in the large figure of page 19 ?
}''
\smallbreak

The words in correspondence with the trees of the relations have been
written.
\medbreak

%%%%%%%%%%%%%%%%%%%%%%%%%%%%%%%%%%%%%%%%%%%%%%%%%%%%%%%%%%%%%%%%%%%%%%%%
\item `` {\it
The definition of $z_d$ page 20 needs to be checked (composition index)
}''
\smallbreak

Indeed, you are right, the index was wrong. We changed "d/2" into
"d-1/2".
\medbreak

%%%%%%%%%%%%%%%%%%%%%%%%%%%%%%%%%%%%%%%%%%%%%%%%%%%%%%%%%%%%%%%%%%%%%%%%
\item `` {\it
About table of numbers page 23, do you see any interesting patterns ?
possible formulas ? rationality of generating series ?
}''
\smallbreak

It seems that it is not the case. Even with the package gfun on
Maple, we do not see any pattern, formulas, \dots
\medbreak

%%%%%%%%%%%%%%%%%%%%%%%%%%%%%%%%%%%%%%%%%%%%%%%%%%%%%%%%%%%%%%%%%%%%%%%%
\item `` {\it
In section 4, the sentence "The reader can check" (repeated 4 times)
is a bit terse. This is certainly feasible by hand. The reader would
appreciate to be given hints of arguments for confluence (no critical
pair ?) and termination. Or maybe there are available software that
can swallow this kind of rewriting system and check that is is closed ?
}''
\smallbreak

We added the "using the Buchberger algorithm for operads" for the first
occurrence of "The reader can check". It is quite easy to run the
algorithm by hand on those examples.
\medbreak

%%%%%%%%%%%%%%%%%%%%%%%%%%%%%%%%%%%%%%%%%%%%%%%%%%%%%%%%%%%%%%%%%%%%%%%%
\item `` {\it
Question: make a statement about the non-isomorphism (as set-operads
or as k-linear-operads) between the 4 quotients indexed by compositions.
}''
\smallbreak

This is a good suggestion. We have added a small section (4.2.5) to
explain this property.
\medbreak

%%%%%%%%%%%%%%%%%%%%%%%%%%%%%%%%%%%%%%%%%%%%%%%%%%%%%%%%%%%%%%%%%%%%%%%%
\item `` {\it
For the conjecture of section 4.3, up to which arity was it verified ?
}''
\smallbreak

Until arity 40. We have added this information in the article.
\medbreak

\end{itemize}

\end{document}
