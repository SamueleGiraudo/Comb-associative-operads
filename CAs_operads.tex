%%%%%%%%%%%%%%%%%%%%%%%%%%%%%%%%%%%%%%%%%%%%%%%%%%%%%%%%%%%%%%%%%%%%%%%%
%%%%%%%%%%%%%%%%%%%%%%%%%%%%%%%%%%%%%%%%%%%%%%%%%%%%%%%%%%%%%%%%%%%%%%%%
%%%%%%%%%%%%%%%%%%%%%%%%%%%%%%%%%%%%%%%%%%%%%%%%%%%%%%%%%%%%%%%%%%%%%%%%
\section{Generalizations of the associative operad}
\label{sec:CAs_d}

In this section, we define comb associative operads and we show that the
set of such operads admits a lattice structure, isomorphic to the lattice
division for nonnegative integers. We also relate this lattice to the
lattice of linear magmatic quotients.

%%%%%%%%%%%%%%%%%%%%%%%%%%%%%%%%%%%%%%%%%%%%%%%%%%%%%%%%%%%%%%%%%%%%%%%%
%%%%%%%%%%%%%%%%%%%%%%%%%%%%%%%%%%%%%%%%%%%%%%%%%%%%%%%%%%%%%%%%%%%%%%%%
\subsection{Comb associative operads}

Let, for any integer $\gamma \geq 1$, the binary trees
\begin{equation}
    \LComb{\gamma} :=
    \underbrace{(\dots (\Product \circ_1 \Product) \circ_1 \dots)
        \circ_1 \Product}
    _{\gamma \mbox{ \footnotesize operands}}
\end{equation}
and
\begin{equation}
    \RComb{\gamma} :=
    \underbrace{\Product \circ_2
        (\dots \circ_2 (\Product \circ_2 \Product) \dots)}
    _{\gamma \mbox{ \footnotesize operands}}
\end{equation}
be respectively the left and the right comb of degree $\gamma$.
Let us now define for any $\gamma \geq 1$ the \Def{$\gamma$-comb
associative operad} $\CAs{\gamma}$ as the quotient operad
$\Mag/_{\CongrCAs{\gamma}}$ where $\CongrCAs{\gamma}$ is the smallest
operad congruence of $\Mag$ satisfying
\begin{equation} \label{equ:congruence_CAs_gamma}
    \LComb{\gamma} \enspace \CongrCAs{\gamma} \enspace \RComb{\gamma}.
\end{equation}
Notice that $\CongrCAs{1}$ is trivial so that
$\CAs{1} = \Mag$ and that $\CongrCAs{2}$ is the operad congruence
defined by
\begin{equation}
    \begin{tikzpicture}[xscale=.24,yscale=.24,Centering]
        \node(0)at(0.00,-3.33){};
        \node(2)at(2.00,-3.33){};
        \node(4)at(4.00,-1.67){};
        \node[NodeST](1)at(1.00,-1.67)
            {\begin{math}\Product\end{math}};
        \node[NodeST](3)at(3.00,0.00)
            {\begin{math}\Product\end{math}};
        \draw[Edge](0)--(1);
        \draw[Edge](1)--(3);
        \draw[Edge](2)--(1);
        \draw[Edge](4)--(3);
        \node(r)at(3.00,1.5){};
        \draw[Edge](r)--(3);
    \end{tikzpicture}
    \Congr
    \begin{tikzpicture}[xscale=.24,yscale=.24,Centering]
        \node(0)at(0.00,-1.67){};
        \node(2)at(2.00,-3.33){};
        \node(4)at(4.00,-3.33){};
        \node[NodeST](1)at(1.00,0.00)
                {\begin{math}\Product\end{math}};
        \node[NodeST](3)at(3.00,-1.67)
                {\begin{math}\Product\end{math}};
        \draw[Edge](0)--(1);
        \draw[Edge](2)--(3);
        \draw[Edge](3)--(1);
        \draw[Edge](4)--(3);
        \node(r)at(1.00,1.5){};
        \draw[Edge](r)--(1);
    \end{tikzpicture}\,
\end{equation}
so that $\CAs{2} = \As$.
Let also
\begin{equation}
    \CAsAll := \left\{\CAs{\gamma} : \gamma\geq 1\right\}
\end{equation}
be the set of all the $\gamma$-comb associative operads.
\medbreak

%%%%%%%%%%%%%%%%%%%%%%%%%%%%%%%%%%%%%%%%%%%%%%%%%%%%%%%%%%%%%%%%%%%%%%%%
%%%%%%%%%%%%%%%%%%%%%%%%%%%%%%%%%%%%%%%%%%%%%%%%%%%%%%%%%%%%%%%%%%%%%%%%
\subsection{Lattice of comb associative operads}

In order to introduce the lattice structure of $\CAsAll$, we study operad
morphisms betweeen its elements by mean of intermediate lemmas.

\begin{Lemma} \label{lem:first_dimensions_CAs}
    For any positive integers $\gamma$ and $n$ such that $\gamma \geq 2$
    and $n \leq \gamma + 1$,
    \begin{equation}
        \# \CAs{\gamma}(n) =
        \begin{cases}
            \Catalan(n)
                & \mbox{ if } n \leq \gamma, \\
            \Catalan(\gamma + 1) - 1
                & \mbox{ otherwise } (n = \gamma + 1).
        \end{cases}
    \end{equation}
\end{Lemma}
\begin{proof}
    Since the equivalence relation $\CongrCAs{\gamma}$ is trivial on the
    binary trees of degrees $d < \gamma$, and since a binary tree of
    degree $d$ has arity $n := d + 1$, one has
    $\# \CAs{\gamma}(n) = \# \Mag(n) = \Catalan(n)$ with $n \leq \gamma$.
    Besides, by definition of $\CongrCAs{\gamma}$, all the
    $\CongrCAs{\gamma}$-equivalence classes of binary trees of degree
    $\gamma$ are trivial, except one consisting in the pair
    $\left\{\LComb{\gamma}, \RComb{\gamma}\right\}$. Therefore, since a
    binary tree of degree $d$ has arity $n := \gamma + 1$,
    \begin{math}
        \# \CAs{\gamma}(n)
        = \# \Mag(\gamma + 1) - 1
        = \Catalan(\gamma + 1) - 1.
    \end{math}
\end{proof}
\medbreak

\begin{Lemma} \label{lem:surjective_morphisms_CAs}
    Let $\gamma$ and $\gamma'$ be two positive integers. If
    $\varphi:\CAs{\gamma'} \to \CAs{\gamma}$ is an operad morphism, then
    it is surjective and is unique.
\end{Lemma}
\begin{proof}
    The operad $\CAs{\gamma'}$ is generated by one binary generator
    $[\Product]_{\CongrCAs{\gamma'}}$, which is the image of $\Product$
    in $\CAs{\gamma'}$. Hence, $\varphi$ is entirely determined by the
    image $\varphi\left([\Product]_{\CongrCAs{\gamma'}}\right)$.
    Moreover, $\varphi([\Product]_{\CongrCAs{\gamma'}})$ has to be of
    arity $2$ in $\CAs{\gamma}$, so that we necessarily have
    \begin{math}
        \varphi\left([\Product]_{\CongrCAs{\gamma'}}\right)
        =
        [\Product]_{\CongrCAs{\gamma}}.
    \end{math}
    Hence, if $\varphi$ exists, it is the unique operad morphism from
    $\CAs{\gamma'}$ to $\CAs{\gamma}$. In this case,
    $[\Product]_{\CongrCAs{\gamma'}}$ being in the image of $\varphi$,
    the latter is surjective.
\end{proof}
\medbreak

\begin{Lemma} \label{lem:injective_morphisms_CAs}
    Let $\gamma$ and $\gamma'$ be two positive integers and
    $\varphi:\CAs{\gamma'} \to \CAs{\gamma}$ be an operad morphism.
    Then, $\varphi$ is injective if and only if $\gamma = \gamma'$.
\end{Lemma}
\begin{proof}
    Assume that $\varphi$ is injective. By
    Lemma~\ref{lem:surjective_morphisms_CAs}, $\varphi$ is also
    surjective, so that $\varphi$ is an isomorphism. If
    $\gamma \ne \gamma'$, by Lemma~\ref{lem:first_dimensions_CAs}, there
    is a positive integer $n$ such that
    $\# \CAs{\gamma}(n) \ne \# \CAs{\gamma'}(n)$. This is contradictory
    with the fact that $\CAs{\gamma}$ and $\CAs{\gamma'}$ are
    isomorphic. Hence, $\gamma = \gamma'$.
    \smallbreak

    Conversely, if $\gamma = \gamma'$, the only operad morphism from
    $\CAs{\gamma}$ to itself sends the generator
    $[\Product]_{\CongrCAs{\gamma}}$ to itself. This maps extends as
    an operad morphism into the identity morphism which is of course
    injective.
\end{proof}
\medbreak

\begin{Lemma} \label{lem:morphism_CAs}
    Let $\gamma$ and $\gamma'$ be two positive integers. There exists an
    operad morphism $\varphi:\CAs{\gamma'} \to \CAs{\gamma}$ if and only
    if
    \begin{math}
      \LComb{\gamma'} \CongrCAs{\gamma} \RComb{\gamma'}.
    \end{math}
\end{Lemma}
\begin{proof}
    Assume that $\varphi:\CAs{\gamma'} \to \CAs{\gamma}$ is an operad
    morphism. By Lemma~\ref{lem:surjective_morphisms_CAs}, $\varphi$
    satisfies
    \begin{math}
        \varphi\left(\left[\Tfr\right]_{\CongrCAs{\gamma'}}\right)
        = \left[\Tfr\right]_{\CongrCAs{\gamma}}
    \end{math}
    for any binary tree $\Tfr$.
    Since
    \begin{math}
        \varphi\left(
        \left[\LComb{\gamma'}\right]_{\CongrCAs{\gamma'}}
        \right)
        =
        \varphi\left(
        \left[\RComb{\gamma'}\right]_{\CongrCAs{\gamma'}}
        \right),
    \end{math}
    we have
    \begin{math}
        \left[\LComb{\gamma'}\right]_{\CongrCAs{\gamma}}
        =
        \left[\RComb{\gamma'}\right]_{\CongrCAs{\gamma}},
    \end{math}
    which is equivalent to the fact that
    \begin{math}
        \LComb{\gamma'} \CongrCAs{\gamma} \RComb{\gamma'}.
    \end{math}
    \smallbreak

    Conversely, when
    \begin{math}
        \LComb{\gamma'}\CongrCAs{\gamma}\RComb{\gamma'},
    \end{math}
    let $\varphi:\CAs{\gamma'}(2) \to \CAs{\gamma}(2)$ be the map
    defined by
    \begin{math}
        \varphi\left(
        \left[\Product\right]_{\CongrCAs{\gamma'}}\right)
        :=
        \left[\Product\right]_{\CongrCAs{\gamma}}.
    \end{math}
    Now, since $\CongrCAs{\gamma}$ is coarser than $\CongrCAs{\gamma'}$,
    $\varphi$ extends (in a unique way) into an operad morphism, whence
    the statement of the lemma.
\end{proof}
\medbreak

We define the binary relation $\OrdCAs$ on $\CAsAll$ as follows: we have
$\CAs{\gamma} \OrdCAs \CAs{\gamma'}$ if and only if there exists a
morphism $\varphi:\CAs{\gamma'} \to \CAs{\gamma}$.
\medbreak

\begin{Proposition}\label{prop:poset_CAs}
    The binary relation $\OrdCAs$ is a partial order relation on
    $\CAsAll$.
\end{Proposition}
\begin{proof}
    The binary relation $\OrdCAs$ is reflexive since there exists the
    identity morphism of $\CAs{\gamma}$ for every positive integer
    $\gamma$. Let us now assume that
    $\varphi:\CAs{\gamma''} \to \CAs{\gamma'}$ and
    $\psi:\CAs{\gamma'} \to \CAs{\gamma}$ are two operad morphisms. By
    Lemma~\ref{lem:morphism_CAs}, one has
    $\LComb{\gamma''} \CongrCAs{\gamma'} \RComb{\gamma''}$ and
    $\LComb{\gamma'} \CongrCAs{\gamma} \RComb{\gamma'}$. This implies
    that $\CongrCAs{\gamma}$ is coarser than $\CongrCAs{\gamma'}$ and
    that $\CongrCAs{\gamma'}$ is coarser that $\CongrCAs{\gamma''}$. Now,
    again by Lemma~\ref{lem:morphism_CAs}, $\psi \circ \varphi$ is an
    operad morphism. Hence, $\OrdCAs$ is transitive. Finally, let us
    assume that there exist morphisms
    $\varphi:\CAs{\gamma'}\to\CAs{\gamma}$ and
    $\psi:\CAs{\gamma}\to\CAs{\gamma'}$. In particular,
    $\psi \circ \varphi$ and $\varphi \circ \psi$ are endomorphisms of
    $\CAs{\gamma'}$ and $\CAs{\gamma}$, respectively. From
    Lemma~\ref{lem:surjective_morphisms_CAs}, these two morphisms are
    identity morphisms, so that $\varphi$ and $\psi$ are injective. From
    Lemma~\ref{lem:injective_morphisms_CAs}, $\gamma$ and $\gamma'$ are
    equal, which proves that $\OrdCAs$ is anti-symmetric. Hence,
    $\OrdCAs$ is a partial order.
\end{proof}
\medbreak

In order to show that the poset $\left(\CAsAll, \OrdCAs\right)$ is also a
lattice, we relate $\left(\CAsAll, \OrdCAs\right)$ with the lattice of
integers $\left(\N,\mid, \gcd, \lcm\right)$, where $\mid$ denotes the
division relation, $\gcd$ denotes the greatest common divisor, and
$\lcm$ the least common multiple operators, respectively.

Recall that
$\LRank(\Tfr)$ denotes the left rank of a binary tree $\Tfr$, defined in
Section~\ref{sec:operad_Mag}.
\medbreak

\begin{Lemma}\label{lem:left_rank_and_CongrCAs}
    Let $\gamma$ be a strictly positive integer and let $\Tfr$ and
    $\Tfr'$ be two binary trees. If $\Tfr \CongrCAs{\gamma} \Tfr'$, then
    \begin{equation}
        \LRank(\Tfr) \mod \gamma - 1
        \enspace = \enspace
        \LRank(\Tfr') \mod \gamma - 1.
    \end{equation}
\end{Lemma}
\begin{proof}
  We consider the rewrite rule $\LComb{\gamma} \Rew \RComb{\gamma}$ on
  $\Mag$. 
    \smallbreak

    First, we show that $\Tfr \RewContext \Tfr'$ implies that
    $\LRank\left(\Tfr'\right) - \LRank\left(\Tfr\right)$ is
    divisible by $\gamma - 1$. \Todo{à faire.}
    \smallbreak

    Assume that $\Tfr \RewContextRT \Tfr'$, so that there exist binary trees
    $\Tfr_1$, \dots, $\Tfr_k$ such that $\Tfr_1 = \Tfr$,
    $\Tfr_k = \Tfr'$ and for every $i \in \{1, \dots, k - 1\}$, we have
    $\Tfr_i\RewContext \Tfr_{i + 1}$. The integer
    \begin{math}
        \LRank\left(\Tfr_{i + 1}\right)
        -
        \LRank\left(\Tfr_i\right)
    \end{math}
    is divisible by $\gamma - 1$ from the first part of the proof, so
    that $\LRank\left(\Tfr'\right) - \LRank\left(\Tfr\right)$ is
    divisible by $\gamma-1$.
    \smallbreak

    Assume that $\Tfr \RewContextRST \Tfr'$, so that there exist binary trees
    $\Tfr_1$, \dots, $\Tfr_k$ such that $\Tfr_1 = \Tfr$,
    $\Tfr_k = \Tfr'$ and for every $i \in \{1, \dots, k - 1\}$, we have
    $\Tfr_i \RewContextRT \Tfr_{i + 1}$ or $\Tfr_{i + 1} \RewContextRT
    \Tfr_i$. From the second part of the proof, the integer
    \begin{math}
        \LRank\left(\Tfr_{i + 1}\right)
        -
        \LRank\left(\Tfr_i\right)
    \end{math}
    is divisible by $\gamma - 1$, so that
    $\LRank\left(\Tfr'\right) - \LRank\left(\Tfr\right)$ is
    divisible by $\gamma - 1$.
    \smallbreak

    The congruence $\CongrCAs{\gamma}$ is the equivalence relation
    induced by the rewrite relation $\RewContext$ induced by the rewrite
    rule $\LComb{\gamma} \Rew \RComb{\gamma}$, that is we have $\Tfr
    \CongrCAs{\gamma} \Tfr'$ if and only if $\Tfr \RewContextRST \Tfr'$.
    Hence, from the third part of the proof, $\Tfr \CongrCAs{\gamma}
    \Tfr'$ implies that
    \begin{math}
        \LRank\left(\Tfr'\right) - \LRank\left(\Tfr\right)
    \end{math}
    is divisible by $\gamma - 1$.
\end{proof}
\medbreak

\begin{Proposition} \label{prop:division_CAs}
    Let $\gamma$ and $\gamma'$ be two positive integers. Then, there
    exists a morphism $\varphi:\CAs{\gamma'} \to \CAs{\gamma}$ if and
    only if $\left(\gamma - 1\right) \mid \left(\gamma' - 1\right)$.
\end{Proposition}
\begin{proof}
    From Lemma~\ref{lem:morphism_CAs}, it is enough to show that
    $\LComb{\gamma'} \CongrCAs{\gamma} \RComb{\gamma'}$ if and only if
    $\left(\gamma-1\right) \mid \left(\gamma'-1\right)$. If
    $\LComb{\gamma'}\CongrCAs{\gamma}\RComb{\gamma'}$, from
    Lemma~\ref{lem:left_rank_and_CongrCAs},
    \begin{math}
        \LRank\left(\LComb{\gamma'}\right)
        - \LRank\left(\RComb{\gamma'}\right)
        = \gamma' - 1
    \end{math}
    is divisible by $\gamma-1$, which shows the direct implication.
    \smallbreak

    Conversely, if
    $\left(\gamma-1\right) \mid \left(\gamma'-1\right)$, the rewrite
    rule $\LComb{\gamma} \Rew \RComb{\gamma}$ induces the
    sequence
    \begin{multline}
        \LComb{\gamma'} \enspace = \enspace
        \begin{tikzpicture}[xscale=.19,yscale=.25,Centering]
            \node(0)at(0.00,-13.12){};
            \node(10)at(10.00,-5.62){};
            \node(12)at(12.00,-3.75){};
            \node(14)at(14.00,-1.88){};
            \node(2)at(2.00,-13.12){};
            \node(4)at(4.00,-11.25){};
            \node(6)at(6.00,-9.38){};
            \node(8)at(8.00,-7.50){};
            \node[NodeST](1)at(1.00,-11.25)
                {\begin{math}\Product\end{math}};
            \node[NodeST](11)at(11.00,-1.88)
                {\begin{math}\Product\end{math}};
            \node[NodeST](13)at(13.00,0.00)
                {\begin{math}\Product\end{math}};
            \node[NodeST](3)at(3.00,-9.38)
                {\begin{math}\Product\end{math}};
            \node[NodeST](5)at(5.00,-7.50)
                {\begin{math}\Product\end{math}};
            \node[NodeST](7)at(7.00,-5.62)
                {\begin{math}\Product\end{math}};
            \node[NodeST](9)at(9.00,-3.75)
                {\begin{math}\Product\end{math}};
            \draw[Edge](0)--(1);
            \draw[Edge,dotted](1)--(3);
            \draw[Edge](10)--(9);
            \draw[Edge](11)--(13);
            \draw[Edge](12)--(11);
            \draw[Edge](14)--(13);
            \draw[Edge](2)--(1);
            \draw[Edge,dotted](3)--(5);
            \draw[Edge](4)--(3);
            \draw[Edge,dotted](5)--(7);
            \draw[Edge](6)--(5);
            \draw[Edge](7)--(9);
            \draw[Edge](8)--(7);
            \draw[Edge,dotted](9)--(11);
            \node(r)at(13.00,1.41){};
            \draw[Edge](r)--(13);
            %
            \node[fit=(1)(3),rotate fit=0,inner sep=0pt,
                rounded corners,draw=Col5!80]{};
            \node[fit=(5)(7),rotate fit=0,inner sep=0pt,
                rounded corners,draw=Col5!80]{};
            \node[fit=(9)(11),rotate fit=0,inner sep=0pt,
                rounded corners,draw=Col5!80]{};
        \end{tikzpicture}
        \enspace \Rew \enspace
            \begin{tikzpicture}[xscale=.19,yscale=.2,Centering]
            \node(0)at(0.00,-12.50){};
            \node(10)at(10.00,-5.00){};
            \node(12)at(12.00,-7.50){};
            \node(14)at(14.00,-7.50){};
            \node(2)at(2.00,-12.50){};
            \node(4)at(4.00,-10.00){};
            \node(6)at(6.00,-7.50){};
            \node(8)at(8.00,-5.00){};
            \node[NodeST](1)at(1.00,-10.00)
                {\begin{math}\Product\end{math}};
            \node[NodeST](11)at(11.00,-2.50)
                {\begin{math}\Product\end{math}};
            \node[NodeST](13)at(13.00,-5.00)
                {\begin{math}\Product\end{math}};
            \node[NodeST](3)at(3.00,-7.50)
                {\begin{math}\Product\end{math}};
            \node[NodeST](5)at(5.00,-5.00)
                {\begin{math}\Product\end{math}};
            \node[NodeST](7)at(7.00,-2.50)
                {\begin{math}\Product\end{math}};
            \node[NodeST](9)at(9.00,0.00)
                {\begin{math}\Product\end{math}};
            \draw[Edge](0)--(1);
            \draw[Edge,dotted](1)--(3);
            \draw[Edge](10)--(11);
            \draw[Edge](11)--(9);
            \draw[Edge](12)--(13);
            \draw[Edge,dotted](13)--(11);
            \draw[Edge](14)--(13);
            \draw[Edge](2)--(1);
            \draw[Edge,dotted](3)--(5);
            \draw[Edge](4)--(3);
            \draw[Edge,dotted](5)--(7);
            \draw[Edge](6)--(5);
            \draw[Edge](7)--(9);
            \draw[Edge](8)--(7);
            \node(r)at(9.00,1.88){};
            \draw[Edge](r)--(9);
            %
            \node[fit=(1)(3),rotate fit=0,inner sep=0pt,
                rounded corners,draw=Col5!80]{};
            \node[fit=(5)(7),rotate fit=0,inner sep=0pt,
                rounded corners,draw=Col5!80]{};
            \node[fit=(11)(13),rotate fit=0,inner sep=0pt,
                rounded corners,draw=Col5!80]{};
        \end{tikzpicture}
        \enspace \Rew \enspace
        \begin{tikzpicture}[xscale=.19,yscale=.2,Centering]
            \node(0)at(0.00,-7.50){};
            \node(10)at(10.00,-10.00){};
            \node(12)at(12.00,-12.50){};
            \node(14)at(14.00,-12.50){};
            \node(2)at(2.00,-7.50){};
            \node(4)at(4.00,-5.00){};
            \node(6)at(6.00,-5.00){};
            \node(8)at(8.00,-7.50){};
            \node[NodeST](1)at(1.00,-5.00)
                {\begin{math}\Product\end{math}};
            \node[NodeST](11)at(11.00,-7.50)
                {\begin{math}\Product\end{math}};
            \node[NodeST](13)at(13.00,-10.00)
                {\begin{math}\Product\end{math}};
            \node[NodeST](3)at(3.00,-2.50)
                {\begin{math}\Product\end{math}};
            \node[NodeST](5)at(5.00,0.00)
                {\begin{math}\Product\end{math}};
            \node[NodeST](7)at(7.00,-2.50)
                {\begin{math}\Product\end{math}};
            \node[NodeST](9)at(9.00,-5.00)
                {\begin{math}\Product\end{math}};
            \draw[Edge](0)--(1);
            \draw[Edge,dotted](1)--(3);
            \draw[Edge](10)--(11);
            \draw[Edge](11)--(9);
            \draw[Edge](12)--(13);
            \draw[Edge,dotted](13)--(11);
            \draw[Edge](14)--(13);
            \draw[Edge](2)--(1);
            \draw[Edge,dotted](3)--(5);
            \draw[Edge](4)--(3);
            \draw[Edge](6)--(7);
            \draw[Edge](7)--(5);
            \draw[Edge](8)--(9);
            \draw[Edge,dotted](9)--(7);
            \node(r)at(5.00,1.88){};
            \draw[Edge](r)--(5);
            %
            \node[fit=(1)(3),rotate fit=0,inner sep=0pt,
                rounded corners,draw=Col5!80]{};
            \node[fit=(7)(9),rotate fit=0,inner sep=0pt,
                rounded corners,draw=Col5!80]{};
            \node[fit=(11)(13),rotate fit=0,inner sep=0pt,
                rounded corners,draw=Col5!80]{};
        \end{tikzpicture} \\
        \enspace \RewRT \enspace
        \begin{tikzpicture}[xscale=.19,yscale=.25,Centering]
            \node(0)at(0.00,-1.88){};
            \node(10)at(10.00,-11.25){};
            \node(12)at(12.00,-13.12){};
            \node(14)at(14.00,-13.12){};
            \node(2)at(2.00,-3.75){};
            \node(4)at(4.00,-5.62){};
            \node(6)at(6.00,-7.50){};
            \node(8)at(8.00,-9.38){};
            \node[NodeST](1)at(1.00,0.00)
                {\begin{math}\Product\end{math}};
            \node[NodeST](11)at(11.00,-9.38)
                {\begin{math}\Product\end{math}};
            \node[NodeST](13)at(13.00,-11.25)
                {\begin{math}\Product\end{math}};
            \node[NodeST](3)at(3.00,-1.88)
                {\begin{math}\Product\end{math}};
            \node[NodeST](5)at(5.00,-3.75)
                {\begin{math}\Product\end{math}};
            \node[NodeST](7)at(7.00,-5.62)
                {\begin{math}\Product\end{math}};
            \node[NodeST](9)at(9.00,-7.50)
                {\begin{math}\Product\end{math}};
            \draw[Edge](0)--(1);
            \draw[Edge](10)--(11);
            \draw[Edge,dotted](11)--(9);
            \draw[Edge](12)--(13);
            \draw[Edge,dotted](13)--(11);
            \draw[Edge](14)--(13);
            \draw[Edge](2)--(3);
            \draw[Edge](3)--(1);
            \draw[Edge](4)--(5);
            \draw[Edge,dotted](5)--(3);
            \draw[Edge](6)--(7);
            \draw[Edge](7)--(5);
            \draw[Edge](8)--(9);
            \draw[Edge,dotted](9)--(7);
            \node(r)at(1.00,1.41){};
            \draw[Edge](r)--(1);
            %
            \node[fit=(3)(5),rotate fit=0,inner sep=0pt,
                rounded corners,draw=Col5!80]{};
            \node[fit=(7)(9),rotate fit=0,inner sep=0pt,
                rounded corners,draw=Col5!80]{};
            \node[fit=(11)(13),rotate fit=0,inner sep=0pt,
                rounded corners,draw=Col5!80]{};
        \end{tikzpicture}
        \enspace = \enspace
        \RComb{\gamma'}
    \end{multline}
    of rewriting steps, where squared regions denotes left or right
    comb trees of degree $\gamma - 1$. Hence, we have
    $\LComb{\gamma'}\CongrCAs{\gamma}\RComb{\gamma'}$.
\end{proof}
\medbreak

Proposition~\ref{prop:division_CAs} implies the following result.
\medbreak

\begin{Theorem}\label{thm:lattice_CAs}
  The tuple $\LatCAs$ is a lattice, where $\InfCAs$ and $\SupCAs$ are
  defined, for any positive integers $\gamma$ and $\gamma'$, as follows:
    \begin{equation}
      \CAs{\gamma}\InfCAs\CAs{\gamma'}=\CAs{\gcd \left(\gamma-1, \gamma'-1\right)}
    \end{equation}
    and
    \begin{equation}
      \CAs{\gamma}\SupCAs\CAs{\gamma'}=\CAs{\lcm \left(\gamma-1, \gamma'-1\right)}.
    \end{equation}
\end{Theorem}
\medbreak

\begin{Remark}
  The lattice $\left(\N,\mid, \gcd, \lcm\right)$ admits $1$ as minimum
  element and $0$ as maximum element: any nonnegative integer is
  divisible by $1$ and divides $0$. The corresponding property for
  $\LatCAs$ is that it admits $\As=\CAs{2}$ as minimum element and $\Mag=
  \CAs{1}$ as maximum element: any comb associative operad projects onto
  $\As$ and is a quotient of $\Mag$.
\end{Remark}

We terminate this section by relating the lattice
$\LatQMag$ introduced in Section~\ref{sec:Magmatic_operads} to the
lattice $\LatCAs$.

As explained in Section~\ref{sec:operad_Mag}, a set-theoreic operad can
be embedded into a linear operad, so that the operads $\CAs{\gamma}$ can
be embedded into quotient operads $\KCAs{\gamma}$ of $\KMag$. Formally,
the operad $\KCAs{\gamma}$ is equal to $\KMag/I_{\gamma}$, where
$I_{\gamma}$ is the operadic ideal of $\KMag$ generated by
$\LComb{\gamma}-\RComb{\gamma}$.

We obtain a new lattice $\LatKCAs$, where $\KCAsAll$ is the set of all
operads $\KCAs{\gamma}$. In this linear framework, the condition
$\KCAs{\gamma}\OrdCAs\KCAs{\gamma'}$ means that the dimension of the
space $\text{hom}\left(\KCAs{\gamma'},\KCAs{\gamma}\right)$ is equal to
$1$. Hence, $\LatKCAs$ is related to $\LatQMag$ by the following
theorem.

\begin{Theorem}
  The inclusion
  $\iota:\left(\KCAsAll,\OrdCAs\right)\to\left(\QMag,\OrdQMag\right)$ is
  non decreasing. In particular, for any positive integers $\gamma$ and
  $\gamma'$, we have
  \begin{equation} \label{equ:comparison_of_lattice_operations}
    \KCAs{\gcd \left(\gamma-1, \gamma'-1\right)}\OrdQMag
    \KCAs{\gamma}\InfQMag\KCAs{\gamma'}
  \end{equation}
  and
  \begin{equation}
    \KCAs{\gamma}\SupQMag\KCAs{\gamma'}\OrdQMag
    \KCAs{\lcm \left(\gamma-1, \gamma'-1\right)}.
  \end{equation}

  \end{Theorem}

\begin{Remark}
  Note that $\LatKCAs$ does not embed into a sub-lattice of $\LatQMag$,
  that is $\iota$ is not a lattice morphism. Consider for instance
  $\gamma=3$ and $\gamma'=4$, then $\KCAs{\gamma}\InfCAs\KCAs{\gamma'}$
  is equal to $\KCAs{2}=\K\As$, whereas
  $\KCAs{\gamma}\InfQMag\KCAs{\gamma'}$ is quotient of $\KMag$ be the two
  relations $\LComb{3}-\RComb{3}$ and $\LComb{4}-\RComb{4}$.
\end{Remark}

%%%%%%%%%%%%%%%%%%%%%%%%%%%%%%%%%%%%%%%%%%%%%%%%%%%%%%%%%%%%%%%%%%%%%%%%
%%%%%%%%%%%%%%%%%%%%%%%%%%%%%%%%%%%%%%%%%%%%%%%%%%%%%%%%%%%%%%%%%%%%%%%%
%%%%%%%%%%%%%%%%%%%%%%%%%%%%%%%%%%%%%%%%%%%%%%%%%%%%%%%%%%%%%%%%%%%%%%%%
\section{The \texorpdfstring{$3$}{3}-comb associative operad}
\label{sec:CAs_3}
We now focus on the study of the operad $\CAs{3}$. By definition, this
operad is the quotient of $\Mag$ by the operad congruence spanned by the
relation
\begin{equation} \label{equ:rew_1}
    \begin{tikzpicture}[xscale=.22,yscale=.23,Centering]
        \node(0)at(0.00,-5.25){};
        \node(2)at(2.00,-5.25){};
        \node(4)at(4.00,-3.50){};
        \node(6)at(6.00,-1.75){};
        \node[NodeST](1)at(1.00,-3.50){\begin{math}\Product\end{math}};
        \node[NodeST](3)at(3.00,-1.75){\begin{math}\Product\end{math}};
        \node[NodeST](5)at(5.00,0.00){\begin{math}\Product\end{math}};
        \draw[Edge](0)--(1);
        \draw[Edge](1)--(3);
        \draw[Edge](2)--(1);
        \draw[Edge](3)--(5);
        \draw[Edge](4)--(3);
        \draw[Edge](6)--(5);
        \node(r)at(5.00,1.5){};
        \draw[Edge](r)--(5);
    \end{tikzpicture}
    \enspace \Rew \enspace
    \begin{tikzpicture}[xscale=.22,yscale=.23,Centering]
        \node(0)at(0.00,-1.75){};
        \node(2)at(2.00,-3.50){};
        \node(4)at(4.00,-5.25){};
        \node(6)at(6.00,-5.25){};
        \node[NodeST](1)at(1.00,0.00){\begin{math}\Product\end{math}};
        \node[NodeST](3)at(3.00,-1.75){\begin{math}\Product\end{math}};
        \node[NodeST](5)at(5.00,-3.50){\begin{math}\Product\end{math}};
        \draw[Edge](0)--(1);
        \draw[Edge](2)--(3);
        \draw[Edge](3)--(1);
        \draw[Edge](4)--(5);
        \draw[Edge](5)--(3);
        \draw[Edge](6)--(5);
        \node(r)at(1.00,1.5){};
        \draw[Edge](r)--(1);
    \end{tikzpicture}\,.
\end{equation}
This rewrite rule is compatible with the lexicographic order on prefix
words presented at the beginning of Section~\ref{sec:operad_Mag} in the
sense that the prefix word of the left member of~\eqref{equ:rew_1} is
lexicographically greater than the prefix word of the right one.
\medbreak

However, the rewrite relation $\RewContext$ induced by $\Rew$ is not
confluent. Indeed, we have
\begin{equation} \label{equ:branching_pair_CAs_3}
    \begin{tikzpicture}[xscale=.22,yscale=.22,Centering]
        \node(0)at(0.00,-7.20){};
        \node(2)at(2.00,-7.20){};
        \node(4)at(4.00,-5.40){};
        \node(6)at(6.00,-3.60){};
        \node(8)at(8.00,-1.80){};
        \node[NodeST](1)at(1.00,-5.40){\begin{math}\Product\end{math}};
        \node[NodeST](3)at(3.00,-3.60){\begin{math}\Product\end{math}};
        \node[NodeST](5)at(5.00,-1.80){\begin{math}\Product\end{math}};
        \node[NodeST](7)at(7.00,0.00){\begin{math}\Product\end{math}};
        \draw[Edge](0)--(1);
        \draw[Edge](1)--(3);
        \draw[Edge](2)--(1);
        \draw[Edge](3)--(5);
        \draw[Edge](4)--(3);
        \draw[Edge](5)--(7);
        \draw[Edge](6)--(5);
        \draw[Edge](8)--(7);
        \node(r)at(7.00,1.35){};
        \draw[Edge](r)--(7);
    \end{tikzpicture}
    \enspace \RewContext \enspace
    \begin{tikzpicture}[xscale=.22,yscale=.20,Centering]
        \node(0)at(0.00,-4.50){};
        \node(2)at(2.00,-4.50){};
        \node(4)at(4.00,-4.50){};
        \node(6)at(6.00,-6.75){};
        \node(8)at(8.00,-6.75){};
        \node[NodeST](1)at(1.00,-2.25){\begin{math}\Product\end{math}};
        \node[NodeST](3)at(3.00,0.00){\begin{math}\Product\end{math}};
        \node[NodeST](5)at(5.00,-2.25){\begin{math}\Product\end{math}};
        \node[NodeST](7)at(7.00,-4.50){\begin{math}\Product\end{math}};
        \draw[Edge](0)--(1);
        \draw[Edge](1)--(3);
        \draw[Edge](2)--(1);
        \draw[Edge](4)--(5);
        \draw[Edge](5)--(3);
        \draw[Edge](6)--(7);
        \draw[Edge](7)--(5);
        \draw[Edge](8)--(7);
        \node(r)at(3.00,1.74){};
        \draw[Edge](r)--(3);
    \end{tikzpicture}
    \qquad \mbox{and} \qquad
    \begin{tikzpicture}[xscale=.22,yscale=.22,Centering]
        \node(0)at(0.00,-7.20){};
        \node(2)at(2.00,-7.20){};
        \node(4)at(4.00,-5.40){};
        \node(6)at(6.00,-3.60){};
        \node(8)at(8.00,-1.80){};
        \node[NodeST](1)at(1.00,-5.40){\begin{math}\Product\end{math}};
        \node[NodeST](3)at(3.00,-3.60){\begin{math}\Product\end{math}};
        \node[NodeST](5)at(5.00,-1.80){\begin{math}\Product\end{math}};
        \node[NodeST](7)at(7.00,0.00){\begin{math}\Product\end{math}};
        \draw[Edge](0)--(1);
        \draw[Edge](1)--(3);
        \draw[Edge](2)--(1);
        \draw[Edge](3)--(5);
        \draw[Edge](4)--(3);
        \draw[Edge](5)--(7);
        \draw[Edge](6)--(5);
        \draw[Edge](8)--(7);
        \node(r)at(7.00,1.35){};
        \draw[Edge](r)--(7);
    \end{tikzpicture}
    \enspace \RewContext \enspace
    \begin{tikzpicture}[xscale=.22,yscale=.22,Centering]
        \node(0)at(0.00,-3.60){};
        \node(2)at(2.00,-5.40){};
        \node(4)at(4.00,-7.20){};
        \node(6)at(6.00,-7.20){};
        \node(8)at(8.00,-1.80){};
        \node[NodeST](1)at(1.00,-1.80){\begin{math}\Product\end{math}};
        \node[NodeST](3)at(3.00,-3.60){\begin{math}\Product\end{math}};
        \node[NodeST](5)at(5.00,-5.40){\begin{math}\Product\end{math}};
        \node[NodeST](7)at(7.00,0.00){\begin{math}\Product\end{math}};
        \draw[Edge](0)--(1);
        \draw[Edge](1)--(7);
        \draw[Edge](2)--(3);
        \draw[Edge](3)--(1);
        \draw[Edge](4)--(5);
        \draw[Edge](5)--(3);
        \draw[Edge](6)--(5);
        \draw[Edge](8)--(7);
        \node(r)at(7.00,1.5){};
        \draw[Edge](r)--(7);
    \end{tikzpicture}\,,
\end{equation}
and the two right members of~\eqref{equ:branching_pair_CAs_3} form a
branching pair which is not joinable.
\medbreak

In order to transform the rewrite relation induced by~\eqref{equ:rew_1}
into a convergent one, we apply the Buchberger algorithm for
operads~\cite[Section 3.7]{DK10} with respect to the lexicographic order
on prefix words. Following this algorithm, we need to put the right
members of~\eqref{equ:branching_pair_CAs_3} in relation by $\Rew$. To
respect the lexicographic property of the prefix words, this leads to
the new relation
\begin{equation} \label{equ:rew_2}
    \begin{tikzpicture}[xscale=.22,yscale=.22,Centering]
        \node(0)at(0.00,-3.60){};
        \node(2)at(2.00,-5.40){};
        \node(4)at(4.00,-7.20){};
        \node(6)at(6.00,-7.20){};
        \node(8)at(8.00,-1.80){};
        \node[NodeST](1)at(1.00,-1.80){\begin{math}\Product\end{math}};
        \node[NodeST](3)at(3.00,-3.60){\begin{math}\Product\end{math}};
        \node[NodeST](5)at(5.00,-5.40){\begin{math}\Product\end{math}};
        \node[NodeST](7)at(7.00,0.00){\begin{math}\Product\end{math}};
        \draw[Edge](0)--(1);
        \draw[Edge](1)--(7);
        \draw[Edge](2)--(3);
        \draw[Edge](3)--(1);
        \draw[Edge](4)--(5);
        \draw[Edge](5)--(3);
        \draw[Edge](6)--(5);
        \draw[Edge](8)--(7);
        \node(r)at(7.00,1.5){};
        \draw[Edge](r)--(7);
    \end{tikzpicture}
    \enspace \Rew \enspace
    \begin{tikzpicture}[xscale=.22,yscale=.20,Centering]
        \node(0)at(0.00,-4.50){};
        \node(2)at(2.00,-4.50){};
        \node(4)at(4.00,-4.50){};
        \node(6)at(6.00,-6.75){};
        \node(8)at(8.00,-6.75){};
        \node[NodeST](1)at(1.00,-2.25){\begin{math}\Product\end{math}};
        \node[NodeST](3)at(3.00,0.00){\begin{math}\Product\end{math}};
        \node[NodeST](5)at(5.00,-2.25){\begin{math}\Product\end{math}};
        \node[NodeST](7)at(7.00,-4.50){\begin{math}\Product\end{math}};
        \draw[Edge](0)--(1);
        \draw[Edge](1)--(3);
        \draw[Edge](2)--(1);
        \draw[Edge](4)--(5);
        \draw[Edge](5)--(3);
        \draw[Edge](6)--(7);
        \draw[Edge](7)--(5);
        \draw[Edge](8)--(7);
        \node(r)at(3.00,1.74){};
        \draw[Edge](r)--(3);
    \end{tikzpicture}\,.
\end{equation}
The Buchberger algorithm applied on binary trees of degrees $5$, $6$,
and $7$ provides the new relations \\
\begin{minipage}{7cm}
\begin{equation} \label{equ:rew_3}
    \begin{tikzpicture}[xscale=.23,yscale=.21,Centering]
        \node(0)at(0.00,-1.83){};
        \node(10)at(10.00,-5.50){};
        \node(2)at(2.00,-3.67){};
        \node(4)at(4.00,-7.33){};
        \node(6)at(6.00,-9.17){};
        \node(8)at(8.00,-9.17){};
        \node[NodeST](1)at(1.00,0.00){\begin{math}\Product\end{math}};
        \node[NodeST](3)at(3.00,-1.83){\begin{math}\Product\end{math}};
        \node[NodeST](5)at(5.00,-5.50){\begin{math}\Product\end{math}};
        \node[NodeST](7)at(7.00,-7.33){\begin{math}\Product\end{math}};
        \node[NodeST](9)at(9.00,-3.67){\begin{math}\Product\end{math}};
        \draw[Edge](0)--(1);
        \draw[Edge](10)--(9);
        \draw[Edge](2)--(3);
        \draw[Edge](3)--(1);
        \draw[Edge](4)--(5);
        \draw[Edge](5)--(9);
        \draw[Edge](6)--(7);
        \draw[Edge](7)--(5);
        \draw[Edge](8)--(7);
        \draw[Edge](9)--(3);
        \node(r)at(1.00,1.75){};
        \draw[Edge](r)--(1);
    \end{tikzpicture}
    \enspace \Rew \enspace
    \begin{tikzpicture}[xscale=.22,yscale=.24,Centering]
        \node(0)at(0.00,-1.83){};
        \node(10)at(10.00,-9.17){};
        \node(2)at(2.00,-3.67){};
        \node(4)at(4.00,-5.50){};
        \node(6)at(6.00,-7.33){};
        \node(8)at(8.00,-9.17){};
        \node[NodeST](1)at(1.00,0.00){\begin{math}\Product\end{math}};
        \node[NodeST](3)at(3.00,-1.83){\begin{math}\Product\end{math}};
        \node[NodeST](5)at(5.00,-3.67){\begin{math}\Product\end{math}};
        \node[NodeST](7)at(7.00,-5.50){\begin{math}\Product\end{math}};
        \node[NodeST](9)at(9.00,-7.33){\begin{math}\Product\end{math}};
        \draw[Edge](0)--(1);
        \draw[Edge](10)--(9);
        \draw[Edge](2)--(3);
        \draw[Edge](3)--(1);
        \draw[Edge](4)--(5);
        \draw[Edge](5)--(3);
        \draw[Edge](6)--(7);
        \draw[Edge](7)--(5);
        \draw[Edge](8)--(9);
        \draw[Edge](9)--(7);
        \node(r)at(1.00,1.5){};
        \draw[Edge](r)--(1);
    \end{tikzpicture}
\end{equation}
\end{minipage},
\begin{minipage}{7cm}
\begin{equation}\label{equ:rew_4}
     \begin{tikzpicture}[xscale=.2,yscale=.19,Centering]
        \node(0)at(0.00,-2.20){};
        \node(10)at(10.00,-6.60){};
        \node(2)at(2.00,-6.60){};
        \node(4)at(4.00,-8.80){};
        \node(6)at(6.00,-8.80){};
        \node(8)at(8.00,-6.60){};
        \node[NodeST](1)at(1.00,0.00){\begin{math}\Product\end{math}};
        \node[NodeST](3)at(3.00,-4.40){\begin{math}\Product\end{math}};
        \node[NodeST](5)at(5.00,-6.60){\begin{math}\Product\end{math}};
        \node[NodeST](7)at(7.00,-2.20){\begin{math}\Product\end{math}};
        \node[NodeST](9)at(9.00,-4.40){\begin{math}\Product\end{math}};
        \draw[Edge](0)--(1);
        \draw[Edge](10)--(9);
        \draw[Edge](2)--(3);
        \draw[Edge](3)--(7);
        \draw[Edge](4)--(5);
        \draw[Edge](5)--(3);
        \draw[Edge](6)--(5);
        \draw[Edge](7)--(1);
        \draw[Edge](8)--(9);
        \draw[Edge](9)--(7);
        \node(r)at(1.00,2){};
        \draw[Edge](r)--(1);
    \end{tikzpicture}
    \enspace \Rew \enspace
    \begin{tikzpicture}[xscale=.22,yscale=.24,Centering]
        \node(0)at(0.00,-1.83){};
        \node(10)at(10.00,-9.17){};
        \node(2)at(2.00,-3.67){};
        \node(4)at(4.00,-5.50){};
        \node(6)at(6.00,-7.33){};
        \node(8)at(8.00,-9.17){};
        \node[NodeST](1)at(1.00,0.00){\begin{math}\Product\end{math}};
        \node[NodeST](3)at(3.00,-1.83){\begin{math}\Product\end{math}};
        \node[NodeST](5)at(5.00,-3.67){\begin{math}\Product\end{math}};
        \node[NodeST](7)at(7.00,-5.50){\begin{math}\Product\end{math}};
        \node[NodeST](9)at(9.00,-7.33){\begin{math}\Product\end{math}};
        \draw[Edge](0)--(1);
        \draw[Edge](10)--(9);
        \draw[Edge](2)--(3);
        \draw[Edge](3)--(1);
        \draw[Edge](4)--(5);
        \draw[Edge](5)--(3);
        \draw[Edge](6)--(7);
        \draw[Edge](7)--(5);
        \draw[Edge](8)--(9);
        \draw[Edge](9)--(7);
        \node(r)at(1.00,1.75){};
        \draw[Edge](r)--(1);
    \end{tikzpicture}
\end{equation}
\end{minipage}, \\
\begin{minipage}{7cm}
\begin{equation} \label{equ:rew_5}
    \begin{tikzpicture}[xscale=.2,yscale=.2,Centering]
        \node(0)at(0.00,-2.17){};
        \node(10)at(10.00,-10.83){};
        \node(12)at(12.00,-10.83){};
        \node(2)at(2.00,-4.33){};
        \node(4)at(4.00,-8.67){};
        \node(6)at(6.00,-8.67){};
        \node(8)at(8.00,-8.67){};
        \node[NodeST](1)at(1.00,0.00){\begin{math}\Product\end{math}};
        \node[NodeST](11)at(11.00,-8.67)
            {\begin{math}\Product\end{math}};
        \node[NodeST](3)at(3.00,-2.17){\begin{math}\Product\end{math}};
        \node[NodeST](5)at(5.00,-6.50){\begin{math}\Product\end{math}};
        \node[NodeST](7)at(7.00,-4.33){\begin{math}\Product\end{math}};
        \node[NodeST](9)at(9.00,-6.50){\begin{math}\Product\end{math}};
        \draw[Edge](0)--(1);
        \draw[Edge](10)--(11);
        \draw[Edge](11)--(9);
        \draw[Edge](12)--(11);
        \draw[Edge](2)--(3);
        \draw[Edge](3)--(1);
        \draw[Edge](4)--(5);
        \draw[Edge](5)--(7);
        \draw[Edge](6)--(5);
        \draw[Edge](7)--(3);
        \draw[Edge](8)--(9);
        \draw[Edge](9)--(7);
        \node(r)at(1.00,1.75){};
        \draw[Edge](r)--(1);
    \end{tikzpicture}
    \Rew
    \begin{tikzpicture}[xscale=.21,yscale=.22,Centering]
        \node(0)at(0.00,-1.86){};
        \node(10)at(10.00,-11.14){};
        \node(12)at(12.00,-9.29){};
        \node(2)at(2.00,-3.71){};
        \node(4)at(4.00,-5.57){};
        \node(6)at(6.00,-7.43){};
        \node(8)at(8.00,-11.14){};
        \node[NodeST](1)at(1.00,0.00){\begin{math}\Product\end{math}};
        \node[NodeST](11)at(11.00,-7.43)
            {\begin{math}\Product\end{math}};
        \node[NodeST](3)at(3.00,-1.86){\begin{math}\Product\end{math}};
        \node[NodeST](5)at(5.00,-3.71){\begin{math}\Product\end{math}};
        \node[NodeST](7)at(7.00,-5.57){\begin{math}\Product\end{math}};
        \node[NodeST](9)at(9.00,-9.29){\begin{math}\Product\end{math}};
        \draw[Edge](0)--(1);
        \draw[Edge](10)--(9);
        \draw[Edge](11)--(7);
        \draw[Edge](12)--(11);
        \draw[Edge](2)--(3);
        \draw[Edge](3)--(1);
        \draw[Edge](4)--(5);
        \draw[Edge](5)--(3);
        \draw[Edge](6)--(7);
        \draw[Edge](7)--(5);
        \draw[Edge](8)--(9);
        \draw[Edge](9)--(11);
        \node(r)at(1.00,1.75){};
        \draw[Edge](r)--(1);
    \end{tikzpicture}
\end{equation}
\end{minipage},
\begin{minipage}{7cm}
\begin{equation} \label{equ:rew_6}
    \begin{tikzpicture}[xscale=.21,yscale=.21,Centering]
        \node(0)at(0.00,-2.17){};
        \node(10)at(10.00,-10.83){};
        \node(12)at(12.00,-10.83){};
        \node(2)at(2.00,-6.50){};
        \node(4)at(4.00,-6.50){};
        \node(6)at(6.00,-6.50){};
        \node(8)at(8.00,-8.67){};
        \node[NodeST](1)at(1.00,0.00){\begin{math}\Product\end{math}};
        \node[NodeST](11)at(11.00,-8.67)
            {\begin{math}\Product\end{math}};
        \node[NodeST](3)at(3.00,-4.33){\begin{math}\Product\end{math}};
        \node[NodeST](5)at(5.00,-2.17){\begin{math}\Product\end{math}};
        \node[NodeST](7)at(7.00,-4.33){\begin{math}\Product\end{math}};
        \node[NodeST](9)at(9.00,-6.50){\begin{math}\Product\end{math}};
        \draw[Edge](0)--(1);
        \draw[Edge](10)--(11);
        \draw[Edge](11)--(9);
        \draw[Edge](12)--(11);
        \draw[Edge](2)--(3);
        \draw[Edge](3)--(5);
        \draw[Edge](4)--(3);
        \draw[Edge](5)--(1);
        \draw[Edge](6)--(7);
        \draw[Edge](7)--(5);
        \draw[Edge](8)--(9);
        \draw[Edge](9)--(7);
        \node(r)at(1.00,1.75){};
        \draw[Edge](r)--(1);
    \end{tikzpicture}
    \Rew
    \begin{tikzpicture}[xscale=.22,yscale=.21,Centering]
        \node(0)at(0.00,-2.17){};
        \node(10)at(10.00,-10.83){};
        \node(12)at(12.00,-10.83){};
        \node(2)at(2.00,-4.33){};
        \node(4)at(4.00,-6.50){};
        \node(6)at(6.00,-10.83){};
        \node(8)at(8.00,-10.83){};
        \node[NodeST](1)at(1.00,0.00){\begin{math}\Product\end{math}};
        \node[NodeST](11)at(11.00,-8.67)
            {\begin{math}\Product\end{math}};
        \node[NodeST](3)at(3.00,-2.17){\begin{math}\Product\end{math}};
        \node[NodeST](5)at(5.00,-4.33){\begin{math}\Product\end{math}};
        \node[NodeST](7)at(7.00,-8.67){\begin{math}\Product\end{math}};
        \node[NodeST](9)at(9.00,-6.50){\begin{math}\Product\end{math}};
        \draw[Edge](0)--(1);
        \draw[Edge](10)--(11);
        \draw[Edge](11)--(9);
        \draw[Edge](12)--(11);
        \draw[Edge](2)--(3);
        \draw[Edge](3)--(1);
        \draw[Edge](4)--(5);
        \draw[Edge](5)--(3);
        \draw[Edge](6)--(7);
        \draw[Edge](7)--(9);
        \draw[Edge](8)--(7);
        \draw[Edge](9)--(5);
        \node(r)at(1.00,1.75){};
        \draw[Edge](r)--(1);
    \end{tikzpicture}\hspace{-8pt}
\end{equation}
\end{minipage}, \\
\begin{minipage}{7cm}
\begin{equation} \label{equ:rew_7}
    \begin{tikzpicture}[xscale=.19,yscale=.17,Centering]
        \node(0)at(0.00,-5.20){};
        \node(10)at(10.00,-7.80){};
        \node(12)at(12.00,-7.80){};
        \node(2)at(2.00,-10.40){};
        \node(4)at(4.00,-10.40){};
        \node(6)at(6.00,-7.80){};
        \node(8)at(8.00,-5.20){};
        \node[NodeST](1)at(1.00,-2.60){\begin{math}\Product\end{math}};
        \node[NodeST](11)at(11.00,-5.20)
            {\begin{math}\Product\end{math}};
        \node[NodeST](3)at(3.00,-7.80){\begin{math}\Product\end{math}};
        \node[NodeST](5)at(5.00,-5.20){\begin{math}\Product\end{math}};
        \node[NodeST](7)at(7.00,0.00){\begin{math}\Product\end{math}};
        \node[NodeST](9)at(9.00,-2.60){\begin{math}\Product\end{math}};
        \draw[Edge](0)--(1);
        \draw[Edge](1)--(7);
        \draw[Edge](10)--(11);
        \draw[Edge](11)--(9);
        \draw[Edge](12)--(11);
        \draw[Edge](2)--(3);
        \draw[Edge](3)--(5);
        \draw[Edge](4)--(3);
        \draw[Edge](5)--(1);
        \draw[Edge](6)--(5);
        \draw[Edge](8)--(9);
        \draw[Edge](9)--(7);
        \node(r)at(7.00,2){};
        \draw[Edge](r)--(7);
    \end{tikzpicture}
    \enspace \Rew \enspace
        \begin{tikzpicture}[xscale=.2,yscale=.19,Centering]
        \node(0)at(0.00,-4.33){};
        \node(10)at(10.00,-10.83){};
        \node(12)at(12.00,-10.83){};
        \node(2)at(2.00,-4.33){};
        \node(4)at(4.00,-4.33){};
        \node(6)at(6.00,-6.50){};
        \node(8)at(8.00,-8.67){};
        \node[NodeST](1)at(1.00,-2.17){\begin{math}\Product\end{math}};
        \node[NodeST](11)at(11.00,-8.67)
            {\begin{math}\Product\end{math}};
        \node[NodeST](3)at(3.00,0.00){\begin{math}\Product\end{math}};
        \node[NodeST](5)at(5.00,-2.17){\begin{math}\Product\end{math}};
        \node[NodeST](7)at(7.00,-4.33){\begin{math}\Product\end{math}};
        \node[NodeST](9)at(9.00,-6.50){\begin{math}\Product\end{math}};
        \draw[Edge](0)--(1);
        \draw[Edge](1)--(3);
        \draw[Edge](10)--(11);
        \draw[Edge](11)--(9);
        \draw[Edge](12)--(11);
        \draw[Edge](2)--(1);
        \draw[Edge](4)--(5);
        \draw[Edge](5)--(3);
        \draw[Edge](6)--(7);
        \draw[Edge](7)--(5);
        \draw[Edge](8)--(9);
        \draw[Edge](9)--(7);
        \node(r)at(3.00,1.75){};
        \draw[Edge](r)--(3);
    \end{tikzpicture}\hspace{-10pt}
\end{equation}
\end{minipage},
\begin{minipage}{7cm}
\begin{equation} \label{equ:rew_8}
    \begin{tikzpicture}[xscale=.2,yscale=.19,Centering]
        \node(0)at(0.00,-2.14){};
        \node(10)at(10.00,-12.86){};
        \node(12)at(12.00,-12.86){};
        \node(14)at(14.00,-12.86){};
        \node(2)at(2.00,-4.29){};
        \node(4)at(4.00,-6.43){};
        \node(6)at(6.00,-8.57){};
        \node(8)at(8.00,-12.86){};
        \node[NodeST](1)at(1.00,0.00){\begin{math}\Product\end{math}};
        \node[NodeST](11)at(11.00,-8.57)
            {\begin{math}\Product\end{math}};
        \node[NodeST](13)at(13.00,-10.71)
            {\begin{math}\Product\end{math}};
        \node[NodeST](3)at(3.00,-2.14){\begin{math}\Product\end{math}};
        \node[NodeST](5)at(5.00,-4.29){\begin{math}\Product\end{math}};
        \node[NodeST](7)at(7.00,-6.43){\begin{math}\Product\end{math}};
        \node[NodeST](9)at(9.00,-10.71){\begin{math}\Product\end{math}};
        \draw[Edge](0)--(1);
        \draw[Edge](10)--(9);
        \draw[Edge](11)--(7);
        \draw[Edge](12)--(13);
        \draw[Edge](13)--(11);
        \draw[Edge](14)--(13);
        \draw[Edge](2)--(3);
        \draw[Edge](3)--(1);
        \draw[Edge](4)--(5);
        \draw[Edge](5)--(3);
        \draw[Edge](6)--(7);
        \draw[Edge](7)--(5);
        \draw[Edge](8)--(9);
        \draw[Edge](9)--(11);
        \node(r)at(1.00,2){};
        \draw[Edge](r)--(1);
    \end{tikzpicture}
    \hspace{-10pt}\Rew
    \begin{tikzpicture}[xscale=.22,yscale=.2,Centering]
        \node(0)at(0.00,-1.88){};
        \node(10)at(10.00,-13.12){};
        \node(12)at(12.00,-13.12){};
        \node(14)at(14.00,-11.25){};
        \node(2)at(2.00,-3.75){};
        \node(4)at(4.00,-5.62){};
        \node(6)at(6.00,-7.50){};
        \node(8)at(8.00,-9.38){};
        \node[NodeST](1)at(1.00,0.00){\begin{math}\Product\end{math}};
        \node[NodeST](11)at(11.00,-11.25)
            {\begin{math}\Product\end{math}};
        \node[NodeST](13)at(13.00,-9.38)
            {\begin{math}\Product\end{math}};
        \node[NodeST](3)at(3.00,-1.88){\begin{math}\Product\end{math}};
        \node[NodeST](5)at(5.00,-3.75){\begin{math}\Product\end{math}};
        \node[NodeST](7)at(7.00,-5.62){\begin{math}\Product\end{math}};
        \node[NodeST](9)at(9.00,-7.50){\begin{math}\Product\end{math}};
        \draw[Edge](0)--(1);
        \draw[Edge](10)--(11);
        \draw[Edge](11)--(13);
        \draw[Edge](12)--(11);
        \draw[Edge](13)--(9);
        \draw[Edge](14)--(13);
        \draw[Edge](2)--(3);
        \draw[Edge](3)--(1);
        \draw[Edge](4)--(5);
        \draw[Edge](5)--(3);
        \draw[Edge](6)--(7);
        \draw[Edge](7)--(5);
        \draw[Edge](8)--(9);
        \draw[Edge](9)--(7);
        \node(r)at(1.00,2){};
        \draw[Edge](r)--(1);
    \end{tikzpicture}\hspace{-20pt}
\end{equation}
\end{minipage}, \\
\begin{minipage}{7cm}
\begin{equation} \label{equ:rew_9}
    \begin{tikzpicture}[xscale=0.18,yscale=.2,Centering]
        \node(0)at(0.00,-2.14){};
        \node(10)at(10.00,-12.86){};
        \node(12)at(12.00,-12.86){};
        \node(14)at(14.00,-10.71){};
        \node(2)at(2.00,-4.29){};
        \node(4)at(4.00,-6.43){};
        \node(6)at(6.00,-10.71){};
        \node(8)at(8.00,-10.71){};
        \node[NodeST](1)at(1.00,0.00){\begin{math}\Product\end{math}};
        \node[NodeST](11)at(11.00,-10.71)
            {\begin{math}\Product\end{math}};
        \node[NodeST](13)at(13.00,-8.57)
            {\begin{math}\Product\end{math}};
        \node[NodeST](3)at(3.00,-2.14){\begin{math}\Product\end{math}};
        \node[NodeST](5)at(5.00,-4.29){\begin{math}\Product\end{math}};
        \node[NodeST](7)at(7.00,-8.57){\begin{math}\Product\end{math}};
        \node[NodeST](9)at(9.00,-6.43){\begin{math}\Product\end{math}};
        \draw[Edge](0)--(1);
        \draw[Edge](10)--(11);
        \draw[Edge](11)--(13);
        \draw[Edge](12)--(11);
        \draw[Edge](13)--(9);
        \draw[Edge](14)--(13);
        \draw[Edge](2)--(3);
        \draw[Edge](3)--(1);
        \draw[Edge](4)--(5);
        \draw[Edge](5)--(3);
        \draw[Edge](6)--(7);
        \draw[Edge](7)--(9);
        \draw[Edge](8)--(7);
        \draw[Edge](9)--(5);
        \node(r)at(1.00,2){};
        \draw[Edge](r)--(1);
    \end{tikzpicture}
    \hspace{-5pt} \Rew \hspace{-5pt}
    \begin{tikzpicture}[xscale=.22,yscale=.22,Centering]
        \node(0)at(0.00,-1.88){};
        \node(10)at(10.00,-11.25){};
        \node(12)at(12.00,-13.12){};
        \node(14)at(14.00,-13.12){};
        \node(2)at(2.00,-3.75){};
        \node(4)at(4.00,-5.62){};
        \node(6)at(6.00,-7.50){};
        \node(8)at(8.00,-9.38){};
        \node[NodeST](1)at(1.00,0.00){\begin{math}\Product\end{math}};
        \node[NodeST](11)at(11.00,-9.38)
            {\begin{math}\Product\end{math}};
        \node[NodeST](13)at(13.00,-11.25)
            {\begin{math}\Product\end{math}};
        \node[NodeST](3)at(3.00,-1.88){\begin{math}\Product\end{math}};
        \node[NodeST](5)at(5.00,-3.75){\begin{math}\Product\end{math}};
        \node[NodeST](7)at(7.00,-5.62){\begin{math}\Product\end{math}};
        \node[NodeST](9)at(9.00,-7.50){\begin{math}\Product\end{math}};
        \draw[Edge](0)--(1);
        \draw[Edge](10)--(11);
        \draw[Edge](11)--(9);
        \draw[Edge](12)--(13);
        \draw[Edge](13)--(11);
        \draw[Edge](14)--(13);
        \draw[Edge](2)--(3);
        \draw[Edge](3)--(1);
        \draw[Edge](4)--(5);
        \draw[Edge](5)--(3);
        \draw[Edge](6)--(7);
        \draw[Edge](7)--(5);
        \draw[Edge](8)--(9);
        \draw[Edge](9)--(7);
        \node(r)at(1.00,1.5){};
        \draw[Edge](r)--(1);
    \end{tikzpicture}\hspace{-25pt}
\end{equation}
\end{minipage},
\begin{minipage}{7cm}
\begin{equation} \label{equ:rew_10}
    \hspace{-2pt}\begin{tikzpicture}[xscale=.2,yscale=.17,Centering]
        \node(0)at(0.00,-5.00){};
        \node(10)at(10.00,-12.50){};
        \node(12)at(12.00,-12.50){};
        \node(14)at(14.00,-12.50){};
        \node(2)at(2.00,-5.00){};
        \node(4)at(4.00,-5.00){};
        \node(6)at(6.00,-7.50){};
        \node(8)at(8.00,-12.50){};
        \node[NodeST](1)at(1.00,-2.50){\begin{math}\Product\end{math}};
        \node[NodeST](11)at(11.00,-7.50)
            {\begin{math}\Product\end{math}};
        \node[NodeST](13)at(13.00,-10.00)
            {\begin{math}\Product\end{math}};
        \node[NodeST](3)at(3.00,0.00){\begin{math}\Product\end{math}};
        \node[NodeST](5)at(5.00,-2.50){\begin{math}\Product\end{math}};
        \node[NodeST](7)at(7.00,-5.00){\begin{math}\Product\end{math}};
        \node[NodeST](9)at(9.00,-10.00){\begin{math}\Product\end{math}};
        \draw[Edge](0)--(1);
        \draw[Edge](1)--(3);
        \draw[Edge](10)--(9);
        \draw[Edge](11)--(7);
        \draw[Edge](12)--(13);
        \draw[Edge](13)--(11);
        \draw[Edge](14)--(13);
        \draw[Edge](2)--(1);
        \draw[Edge](4)--(5);
        \draw[Edge](5)--(3);
        \draw[Edge](6)--(7);
        \draw[Edge](7)--(5);
        \draw[Edge](8)--(9);
        \draw[Edge](9)--(11);
        \node(r)at(3.00,2){};
        \draw[Edge](r)--(3);
    \end{tikzpicture}
    \hspace{-10pt} \Rew \hspace{5pt}
    \begin{tikzpicture}[xscale=.21,yscale=.2,Centering]
        \node(0)at(0.00,-4.29){};
        \node(10)at(10.00,-12.86){};
        \node(12)at(12.00,-12.86){};
        \node(14)at(14.00,-10.71){};
        \node(2)at(2.00,-4.29){};
        \node(4)at(4.00,-4.29){};
        \node(6)at(6.00,-6.43){};
        \node(8)at(8.00,-8.57){};
        \node[NodeST](1)at(1.00,-2.14){\begin{math}\Product\end{math}};
        \node[NodeST](11)at(11.00,-10.71)
            {\begin{math}\Product\end{math}};
        \node[NodeST](13)at(13.00,-8.57)
            {\begin{math}\Product\end{math}};
        \node[NodeST](3)at(3.00,0.00){\begin{math}\Product\end{math}};
        \node[NodeST](5)at(5.00,-2.14){\begin{math}\Product\end{math}};
        \node[NodeST](7)at(7.00,-4.29){\begin{math}\Product\end{math}};
        \node[NodeST](9)at(9.00,-6.43){\begin{math}\Product\end{math}};
        \draw[Edge](0)--(1);
        \draw[Edge](1)--(3);
        \draw[Edge](10)--(11);
        \draw[Edge](11)--(13);
        \draw[Edge](12)--(11);
        \draw[Edge](13)--(9);
        \draw[Edge](14)--(13);
        \draw[Edge](2)--(1);
        \draw[Edge](4)--(5);
        \draw[Edge](5)--(3);
        \draw[Edge](6)--(7);
        \draw[Edge](7)--(5);
        \draw[Edge](8)--(9);
        \draw[Edge](9)--(7);
        \node(r)at(3.00,2){};
        \draw[Edge](r)--(3);
    \end{tikzpicture} \hspace{-25pt}
\end{equation}
\end{minipage}, \\
\begin{minipage}{8.8cm}
\begin{equation} \label{equ:rew_11}
    \begin{tikzpicture}[xscale=.22,yscale=.18,Centering]
        \node(0)at(0.00,-5.00){};
        \node(10)at(10.00,-10.00){};
        \node(12)at(12.00,-12.50){};
        \node(14)at(14.00,-12.50){};
        \node(2)at(2.00,-7.50){};
        \node(4)at(4.00,-7.50){};
        \node(6)at(6.00,-5.00){};
        \node(8)at(8.00,-7.50){};
        \node[NodeST](1)at(1.00,-2.50){\begin{math}\Product\end{math}};
        \node[NodeST](11)at(11.00,-7.50)
            {\begin{math}\Product\end{math}};
        \node[NodeST](13)at(13.00,-10.00)
            {\begin{math}\Product\end{math}};
        \node[NodeST](3)at(3.00,-5.00){\begin{math}\Product\end{math}};
        \node[NodeST](5)at(5.00,0.00){\begin{math}\Product\end{math}};
        \node[NodeST](7)at(7.00,-2.50){\begin{math}\Product\end{math}};
        \node[NodeST](9)at(9.00,-5.00){\begin{math}\Product\end{math}};
        \draw[Edge](0)--(1);
        \draw[Edge](1)--(5);
        \draw[Edge](10)--(11);
        \draw[Edge](11)--(9);
        \draw[Edge](12)--(13);
        \draw[Edge](13)--(11);
        \draw[Edge](14)--(13);
        \draw[Edge](2)--(3);
        \draw[Edge](3)--(1);
        \draw[Edge](4)--(3);
        \draw[Edge](6)--(7);
        \draw[Edge](7)--(5);
        \draw[Edge](8)--(9);
        \draw[Edge](9)--(7);
        \node(r)at(5.00,2){};
        \draw[Edge](r)--(5);
    \end{tikzpicture}
    \enspace \Rew \enspace
    \begin{tikzpicture}[xscale=.21,yscale=.19,Centering]
        \node(0)at(0.00,-4.29){};
        \node(10)at(10.00,-12.86){};
        \node(12)at(12.00,-12.86){};
        \node(14)at(14.00,-10.71){};
        \node(2)at(2.00,-4.29){};
        \node(4)at(4.00,-4.29){};
        \node(6)at(6.00,-6.43){};
        \node(8)at(8.00,-8.57){};
        \node[NodeST](1)at(1.00,-2.14){\begin{math}\Product\end{math}};
        \node[NodeST](11)at(11.00,-10.71)
            {\begin{math}\Product\end{math}};
        \node[NodeST](13)at(13.00,-8.57)
            {\begin{math}\Product\end{math}};
        \node[NodeST](3)at(3.00,0.00){\begin{math}\Product\end{math}};
        \node[NodeST](5)at(5.00,-2.14){\begin{math}\Product\end{math}};
        \node[NodeST](7)at(7.00,-4.29){\begin{math}\Product\end{math}};
        \node[NodeST](9)at(9.00,-6.43){\begin{math}\Product\end{math}};
        \draw[Edge](0)--(1);
        \draw[Edge](1)--(3);
        \draw[Edge](10)--(11);
        \draw[Edge](11)--(13);
        \draw[Edge](12)--(11);
        \draw[Edge](13)--(9);
        \draw[Edge](14)--(13);
        \draw[Edge](2)--(1);
        \draw[Edge](4)--(5);
        \draw[Edge](5)--(3);
        \draw[Edge](6)--(7);
        \draw[Edge](7)--(5);
        \draw[Edge](8)--(9);
        \draw[Edge](9)--(7);
        \node(r)at(3.00,2){};
        \draw[Edge](r)--(3);
    \end{tikzpicture}
\end{equation}
\end{minipage}.


\noindent
We claim that the rewrite relation $\RewContext$ induced by
rewrite rule $\Rew$ satisfying~\eqref{equ:rew_1}, \eqref{equ:rew_2},
\eqref{equ:rew_3}---\eqref{equ:rew_11} is convergent. First, for every
relation $\Tfr \Rew \Tfr'$, we have
$\PrefixWord(\Tfr) > \PrefixWord(\Tfr')$. Therefore, by
Lemma~\ref{lem:prefix_word_termination}, $\RewContext$ is terminating.
Moreover, the greatest degree of a tree appearing in $\Rew$ is~$7$ so
that, from Lemma~\ref{lem:degree_confluence}, to show that $\RewContext$
is convergent, it is enough to prove that each tree of degree at most
$13$ admits exactly one normal form. Equivalently, this amounts to
show that the number of normal forms of trees of arity $n$ is equal
to $\#\CAs{3}(n)$. By computer exploration, we get the same sequence
\begin{equation} \label{equ:dimensions_CAs_3}
    1, 1, 2, 4, 8, 14, 20, 19, 16, 14, 14, 15, 16, 17
\end{equation}
for $\#\CAs{3}(n)$ and for the numbers of normal forms of arity $n$,
when $ 1 \leq n \leq 14$. Hence, we get our following main result.
\medbreak

\begin{Theorem} \label{thm:convergent_rewrite_rule_CAs_3}
    The rewrite rule $\Rew$ satisfying~\eqref{equ:rew_1},
    \eqref{equ:rew_2}, \eqref{equ:rew_3}---\eqref{equ:rew_11} is a
    convergent orientation of the congruence $\CongrCAs{3}$
    of~$\CAs{3}$.
\end{Theorem}
\medbreak

The rewrite rule $\Rew$ has, arity by arity, the cardinalities
\begin{equation}
    0, 0, 0, 1, 1, 2, 3, 4, 0, \dots~.
\end{equation}
We obtain from Theorem~\ref{thm:convergent_rewrite_rule_CAs_3} also
the following consequences.
\medbreak

\begin{Proposition} \label{prop:PBW_basis_CAs_3}
    The set of the trees avoiding as subtrees the ones appearing as
    left members of $\Rew$ is a Poincaré-Birkhoff-Witt basis
    of~$\CAs{3}$.
\end{Proposition}
\medbreak

From Proposition~\ref{prop:PBW_basis_CAs_3}, and by using a result
of~\cite{Gir18} describing a system of equations for the generating
series of syntax trees avoiding some sets of subtrees, we obtain the
following result.
\medbreak

\begin{Proposition} \label{prop:Hilbert_series_CAs_3}
    The Hilbert series of $\CAs{3}$ is
    \begin{equation} \label{equ:Hilbert_series_CAs_3}
        \HilbertSeries_{\CAs{3}}(t) = \frac{t}{(1 - t)^2}
        \left(1 - t + t^2 + t^3 + 2t^4 + 2t^5 - 7t^7 - 2t^8 + t^9 +
        2t^{10} + t^{11}\right).
    \end{equation}
\end{Proposition}
\medbreak

For $n \leq 10$, the dimensions of $\CAs{3}(n)$ are provided by
Sequence~\eqref{equ:dimensions_CAs_3} and for all $n \geq 11$, the
Taylor expansion of~\eqref{equ:Hilbert_series_CAs_3} shows that
$\# \CAs{3}(n) = n + 3$.
\medbreak
