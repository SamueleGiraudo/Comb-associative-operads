%%%%%%%%%%%%%%%%%%%%%%%%%%%%%%%%%%%%%%%%%%%%%%%%%%%%%%%%%%%%%%%%%%%%%%%%
%%%%%%%%%%%%%%%%%%%%%%%%%%%%%%%%%%%%%%%%%%%%%%%%%%%%%%%%%%%%%%%%%%%%%%%%
%%%%%%%%%%%%%%%%%%%%%%%%%%%%%%%%%%%%%%%%%%%%%%%%%%%%%%%%%%%%%%%%%%%%%%%%
\section{Generalizations of the associative operad}
\label{sec:CAs_d}

In this section, we define comb associative operads and we show that the
set of such operads admits a lattice structure, isomorphic to the
lattice division for nonnegative integers. We relate this lattice to the
lattice of linear magmatic quotients. We also provide a finite
convergent presentation of the comb associative operad corresponding
to~$3$.
\medbreak

%%%%%%%%%%%%%%%%%%%%%%%%%%%%%%%%%%%%%%%%%%%%%%%%%%%%%%%%%%%%%%%%%%%%%%%%
%%%%%%%%%%%%%%%%%%%%%%%%%%%%%%%%%%%%%%%%%%%%%%%%%%%%%%%%%%%%%%%%%%%%%%%%
\subsection{Comb associative operads}
Recall first that the \Def{associative operad} $\As$ is the quotient
of $\Mag$ by the smallest operad congruence $\Congr$ satisfying
\begin{equation}
    \begin{tikzpicture}[xscale=.24,yscale=.24,Centering]
        \node(0)at(0.00,-3.33){};
        \node(2)at(2.00,-3.33){};
        \node(4)at(4.00,-1.67){};
        \node[NodeST](1)at(1.00,-1.67)
            {\begin{math}\Product\end{math}};
        \node[NodeST](3)at(3.00,0.00)
            {\begin{math}\Product\end{math}};
        \draw[Edge](0)--(1);
        \draw[Edge](1)--(3);
        \draw[Edge](2)--(1);
        \draw[Edge](4)--(3);
        \node(r)at(3.00,1.5){};
        \draw[Edge](r)--(3);
    \end{tikzpicture}
    \Congr
    \begin{tikzpicture}[xscale=.24,yscale=.24,Centering]
        \node(0)at(0.00,-1.67){};
        \node(2)at(2.00,-3.33){};
        \node(4)at(4.00,-3.33){};
        \node[NodeST](1)at(1.00,0.00)
                {\begin{math}\Product\end{math}};
        \node[NodeST](3)at(3.00,-1.67)
                {\begin{math}\Product\end{math}};
        \draw[Edge](0)--(1);
        \draw[Edge](2)--(3);
        \draw[Edge](3)--(1);
        \draw[Edge](4)--(3);
        \node(r)at(1.00,1.5){};
        \draw[Edge](r)--(1);
    \end{tikzpicture}\,.
\end{equation}
We propose here a generalization of $\Congr$ in order to define
generalizations of $\As$.
\medbreak

Let, for any integer $\gamma \geq 1$, the binary trees $\LComb{\gamma}$
and $\RComb{\gamma}$ be respectively the left and the right combs of
degree $\gamma$. These trees are depicted as
\begin{equation}
    \LComb{\gamma} = \enspace
    \begin{tikzpicture}[xscale=.26,yscale=.3,Centering]
        \node(0)at(0.00,-5.25){};
        \node(2)at(2.00,-5.25){};
        \node(4)at(4.00,-3.50){};
        \node(6)at(6.00,-1.75){};
        \node[NodeST](1)at(1.00,-3.50){\begin{math}\Product\end{math}};
        \node[NodeST](3)at(3.00,-1.75){\begin{math}\Product\end{math}};
        \node[NodeST](5)at(5.00,0.00){\begin{math}\Product\end{math}};
        \draw[Edge](0)--(1);
        \draw[Edge,dotted](1)edge[]node[font=\tiny]{
            \begin{math}\gamma\! -\! 1\end{math}\hspace*{.6cm}}(3);
        \draw[Edge](2)--(1);
        \draw[Edge](3)--(5);
        \draw[Edge](4)--(3);
        \draw[Edge](6)--(5);
        \node(r)at(5.00,1.5){};
        \draw[Edge](r)--(5);
    \end{tikzpicture}
    \qquad \mbox{and} \qquad
    \RComb{\gamma} =
    \begin{tikzpicture}[xscale=.26,yscale=.3,Centering]
        \node(0)at(0.00,-1.75){};
        \node(2)at(2.00,-3.50){};
        \node(4)at(4.00,-5.25){};
        \node(6)at(6.00,-5.25){};
        \node[NodeST](1)at(1.00,0.00){\begin{math}\Product\end{math}};
        \node[NodeST](3)at(3.00,-1.75){\begin{math}\Product\end{math}};
        \node[NodeST](5)at(5.00,-3.50){\begin{math}\Product\end{math}};
        \draw[Edge](0)--(1);
        \draw[Edge](2)--(3);
        \draw[Edge](3)--(1);
        \draw[Edge](4)--(5);
        \draw[Edge,dotted](5)edge[]node[font=\tiny]{
            \hspace*{.6cm}\begin{math}\gamma\! -\! 1\end{math}}(3);
        \draw[Edge](6)--(5);
        \node(r)at(1.00,1.5){};
        \draw[Edge](r)--(1);
    \end{tikzpicture}\,,
\end{equation}
where the values on the dotted edges denote the number of internal
nodes they contain. We shall employ this drawing convention in the
sequel, combined with the convention stipulating that dotted edges
with no value have any number of internal nodes. Let us now define for
any $\gamma \geq 1$ the \Def{$\gamma$-comb associative operad}
$\CAs{\gamma}$ as the quotient operad $\Mag/_{\CongrCAs{\gamma}}$ where
$\CongrCAs{\gamma}$ is the smallest operad congruence of $\Mag$
satisfying
\begin{equation} \label{equ:congruence_CAs_gamma}
    \LComb{\gamma} \enspace \CongrCAs{\gamma} \enspace \RComb{\gamma}.
\end{equation}
Notice that $\CongrCAs{1}$ is trivial so that
$\CAs{1} = \Mag$, and that $\CongrCAs{2}$ is the operad congruence
defining $\As$ so that $\CAs{2} = \As$. Let also
\begin{equation}
    \CAsAll := \left\{\CAs{\gamma} : \gamma\geq 1\right\}
\end{equation}
be the set of all the $\gamma$-comb associative operads.
\medbreak

%%%%%%%%%%%%%%%%%%%%%%%%%%%%%%%%%%%%%%%%%%%%%%%%%%%%%%%%%%%%%%%%%%%%%%%
%%%%%%%%%%%%%%%%%%%%%%%%%%%%%%%%%%%%%%%%%%%%%%%%%%%%%%%%%%%%%%%%%%%%%%%
\subsection{Lattice of comb associative operads}
In order to introduce a lattice structure on $\CAsAll$, we begin by
studying operad morphisms between its elements by mean of intermediate
lemmas.
\medbreak

\begin{Lemma} \label{lem:first_dimensions_CAs}
    For all positive integers $\gamma$ and $n$ such that $\gamma \geq 2$
    and $n \leq \gamma + 1$,
    \begin{equation}
        \# \CAs{\gamma}(n) =
        \begin{cases}
            \Catalan(n)
                & \mbox{ if } n \leq \gamma, \\
            \Catalan(\gamma + 1) - 1
                & \mbox{ otherwise } (n = \gamma + 1).
        \end{cases}
    \end{equation}
\end{Lemma}
\begin{proof}
    Since the equivalence relation $\CongrCAs{\gamma}$ is trivial on the
    binary trees of degrees $d < \gamma$, and since a binary tree of
    degree $d$ has arity $n := d + 1$, one has
    $\# \CAs{\gamma}(n) = \# \Mag(n) = \Catalan(n)$ with
    $n \leq \gamma$. Besides, by definition of $\CongrCAs{\gamma}$, all
    the $\CongrCAs{\gamma}$-equivalence classes of binary trees of
    degree $\gamma$ are trivial, except one consisting in the pair
    $\left\{\LComb{\gamma}, \RComb{\gamma}\right\}$. Therefore, since a
    binary tree of degree $d$ has arity $n := \gamma + 1$,
    \begin{math}
        \# \CAs{\gamma}(n)
        = \# \Mag(\gamma + 1) - 1
        = \Catalan(\gamma + 1) - 1.
    \end{math}
\end{proof}
\medbreak

\begin{Lemma} \label{lem:surjective_morphisms_CAs}
    Let $\gamma$ and $\gamma'$ be two positive integers. If there exists
    an operad morphism $\varphi:\CAs{\gamma'} \to \CAs{\gamma}$, then it
    is surjective and unique.
\end{Lemma}
\begin{proof}
    The operad $\CAs{\gamma'}$ is generated by one binary generator
    $[\Product]_{\CongrCAs{\gamma'}}$, which is the image of the binary
    generator $\Product$ of $\Mag$ in $\CAs{\gamma'}$. Hence, $\varphi$
    is entirely determined by the image
    $\varphi\left([\Product]_{\CongrCAs{\gamma'}}\right)$. Moreover,
    $\varphi([\Product]_{\CongrCAs{\gamma'}})$ has to be of arity $2$ in
    $\CAs{\gamma}$, so that we necessarily have
    \begin{math}
        \varphi\left([\Product]_{\CongrCAs{\gamma'}}\right)
        =
        [\Product]_{\CongrCAs{\gamma}}.
    \end{math}
    Hence, if $\varphi$ exists, it is the unique operad morphism from
    $\CAs{\gamma'}$ to $\CAs{\gamma}$ determined by the previous formula.
    In this case, $[\Product]_{\CongrCAs{\gamma}}$ being in the image of
    $\varphi$, the latter is surjective.
\end{proof}
\medbreak

\begin{Lemma} \label{lem:injective_morphisms_CAs}
    Let $\gamma$ and $\gamma'$ be two positive integers and
    $\varphi:\CAs{\gamma'} \to \CAs{\gamma}$ be an operad morphism.
    Then, $\varphi$ is injective if and only if $\gamma = \gamma'$.
\end{Lemma}
\begin{proof}
    Assume that $\varphi$ is injective. By
    Lemma~\ref{lem:surjective_morphisms_CAs}, $\varphi$ is also
    surjective, so that $\varphi$ is an isomorphism. If
    $\gamma \ne \gamma'$, by Lemma~\ref{lem:first_dimensions_CAs}, there
    is a positive integer $n$ such that
    $\# \CAs{\gamma}(n) \ne \# \CAs{\gamma'}(n)$. This is contradictory
    with the fact that $\CAs{\gamma}$ and $\CAs{\gamma'}$ are
    isomorphic. Hence, $\gamma = \gamma'$.
    \smallbreak

    Conversely, if $\gamma = \gamma'$, the only operad morphism from
    $\CAs{\gamma}$ to itself sends the generator
    $[\Product]_{\CongrCAs{\gamma}}$ to itself. This maps extends as
    an operad morphism into the identity morphism which is of course
    injective.
\end{proof}
\medbreak

\begin{Lemma} \label{lem:morphism_CAs}
    Let $\gamma$ and $\gamma'$ be two positive integers. There exists an
    operad morphism $\varphi:\CAs{\gamma'} \to \CAs{\gamma}$ if and only
    if
    \begin{math}
      \LComb{\gamma'} \CongrCAs{\gamma} \RComb{\gamma'}.
    \end{math}
\end{Lemma}
\begin{proof}
    Assume that $\varphi:\CAs{\gamma'} \to \CAs{\gamma}$ is an operad
    morphism. By Lemma~\ref{lem:surjective_morphisms_CAs}, $\varphi$
    satisfies
    \begin{math}
        \varphi\left(\left[\Tfr\right]_{\CongrCAs{\gamma'}}\right)
        = \left[\Tfr\right]_{\CongrCAs{\gamma}}
    \end{math}
    for any binary tree $\Tfr$.
    Since
    \begin{math}
        \varphi\left(
        \left[\LComb{\gamma'}\right]_{\CongrCAs{\gamma'}}
        \right)
        =
        \varphi\left(
        \left[\RComb{\gamma'}\right]_{\CongrCAs{\gamma'}}
        \right),
    \end{math}
    we have
    \begin{math}
        \left[\LComb{\gamma'}\right]_{\CongrCAs{\gamma}}
        =
        \left[\RComb{\gamma'}\right]_{\CongrCAs{\gamma}},
    \end{math}
    which is equivalent to the fact that
    \begin{math}
        \LComb{\gamma'} \CongrCAs{\gamma} \RComb{\gamma'}.
    \end{math}
    \smallbreak

    Conversely, when
    \begin{math}
        \LComb{\gamma'}\CongrCAs{\gamma}\RComb{\gamma'},
    \end{math}
    let $\varphi:\CAs{\gamma'}(2) \to \CAs{\gamma}(2)$ be the map
    defined by
    \begin{math}
        \varphi\left(
        \left[\Product\right]_{\CongrCAs{\gamma'}}\right)
        :=
        \left[\Product\right]_{\CongrCAs{\gamma}}.
    \end{math}
    Now, since $\CongrCAs{\gamma}$ is coarser than $\CongrCAs{\gamma'}$,
    $\varphi$ extends (in a unique way) into an operad morphism, whence
    the statement of the lemma.
\end{proof}
\medbreak

We define the binary relation $\OrdCAs$ on $\CAsAll$ as follows: we have
$\CAs{\gamma} \OrdCAs \CAs{\gamma'}$ if and only if there exists a
morphism $\varphi:\CAs{\gamma'} \to \CAs{\gamma}$.
\medbreak

\begin{Proposition}\label{prop:poset_CAs}
    The binary relation $\OrdCAs$ is a partial order relation on
    $\CAsAll$.
\end{Proposition}
\begin{proof}
    The binary relation $\OrdCAs$ is reflexive since there exists the
    identity morphism of $\CAs{\gamma}$ for every positive integer
    $\gamma$. Let us now assume that
    $\varphi:\CAs{\gamma''} \to \CAs{\gamma'}$ and
    $\psi:\CAs{\gamma'} \to \CAs{\gamma}$ are two operad morphisms. By
    Lemma~\ref{lem:morphism_CAs}, one has
    $\LComb{\gamma''} \CongrCAs{\gamma'} \RComb{\gamma''}$ and
    $\LComb{\gamma'} \CongrCAs{\gamma} \RComb{\gamma'}$. This implies
    that $\CongrCAs{\gamma}$ is coarser than $\CongrCAs{\gamma'}$ and
    that $\CongrCAs{\gamma'}$ is coarser that $\CongrCAs{\gamma''}$. Now,
    again by Lemma~\ref{lem:morphism_CAs}, $\psi \circ \varphi$ is an
    operad morphism. Hence, $\OrdCAs$ is transitive. Finally, let us
    assume that there exist morphisms
    $\varphi:\CAs{\gamma'}\to\CAs{\gamma}$ and
    $\psi:\CAs{\gamma}\to\CAs{\gamma'}$. In particular,
    $\psi \circ \varphi$ and $\varphi \circ \psi$ are endomorphisms of
    $\CAs{\gamma'}$ and $\CAs{\gamma}$, respectively. From
    Lemma~\ref{lem:surjective_morphisms_CAs}, these two morphisms are
    identity morphisms, so that $\varphi$ and $\psi$ are injective. From
    Lemma~\ref{lem:injective_morphisms_CAs}, $\gamma$ and $\gamma'$ are
    equal, which proves that $\OrdCAs$ is anti-symmetric. Hence,
    $\OrdCAs$ is a partial order.
\end{proof}
\medbreak

In order to show that $\left(\CAsAll, \OrdCAs\right)$ extends into a
lattice, we relate $\left(\CAsAll, \OrdCAs\right)$ with the lattice of
integers $\left(\N,\mid, \gcd, \lcm\right)$, where $\mid$ denotes the
division relation, $\gcd$ denotes the greatest common divisor, and
$\lcm$ the least common multiple operators, respectively.
\medbreak

Recall that $\LRank(\Tfr)$ denotes the left rank of a binary tree
$\Tfr$, defined in Section~\ref{sec:operad_Mag}. Besides, to simplify
the notation, we shall write $\bar{a}$ instead of $a - 1$ for any
integer~$a$.
\medbreak

\begin{Lemma}\label{lem:left_rank_and_CongrCAs}
    Let $\gamma$ be a positive integer and let $\Tfr$ and
    $\Tfr'$ be two binary trees. If $\Tfr \CongrCAs{\gamma} \Tfr'$, then
    \begin{equation}
        \LRank(\Tfr) \pmod{\bar{\gamma}}
        \enspace = \enspace
        \LRank\left(\Tfr'\right) \pmod{\bar{\gamma}}.
    \end{equation}
\end{Lemma}
\begin{proof}
  We consider the rewrite rule $\LComb{\gamma} \Rew \RComb{\gamma}$
    on $\Mag$. First, we show that
    \begin{equation} \label{eq:rewriting_invariant}
        \Tfr \RewContext \Tfr'
        \mbox{ implies }
        \LRank\left(\Tfr'\right) \pmod{\bar{\gamma}} = \LRank\left(\Tfr\right)
        \, \pmod{\bar{\gamma}}.
    \end{equation}
    Indeed, for all $\Tfr \in \Mag$, let
    $\Tfr_1, \dots, \Tfr_{\LRank\left(\Tfr\right)}$ be the binary trees
    such that
    \begin{math}
      \Tfr =
       \LComb{\LRank\left(\Tfr\right)} \circ
      \left[\, \Leaf \, \Tfr_1, \dots,
      \Tfr_{\LRank\left(\Tfr\right)} \right].
    \end{math}
    Thus, if $\Tfr \RewContext \Tfr'$, then one of the following two
    cases occurs:
    \begin{enumerate}[label={(\it\roman*)}]
    \item the rewrite step is applied on one of the trees $\Tfr_i$,
      that is there exists $\Tfr_i'$ such that
            \begin{math}
              \Tfr' = \LComb{\LRank\left(\Tfr\right)} \circ
              \left[\, \Leaf ,\Tfr_1, \dots, \Tfr_i',\dots,
                    \Tfr_{\LRank\left(\Tfr\right)}
              \right] \mbox{, so that }
              \LRank\left(\Tfr'\right) = \LRank\left(\Tfr\right);
            \end{math}
          \item the rewrite step is applied in the left branch beginning
            at the root, that is there exists $i$ such that
            \begin{math}
              \Tfr' = \LComb{\LRank\left(\Tfr\right)-\bar{\gamma}} \circ
              \left[ \Leaf ,\Tfr_1, \dots,
                 \RComb{\bar{\gamma}} \circ
                 [\Tfr_{i},\dots, \Tfr_{i+\bar{\gamma}}]
                 ,\dots, \Tfr_{\LRank\left(\Tfr\right)}
              \right]
              \mbox{, so that }\\
              \LRank\left(\Tfr'\right) =
              \LRank\left(\Tfr\right) - \bar{\gamma} =
              \LRank\left(\Tfr\right) \pmod{\bar{\gamma}}
            \end{math}.
    \end{enumerate}
    This concludes the proof of~\eqref{eq:rewriting_invariant}.
    \smallbreak

    We deduce from~\eqref{eq:rewriting_invariant} that the equivalence
    relation $\simeq$ defined by
    \begin{equation}
        \Tfr \simeq \Tfr' \mbox{ if and only if }
        \LRank\left(\Tfr'\right) \pmod{\bar{\gamma}} =
        \LRank\left(\Tfr\right) \, \pmod{\bar{\gamma}}
    \end{equation}
    contains the relation $\RewContext$, so that it contains its
    transitive symmetric closure. The latter is $\CongrCAs{\gamma}$,
    which concludes the proof of Lemma~\ref{lem:left_rank_and_CongrCAs}.
\end{proof}
\medbreak

\begin{Proposition} \label{prop:division_CAs}
    Let $\gamma$ and $\gamma'$ be two positive integers. Then, there
    exists a morphism $\varphi:\CAs{\gamma'} \to \CAs{\gamma}$ if and
    only if $\bar{\gamma} \mid \bar{\gamma'}$.
\end{Proposition}
\begin{proof}
    From Lemma~\ref{lem:morphism_CAs}, it is enough to show that
    $\LComb{\gamma'} \CongrCAs{\gamma} \RComb{\gamma'}$ if and only if
    $\bar{\gamma} \mid \bar{\gamma'}$. If
    $\LComb{\gamma'}\CongrCAs{\gamma}\RComb{\gamma'}$, $\gamma$ is
    smaller than $\gamma'$ (this is a consequence of the existence of a
    surjective morphism $\varphi:\CAs{\gamma'} \to \CAs{\gamma}$ and
    Lemma~\ref{lem:first_dimensions_CAs}). From
    Lemma~\ref{lem:left_rank_and_CongrCAs}, we deduce that
    \begin{math}
        \LRank\left(\LComb{\gamma'}\right)
        - \LRank\left(\RComb{\gamma'}\right)
        = \bar{\gamma'}
    \end{math}
    is divisible by $\bar{\gamma}$, which shows the direct implication.
    \smallbreak

    Conversely, if
    $\bar{\gamma} \mid \bar{\gamma'}$, the rewrite
    rule $\LComb{\gamma} \Rew \RComb{\gamma}$ induces the
    sequence
    \begin{multline}
        \LComb{\gamma'} \enspace = \enspace
        \begin{tikzpicture}[xscale=.19,yscale=.25,Centering]
            \node(0)at(0.00,-13.12){};
            \node(10)at(10.00,-5.62){};
            \node(12)at(12.00,-3.75){};
            \node(14)at(14.00,-1.88){};
            \node(2)at(2.00,-13.12){};
            \node(4)at(4.00,-11.25){};
            \node(6)at(6.00,-9.38){};
            \node(8)at(8.00,-7.50){};
            \node[NodeST](1)at(1.00,-11.25)
                {\begin{math}\Product\end{math}};
            \node[NodeST](11)at(11.00,-1.88)
                {\begin{math}\Product\end{math}};
            \node[NodeST](13)at(13.00,0.00)
                {\begin{math}\Product\end{math}};
            \node[NodeST](3)at(3.00,-9.38)
                {\begin{math}\Product\end{math}};
            \node[NodeST](5)at(5.00,-7.50)
                {\begin{math}\Product\end{math}};
            \node[NodeST](7)at(7.00,-5.62)
                {\begin{math}\Product\end{math}};
            \node[NodeST](9)at(9.00,-3.75)
                {\begin{math}\Product\end{math}};
            \draw[Edge](0)--(1);
            \draw[Edge,dotted](1)edge[]node[font=\footnotesize]{
                \begin{math}\bar{\gamma}\end{math}\hspace*{.5cm}}(3);
            \draw[Edge](10)--(9);
            \draw[Edge](11)--(13);
            \draw[Edge](12)--(11);
            \draw[Edge](14)--(13);
            \draw[Edge](2)--(1);
            \draw[Edge,dotted](3)--(5);
            \draw[Edge](4)--(3);
            \draw[Edge,dotted](5)edge[]node[font=\footnotesize]{
                \begin{math}\bar{\gamma}\end{math}\hspace*{.5cm}}(7);
            \draw[Edge](6)--(5);
            \draw[Edge](7)--(9);
            \draw[Edge](8)--(7);
            \draw[Edge,dotted](9)edge[]node[font=\footnotesize]{
                \begin{math}\bar{\gamma}\end{math}\hspace*{.5cm}}(11);
            \node(r)at(13.00,1.41){};
            \draw[Edge](r)--(13);
        \end{tikzpicture}
        \enspace \RewContext \enspace
        \begin{tikzpicture}[xscale=.19,yscale=.2,Centering]
            \node(0)at(0.00,-12.50){};
            \node(10)at(10.00,-5.00){};
            \node(12)at(12.00,-7.50){};
            \node(14)at(14.00,-7.50){};
            \node(2)at(2.00,-12.50){};
            \node(4)at(4.00,-10.00){};
            \node(6)at(6.00,-7.50){};
            \node(8)at(8.00,-5.00){};
            \node[NodeST](1)at(1.00,-10.00)
                {\begin{math}\Product\end{math}};
            \node[NodeST](11)at(11.00,-2.50)
                {\begin{math}\Product\end{math}};
            \node[NodeST](13)at(13.00,-5.00)
                {\begin{math}\Product\end{math}};
            \node[NodeST](3)at(3.00,-7.50)
                {\begin{math}\Product\end{math}};
            \node[NodeST](5)at(5.00,-5.00)
                {\begin{math}\Product\end{math}};
            \node[NodeST](7)at(7.00,-2.50)
                {\begin{math}\Product\end{math}};
            \node[NodeST](9)at(9.00,0.00)
                {\begin{math}\Product\end{math}};
            \draw[Edge](0)--(1);
            \draw[Edge,dotted](1)edge[]node[font=\footnotesize]{
                \begin{math}\bar{\gamma}\end{math}\hspace*{.5cm}}(3);
            \draw[Edge](10)--(11);
            \draw[Edge](11)--(9);
            \draw[Edge](12)--(13);
            \draw[Edge,dotted](13)edge[]node[font=\footnotesize]{
                \hspace*{.5cm}\begin{math}\bar{\gamma}\end{math}}(11);
            \draw[Edge](14)--(13);
            \draw[Edge](2)--(1);
            \draw[Edge,dotted](3)--(5);
            \draw[Edge](4)--(3);
            \draw[Edge,dotted](5)edge[]node[font=\footnotesize]{
                \begin{math}\bar{\gamma}\end{math}\hspace*{.5cm}}(7);
            \draw[Edge](6)--(5);
            \draw[Edge](7)--(9);
            \draw[Edge](8)--(7);
            \node(r)at(9.00,1.88){};
            \draw[Edge](r)--(9);
        \end{tikzpicture} \\
        \enspace \RewContext \enspace
        \begin{tikzpicture}[xscale=.19,yscale=.2,Centering]
            \node(0)at(0.00,-7.50){};
            \node(10)at(10.00,-10.00){};
            \node(12)at(12.00,-12.50){};
            \node(14)at(14.00,-12.50){};
            \node(2)at(2.00,-7.50){};
            \node(4)at(4.00,-5.00){};
            \node(6)at(6.00,-5.00){};
            \node(8)at(8.00,-7.50){};
            \node[NodeST](1)at(1.00,-5.00)
                {\begin{math}\Product\end{math}};
            \node[NodeST](11)at(11.00,-7.50)
                {\begin{math}\Product\end{math}};
            \node[NodeST](13)at(13.00,-10.00)
                {\begin{math}\Product\end{math}};
            \node[NodeST](3)at(3.00,-2.50)
                {\begin{math}\Product\end{math}};
            \node[NodeST](5)at(5.00,0.00)
                {\begin{math}\Product\end{math}};
            \node[NodeST](7)at(7.00,-2.50)
                {\begin{math}\Product\end{math}};
            \node[NodeST](9)at(9.00,-5.00)
                {\begin{math}\Product\end{math}};
            \draw[Edge](0)--(1);
            \draw[Edge,dotted](1)edge[]node[font=\footnotesize]{
                \begin{math}\bar{\gamma}\end{math}\hspace*{.5cm}}(3);
            \draw[Edge](10)--(11);
            \draw[Edge](11)--(9);
            \draw[Edge](12)--(13);
            \draw[Edge,dotted](13)edge[]node[font=\footnotesize]{
                \hspace*{.5cm}\begin{math}\bar{\gamma}\end{math}}(11);
            \draw[Edge](14)--(13);
            \draw[Edge](2)--(1);
            \draw[Edge,dotted](3)--(5);
            \draw[Edge](4)--(3);
            \draw[Edge](6)--(7);
            \draw[Edge](7)--(5);
            \draw[Edge](8)--(9);
            \draw[Edge,dotted](9)edge[]node[font=\footnotesize]{
                \hspace*{.5cm}\begin{math}\bar{\gamma}\end{math}}(7);
            \node(r)at(5.00,1.88){};
            \draw[Edge](r)--(5);
        \end{tikzpicture}
        \enspace \RewContextRT \enspace
        \begin{tikzpicture}[xscale=.19,yscale=.25,Centering]
            \node(0)at(0.00,-1.88){};
            \node(10)at(10.00,-11.25){};
            \node(12)at(12.00,-13.12){};
            \node(14)at(14.00,-13.12){};
            \node(2)at(2.00,-3.75){};
            \node(4)at(4.00,-5.62){};
            \node(6)at(6.00,-7.50){};
            \node(8)at(8.00,-9.38){};
            \node[NodeST](1)at(1.00,0.00)
                {\begin{math}\Product\end{math}};
            \node[NodeST](11)at(11.00,-9.38)
                {\begin{math}\Product\end{math}};
            \node[NodeST](13)at(13.00,-11.25)
                {\begin{math}\Product\end{math}};
            \node[NodeST](3)at(3.00,-1.88)
                {\begin{math}\Product\end{math}};
            \node[NodeST](5)at(5.00,-3.75)
                {\begin{math}\Product\end{math}};
            \node[NodeST](7)at(7.00,-5.62)
                {\begin{math}\Product\end{math}};
            \node[NodeST](9)at(9.00,-7.50)
                {\begin{math}\Product\end{math}};
            \draw[Edge](0)--(1);
            \draw[Edge](10)--(11);
            \draw[Edge,dotted](11)--(9);
            \draw[Edge](12)--(13);
            \draw[Edge,dotted](13)edge[]node[font=\footnotesize]{
                \hspace*{.5cm}\begin{math}\bar{\gamma}\end{math}}(11);
            \draw[Edge](14)--(13);
            \draw[Edge](2)--(3);
            \draw[Edge](3)--(1);
            \draw[Edge](4)--(5);
            \draw[Edge,dotted](5)edge[]node[font=\footnotesize]{
                \hspace*{.5cm}\begin{math}\bar{\gamma}\end{math}}(3);
            \draw[Edge](6)--(7);
            \draw[Edge](7)--(5);
            \draw[Edge](8)--(9);
            \draw[Edge,dotted](9)edge[]node[font=\footnotesize]{
                \hspace*{.5cm}\begin{math}\bar{\gamma}\end{math}}(7);
            \node(r)at(1.00,1.41){};
            \draw[Edge](r)--(1);
        \end{tikzpicture}
        \enspace = \enspace
        \RComb{\gamma'}
    \end{multline}
    of rewriting steps, where squared regions denotes left or right
    comb trees of degree $\gamma - 1$. Hence, we have
    $\LComb{\gamma'}\CongrCAs{\gamma}\RComb{\gamma'}$.
\end{proof}
\medbreak

Proposition~\ref{prop:division_CAs} implies the following result.
\medbreak

\begin{Theorem}\label{thm:lattice_CAs}
  The tuple $\LatCAs$ is a lattice, where $\InfCAs$ and $\SupCAs$ are
  defined, for all positive integers $\gamma$ and $\gamma'$, by
    \begin{equation}
      \CAs{\gamma}\InfCAs\CAs{\gamma'}
      :=\CAs{\gcd \left(\bar{\gamma}, \bar{\gamma'}\right)}
    \end{equation}
    and
    \begin{equation}
      \CAs{\gamma}\SupCAs\CAs{\gamma'}
      :=\CAs{\lcm \left(\bar{\gamma}, \bar{\gamma'}\right)}.
    \end{equation}
\end{Theorem}
\medbreak

The lattice $\left(\N, \mid, \gcd, \lcm\right)$ admits $1$ as minimum
element and $0$ as maximum element: any nonnegative integer is
divisible by $1$ and divides $0$. The corresponding property for
$\LatCAs$ is that it admits $\As=\CAs{2}$ as minimum element and $\Mag=
\CAs{1}$ as maximum element: any comb associative operad projects onto
$\As$ and is a quotient of $\Mag$.
\medbreak

We end this section by relating the lattice $\LatQMag$ introduced in
Section~\ref{sec:Magmatic_operads} with the lattice $\LatCAs$. As
explained in Section~\ref{sec:operad_Mag}, a set-theoretic operad can
be embedded into a linear operad, so that the operads $\CAs{\gamma}$ can
be embedded into quotient operads $\KCAs{\gamma}$ of $\KMag$. Formally,
the operad $\KCAs{\gamma}$ is equal to $\KMag/_{I_{\gamma}}$, where
$I_{\gamma}$ is the operadic ideal of $\KMag$ generated by
$\LComb{\gamma}-\RComb{\gamma}$. We obtain a new lattice $\LatKCAs$,
where $\KCAsAll$ is the set of all operads $\KCAs{\gamma}$. In this
linear framework, the condition $\KCAs{\gamma}\OrdCAs\KCAs{\gamma'}$
means that the dimension of the space
$\Hom\left(\KCAs{\gamma'},\KCAs{\gamma}\right)$ is equal to $1$. Hence,
$\LatKCAs$ is related to $\LatQMag$ by the following theorem.
\medbreak

\begin{Theorem} \label{thm:inclusion_lattice_CAs}
    The inclusion
    \begin{math}
        \iota:\left(\KCAsAll,\OrdCAs\right)
        \to
        \left(\QMag,\OrdQMag\right)
    \end{math}
    is nondecreasing. In particular, for all positive integers $\gamma$
    and $\gamma'$, we have
    \begin{equation} \label{equ:comparison_of_lattice_operations}
        \KCAs{\gcd \left(\bar{\gamma}, \bar{\gamma'}\right)}
        \OrdQMag
        \KCAs{\gamma}\InfQMag\KCAs{\gamma'}
    \end{equation}
    and
    \begin{equation}
        \KCAs{\gamma}\SupQMag\KCAs{\gamma'}
        \OrdQMag
        \KCAs{\lcm \left(\bar{\gamma}, \bar{\gamma'}\right)}.
    \end{equation}
\end{Theorem}
\medbreak

Note that $\LatKCAs$ does not embed as a sublattice of $\LatQMag$, that
is $\iota$ is not a lattice morphism. Consider for instance $\gamma = 3$
and $\gamma' = 4$, so that $\KCAs{\gamma} \InfCAs \KCAs{\gamma'}$ is
equal to $\KCAs{2} = \K\As$, whereas
$\KCAs{\gamma} \InfQMag \KCAs{\gamma'}$ is the quotient of $\KMag$ by the
two relations $\LComb{3} - \RComb{3}$ and $\LComb{4} - \RComb{4}$.
\medbreak

%%%%%%%%%%%%%%%%%%%%%%%%%%%%%%%%%%%%%%%%%%%%%%%%%%%%%%%%%%%%%%%%%%%%%%%%
%%%%%%%%%%%%%%%%%%%%%%%%%%%%%%%%%%%%%%%%%%%%%%%%%%%%%%%%%%%%%%%%%%%%%%%%
\subsection{Completion of comb associative operads}
We are now looking for finite convergent presentations of comb
associative operads. By definition, the operad $\CAs{\gamma}$ is the
quotient of $\Mag$ by the operad congruence spanned by the rewrite rule
\begin{equation} \label{equ:rew_1}
    \LComb{\gamma}
    \enspace \Rew \enspace
    \RComb{\gamma}\,.
\end{equation}
This rewrite rule is compatible with the lexicographic order on prefix
words presented at the beginning of Section~\ref{sec:operad_Mag} in the
sense that the prefix word of the left member of~\eqref{equ:rew_1} is
lexicographically greater than the prefix word of the right one.
\medbreak

However, the rewrite relation $\RewContext$ induced by $\Rew$ is not
confluent for $\gamma\geq 3$. Indeed, we have
\begin{equation} \label{equ:branching_pair_CAs_3}
    \LComb{\gamma+1}
    \enspace \RewContext \enspace
    \begin{tikzpicture}[xscale=.22,yscale=.20,Centering]
        \node(0)at(0.00,-4.50){};
        \node(2)at(2.00,-4.50){};
        \node(6)at(6.00,-6.75){};
        \node(8)at(8.00,-6.75){};
        \node[NodeST](1)at(1.00,-2.25){\begin{math}\Product\end{math}};
        \node[NodeST](3)at(3.00,0.00){\begin{math}\Product\end{math}};
        \node[NodeST](5)at(5.30,-1.4){\begin{math}\gamma\end{math}};
        \node[NodeST](7)at(7.00,-4.50){\begin{math}\Product\end{math}};
        \draw[Edge](0)--(1);
        \draw[Edge](1)--(3);
        \draw[Edge](2)--(1);
        \draw[Edge](6)--(7);
        \draw[Edge, dotted](7)--(3);
        \draw[Edge](8)--(7);
        \node(r)at(3.00,1.74){};
        \draw[Edge](r)--(3);
    \end{tikzpicture}
    \qquad \mbox{and} \qquad
    \LComb{\gamma+1}
    \enspace \RewContext \enspace
    \begin{tikzpicture}[xscale=.22,yscale=.22,Centering]
        \node(0)at(0.00,-3.60){};
        \node(4)at(4.00,-7.20){};
        \node(6)at(6.00,-7.20){};
        \node(8)at(8.00,-1.80){};
        \node[NodeST](1)at(1.00,-1.80){\begin{math}\Product\end{math}};
        \node[NodeST](3)at(3.25,-2.90){\begin{math}\gamma\end{math}};
        \node[NodeST](5)at(5.00,-5.40){\begin{math}\Product\end{math}};
        \node[NodeST](7)at(7.00,0.00){\begin{math}\Product\end{math}};
        \draw[Edge](0)--(1);
        \draw[Edge](1)--(7);
        \draw[Edge,dotted](5)--(1);
        \draw[Edge](4)--(5);
        \draw[Edge](6)--(5);
        \draw[Edge](8)--(7);
        \node(r)at(7.00,1.5){};
        \draw[Edge](r)--(7);
    \end{tikzpicture}\,,
\end{equation}
and the two right members of~\eqref{equ:branching_pair_CAs_3} form a
branching pair which is not joinable (since these two trees are
normal forms of $\RewContext$).
\medbreak

In order to transform the rewrite relation induced by~\eqref{equ:rew_1}
into a convergent one, we apply the Buchberger algorithm for
operads~\cite[Section 3.7]{DK10} with respect to the lexicographic order
on prefix words. We first focus on the special case $\gamma = 3$.
\medbreak

%%%%%%%%%%%%%%%%%%%%%%%%%%%%%%%%%%%%%%%%%%%%%%%%%%%%%%%%%%%%%%%%%%%%%%%%
\subsubsection{The \texorpdfstring{$3$}{3}-comb associative operad}
\label{subsubsec:CAs_3}

The Buchberger algorithm applied on binary trees of degrees $4$ to $7$
provides the new rewrite rules \smallbreak
\begin{minipage}{7cm}
\begin{equation} \label{equ:rew_2}
    \begin{tikzpicture}[xscale=.22,yscale=.22,Centering]
        \node(0)at(0.00,-3.60){};
        \node(2)at(2.00,-5.40){};
        \node(4)at(4.00,-7.20){};
        \node(6)at(6.00,-7.20){};
        \node(8)at(8.00,-1.80){};
        \node[NodeST](1)at(1.00,-1.80){\begin{math}\Product\end{math}};
        \node[NodeST](3)at(3.00,-3.60){\begin{math}\Product\end{math}};
        \node[NodeST](5)at(5.00,-5.40){\begin{math}\Product\end{math}};
        \node[NodeST](7)at(7.00,0.00){\begin{math}\Product\end{math}};
        \draw[Edge](0)--(1);
        \draw[Edge](1)--(7);
        \draw[Edge](2)--(3);
        \draw[Edge](3)--(1);
        \draw[Edge](4)--(5);
        \draw[Edge](5)--(3);
        \draw[Edge](6)--(5);
        \draw[Edge](8)--(7);
        \node(r)at(7.00,1.5){};
        \draw[Edge](r)--(7);
    \end{tikzpicture}
    \enspace \Rew \enspace
    \begin{tikzpicture}[xscale=.22,yscale=.20,Centering]
        \node(0)at(0.00,-4.50){};
        \node(2)at(2.00,-4.50){};
        \node(4)at(4.00,-4.50){};
        \node(6)at(6.00,-6.75){};
        \node(8)at(8.00,-6.75){};
        \node[NodeST](1)at(1.00,-2.25){\begin{math}\Product\end{math}};
        \node[NodeST](3)at(3.00,0.00){\begin{math}\Product\end{math}};
        \node[NodeST](5)at(5.00,-2.25){\begin{math}\Product\end{math}};
        \node[NodeST](7)at(7.00,-4.50){\begin{math}\Product\end{math}};
        \draw[Edge](0)--(1);
        \draw[Edge](1)--(3);
        \draw[Edge](2)--(1);
        \draw[Edge](4)--(5);
        \draw[Edge](5)--(3);
        \draw[Edge](6)--(7);
        \draw[Edge](7)--(5);
        \draw[Edge](8)--(7);
        \node(r)at(3.00,1.74){};
        \draw[Edge](r)--(3);
    \end{tikzpicture}
\end{equation}
\end{minipage},
\begin{minipage}{7cm}
\begin{equation} \label{equ:rew_3}
    \begin{tikzpicture}[xscale=.23,yscale=.21,Centering]
        \node(0)at(0.00,-1.83){};
        \node(10)at(10.00,-5.50){};
        \node(2)at(2.00,-3.67){};
        \node(4)at(4.00,-7.33){};
        \node(6)at(6.00,-9.17){};
        \node(8)at(8.00,-9.17){};
        \node[NodeST](1)at(1.00,0.00){\begin{math}\Product\end{math}};
        \node[NodeST](3)at(3.00,-1.83){\begin{math}\Product\end{math}};
        \node[NodeST](5)at(5.00,-5.50){\begin{math}\Product\end{math}};
        \node[NodeST](7)at(7.00,-7.33){\begin{math}\Product\end{math}};
        \node[NodeST](9)at(9.00,-3.67){\begin{math}\Product\end{math}};
        \draw[Edge](0)--(1);
        \draw[Edge](10)--(9);
        \draw[Edge](2)--(3);
        \draw[Edge](3)--(1);
        \draw[Edge](4)--(5);
        \draw[Edge](5)--(9);
        \draw[Edge](6)--(7);
        \draw[Edge](7)--(5);
        \draw[Edge](8)--(7);
        \draw[Edge](9)--(3);
        \node(r)at(1.00,1.75){};
        \draw[Edge](r)--(1);
    \end{tikzpicture}
    \enspace \Rew \enspace
    \begin{tikzpicture}[xscale=.22,yscale=.24,Centering]
        \node(0)at(0.00,-1.83){};
        \node(10)at(10.00,-9.17){};
        \node(2)at(2.00,-3.67){};
        \node(4)at(4.00,-5.50){};
        \node(6)at(6.00,-7.33){};
        \node(8)at(8.00,-9.17){};
        \node[NodeST](1)at(1.00,0.00){\begin{math}\Product\end{math}};
        \node[NodeST](3)at(3.00,-1.83){\begin{math}\Product\end{math}};
        \node[NodeST](5)at(5.00,-3.67){\begin{math}\Product\end{math}};
        \node[NodeST](7)at(7.00,-5.50){\begin{math}\Product\end{math}};
        \node[NodeST](9)at(9.00,-7.33){\begin{math}\Product\end{math}};
        \draw[Edge](0)--(1);
        \draw[Edge](10)--(9);
        \draw[Edge](2)--(3);
        \draw[Edge](3)--(1);
        \draw[Edge](4)--(5);
        \draw[Edge](5)--(3);
        \draw[Edge](6)--(7);
        \draw[Edge](7)--(5);
        \draw[Edge](8)--(9);
        \draw[Edge](9)--(7);
        \node(r)at(1.00,1.5){};
        \draw[Edge](r)--(1);
    \end{tikzpicture}
\end{equation}
\end{minipage},\\
\begin{minipage}{7cm}
\begin{equation}\label{equ:rew_4}
     \begin{tikzpicture}[xscale=.2,yscale=.19,Centering]
        \node(0)at(0.00,-2.20){};
        \node(10)at(10.00,-6.60){};
        \node(2)at(2.00,-6.60){};
        \node(4)at(4.00,-8.80){};
        \node(6)at(6.00,-8.80){};
        \node(8)at(8.00,-6.60){};
        \node[NodeST](1)at(1.00,0.00){\begin{math}\Product\end{math}};
        \node[NodeST](3)at(3.00,-4.40){\begin{math}\Product\end{math}};
        \node[NodeST](5)at(5.00,-6.60){\begin{math}\Product\end{math}};
        \node[NodeST](7)at(7.00,-2.20){\begin{math}\Product\end{math}};
        \node[NodeST](9)at(9.00,-4.40){\begin{math}\Product\end{math}};
        \draw[Edge](0)--(1);
        \draw[Edge](10)--(9);
        \draw[Edge](2)--(3);
        \draw[Edge](3)--(7);
        \draw[Edge](4)--(5);
        \draw[Edge](5)--(3);
        \draw[Edge](6)--(5);
        \draw[Edge](7)--(1);
        \draw[Edge](8)--(9);
        \draw[Edge](9)--(7);
        \node(r)at(1.00,2){};
        \draw[Edge](r)--(1);
    \end{tikzpicture}
    \enspace \Rew \enspace
    \begin{tikzpicture}[xscale=.22,yscale=.24,Centering]
        \node(0)at(0.00,-1.83){};
        \node(10)at(10.00,-9.17){};
        \node(2)at(2.00,-3.67){};
        \node(4)at(4.00,-5.50){};
        \node(6)at(6.00,-7.33){};
        \node(8)at(8.00,-9.17){};
        \node[NodeST](1)at(1.00,0.00){\begin{math}\Product\end{math}};
        \node[NodeST](3)at(3.00,-1.83){\begin{math}\Product\end{math}};
        \node[NodeST](5)at(5.00,-3.67){\begin{math}\Product\end{math}};
        \node[NodeST](7)at(7.00,-5.50){\begin{math}\Product\end{math}};
        \node[NodeST](9)at(9.00,-7.33){\begin{math}\Product\end{math}};
        \draw[Edge](0)--(1);
        \draw[Edge](10)--(9);
        \draw[Edge](2)--(3);
        \draw[Edge](3)--(1);
        \draw[Edge](4)--(5);
        \draw[Edge](5)--(3);
        \draw[Edge](6)--(7);
        \draw[Edge](7)--(5);
        \draw[Edge](8)--(9);
        \draw[Edge](9)--(7);
        \node(r)at(1.00,1.75){};
        \draw[Edge](r)--(1);
    \end{tikzpicture}
\end{equation}
\end{minipage}, 
\begin{minipage}{7cm}
\begin{equation} \label{equ:rew_5}
    \begin{tikzpicture}[xscale=.2,yscale=.2,Centering]
        \node(0)at(0.00,-2.17){};
        \node(10)at(10.00,-10.83){};
        \node(12)at(12.00,-10.83){};
        \node(2)at(2.00,-4.33){};
        \node(4)at(4.00,-8.67){};
        \node(6)at(6.00,-8.67){};
        \node(8)at(8.00,-8.67){};
        \node[NodeST](1)at(1.00,0.00){\begin{math}\Product\end{math}};
        \node[NodeST](11)at(11.00,-8.67)
            {\begin{math}\Product\end{math}};
        \node[NodeST](3)at(3.00,-2.17){\begin{math}\Product\end{math}};
        \node[NodeST](5)at(5.00,-6.50){\begin{math}\Product\end{math}};
        \node[NodeST](7)at(7.00,-4.33){\begin{math}\Product\end{math}};
        \node[NodeST](9)at(9.00,-6.50){\begin{math}\Product\end{math}};
        \draw[Edge](0)--(1);
        \draw[Edge](10)--(11);
        \draw[Edge](11)--(9);
        \draw[Edge](12)--(11);
        \draw[Edge](2)--(3);
        \draw[Edge](3)--(1);
        \draw[Edge](4)--(5);
        \draw[Edge](5)--(7);
        \draw[Edge](6)--(5);
        \draw[Edge](7)--(3);
        \draw[Edge](8)--(9);
        \draw[Edge](9)--(7);
        \node(r)at(1.00,1.75){};
        \draw[Edge](r)--(1);
    \end{tikzpicture}
    \Rew
    \begin{tikzpicture}[xscale=.21,yscale=.22,Centering]
        \node(0)at(0.00,-1.86){};
        \node(10)at(10.00,-11.14){};
        \node(12)at(12.00,-9.29){};
        \node(2)at(2.00,-3.71){};
        \node(4)at(4.00,-5.57){};
        \node(6)at(6.00,-7.43){};
        \node(8)at(8.00,-11.14){};
        \node[NodeST](1)at(1.00,0.00){\begin{math}\Product\end{math}};
        \node[NodeST](11)at(11.00,-7.43)
            {\begin{math}\Product\end{math}};
        \node[NodeST](3)at(3.00,-1.86){\begin{math}\Product\end{math}};
        \node[NodeST](5)at(5.00,-3.71){\begin{math}\Product\end{math}};
        \node[NodeST](7)at(7.00,-5.57){\begin{math}\Product\end{math}};
        \node[NodeST](9)at(9.00,-9.29){\begin{math}\Product\end{math}};
        \draw[Edge](0)--(1);
        \draw[Edge](10)--(9);
        \draw[Edge](11)--(7);
        \draw[Edge](12)--(11);
        \draw[Edge](2)--(3);
        \draw[Edge](3)--(1);
        \draw[Edge](4)--(5);
        \draw[Edge](5)--(3);
        \draw[Edge](6)--(7);
        \draw[Edge](7)--(5);
        \draw[Edge](8)--(9);
        \draw[Edge](9)--(11);
        \node(r)at(1.00,1.75){};
        \draw[Edge](r)--(1);
    \end{tikzpicture}
\end{equation}
\end{minipage},\\
\begin{minipage}{7cm}
\begin{equation} \label{equ:rew_6}
    \begin{tikzpicture}[xscale=.21,yscale=.21,Centering]
        \node(0)at(0.00,-2.17){};
        \node(10)at(10.00,-10.83){};
        \node(12)at(12.00,-10.83){};
        \node(2)at(2.00,-6.50){};
        \node(4)at(4.00,-6.50){};
        \node(6)at(6.00,-6.50){};
        \node(8)at(8.00,-8.67){};
        \node[NodeST](1)at(1.00,0.00){\begin{math}\Product\end{math}};
        \node[NodeST](11)at(11.00,-8.67)
            {\begin{math}\Product\end{math}};
        \node[NodeST](3)at(3.00,-4.33){\begin{math}\Product\end{math}};
        \node[NodeST](5)at(5.00,-2.17){\begin{math}\Product\end{math}};
        \node[NodeST](7)at(7.00,-4.33){\begin{math}\Product\end{math}};
        \node[NodeST](9)at(9.00,-6.50){\begin{math}\Product\end{math}};
        \draw[Edge](0)--(1);
        \draw[Edge](10)--(11);
        \draw[Edge](11)--(9);
        \draw[Edge](12)--(11);
        \draw[Edge](2)--(3);
        \draw[Edge](3)--(5);
        \draw[Edge](4)--(3);
        \draw[Edge](5)--(1);
        \draw[Edge](6)--(7);
        \draw[Edge](7)--(5);
        \draw[Edge](8)--(9);
        \draw[Edge](9)--(7);
        \node(r)at(1.00,1.75){};
        \draw[Edge](r)--(1);
    \end{tikzpicture}
    \Rew
    \begin{tikzpicture}[xscale=.22,yscale=.21,Centering]
        \node(0)at(0.00,-2.17){};
        \node(10)at(10.00,-10.83){};
        \node(12)at(12.00,-10.83){};
        \node(2)at(2.00,-4.33){};
        \node(4)at(4.00,-6.50){};
        \node(6)at(6.00,-10.83){};
        \node(8)at(8.00,-10.83){};
        \node[NodeST](1)at(1.00,0.00){\begin{math}\Product\end{math}};
        \node[NodeST](11)at(11.00,-8.67)
            {\begin{math}\Product\end{math}};
        \node[NodeST](3)at(3.00,-2.17){\begin{math}\Product\end{math}};
        \node[NodeST](5)at(5.00,-4.33){\begin{math}\Product\end{math}};
        \node[NodeST](7)at(7.00,-8.67){\begin{math}\Product\end{math}};
        \node[NodeST](9)at(9.00,-6.50){\begin{math}\Product\end{math}};
        \draw[Edge](0)--(1);
        \draw[Edge](10)--(11);
        \draw[Edge](11)--(9);
        \draw[Edge](12)--(11);
        \draw[Edge](2)--(3);
        \draw[Edge](3)--(1);
        \draw[Edge](4)--(5);
        \draw[Edge](5)--(3);
        \draw[Edge](6)--(7);
        \draw[Edge](7)--(9);
        \draw[Edge](8)--(7);
        \draw[Edge](9)--(5);
        \node(r)at(1.00,1.75){};
        \draw[Edge](r)--(1);
    \end{tikzpicture}\hspace{-8pt}
\end{equation}
\end{minipage},
\begin{minipage}{7cm}
\begin{equation} \label{equ:rew_7}
    \begin{tikzpicture}[xscale=.19,yscale=.17,Centering]
        \node(0)at(0.00,-5.20){};
        \node(10)at(10.00,-7.80){};
        \node(12)at(12.00,-7.80){};
        \node(2)at(2.00,-10.40){};
        \node(4)at(4.00,-10.40){};
        \node(6)at(6.00,-7.80){};
        \node(8)at(8.00,-5.20){};
        \node[NodeST](1)at(1.00,-2.60){\begin{math}\Product\end{math}};
        \node[NodeST](11)at(11.00,-5.20)
            {\begin{math}\Product\end{math}};
        \node[NodeST](3)at(3.00,-7.80){\begin{math}\Product\end{math}};
        \node[NodeST](5)at(5.00,-5.20){\begin{math}\Product\end{math}};
        \node[NodeST](7)at(7.00,0.00){\begin{math}\Product\end{math}};
        \node[NodeST](9)at(9.00,-2.60){\begin{math}\Product\end{math}};
        \draw[Edge](0)--(1);
        \draw[Edge](1)--(7);
        \draw[Edge](10)--(11);
        \draw[Edge](11)--(9);
        \draw[Edge](12)--(11);
        \draw[Edge](2)--(3);
        \draw[Edge](3)--(5);
        \draw[Edge](4)--(3);
        \draw[Edge](5)--(1);
        \draw[Edge](6)--(5);
        \draw[Edge](8)--(9);
        \draw[Edge](9)--(7);
        \node(r)at(7.00,2){};
        \draw[Edge](r)--(7);
    \end{tikzpicture}
    \enspace \Rew \enspace
        \begin{tikzpicture}[xscale=.2,yscale=.19,Centering]
        \node(0)at(0.00,-4.33){};
        \node(10)at(10.00,-10.83){};
        \node(12)at(12.00,-10.83){};
        \node(2)at(2.00,-4.33){};
        \node(4)at(4.00,-4.33){};
        \node(6)at(6.00,-6.50){};
        \node(8)at(8.00,-8.67){};
        \node[NodeST](1)at(1.00,-2.17){\begin{math}\Product\end{math}};
        \node[NodeST](11)at(11.00,-8.67)
            {\begin{math}\Product\end{math}};
        \node[NodeST](3)at(3.00,0.00){\begin{math}\Product\end{math}};
        \node[NodeST](5)at(5.00,-2.17){\begin{math}\Product\end{math}};
        \node[NodeST](7)at(7.00,-4.33){\begin{math}\Product\end{math}};
        \node[NodeST](9)at(9.00,-6.50){\begin{math}\Product\end{math}};
        \draw[Edge](0)--(1);
        \draw[Edge](1)--(3);
        \draw[Edge](10)--(11);
        \draw[Edge](11)--(9);
        \draw[Edge](12)--(11);
        \draw[Edge](2)--(1);
        \draw[Edge](4)--(5);
        \draw[Edge](5)--(3);
        \draw[Edge](6)--(7);
        \draw[Edge](7)--(5);
        \draw[Edge](8)--(9);
        \draw[Edge](9)--(7);
        \node(r)at(3.00,1.75){};
        \draw[Edge](r)--(3);
    \end{tikzpicture}\hspace{-10pt}
\end{equation}
\end{minipage},\\
\begin{minipage}{7cm}
\begin{equation} \label{equ:rew_8}
    \begin{tikzpicture}[xscale=.2,yscale=.19,Centering]
        \node(0)at(0.00,-2.14){};
        \node(10)at(10.00,-12.86){};
        \node(12)at(12.00,-12.86){};
        \node(14)at(14.00,-12.86){};
        \node(2)at(2.00,-4.29){};
        \node(4)at(4.00,-6.43){};
        \node(6)at(6.00,-8.57){};
        \node(8)at(8.00,-12.86){};
        \node[NodeST](1)at(1.00,0.00){\begin{math}\Product\end{math}};
        \node[NodeST](11)at(11.00,-8.57)
            {\begin{math}\Product\end{math}};
        \node[NodeST](13)at(13.00,-10.71)
            {\begin{math}\Product\end{math}};
        \node[NodeST](3)at(3.00,-2.14){\begin{math}\Product\end{math}};
        \node[NodeST](5)at(5.00,-4.29){\begin{math}\Product\end{math}};
        \node[NodeST](7)at(7.00,-6.43){\begin{math}\Product\end{math}};
        \node[NodeST](9)at(9.00,-10.71){\begin{math}\Product\end{math}};
        \draw[Edge](0)--(1);
        \draw[Edge](10)--(9);
        \draw[Edge](11)--(7);
        \draw[Edge](12)--(13);
        \draw[Edge](13)--(11);
        \draw[Edge](14)--(13);
        \draw[Edge](2)--(3);
        \draw[Edge](3)--(1);
        \draw[Edge](4)--(5);
        \draw[Edge](5)--(3);
        \draw[Edge](6)--(7);
        \draw[Edge](7)--(5);
        \draw[Edge](8)--(9);
        \draw[Edge](9)--(11);
        \node(r)at(1.00,2){};
        \draw[Edge](r)--(1);
    \end{tikzpicture}
    \hspace{-10pt}\Rew
    \begin{tikzpicture}[xscale=.22,yscale=.2,Centering]
        \node(0)at(0.00,-1.88){};
        \node(10)at(10.00,-13.12){};
        \node(12)at(12.00,-13.12){};
        \node(14)at(14.00,-11.25){};
        \node(2)at(2.00,-3.75){};
        \node(4)at(4.00,-5.62){};
        \node(6)at(6.00,-7.50){};
        \node(8)at(8.00,-9.38){};
        \node[NodeST](1)at(1.00,0.00){\begin{math}\Product\end{math}};
        \node[NodeST](11)at(11.00,-11.25)
            {\begin{math}\Product\end{math}};
        \node[NodeST](13)at(13.00,-9.38)
            {\begin{math}\Product\end{math}};
        \node[NodeST](3)at(3.00,-1.88){\begin{math}\Product\end{math}};
        \node[NodeST](5)at(5.00,-3.75){\begin{math}\Product\end{math}};
        \node[NodeST](7)at(7.00,-5.62){\begin{math}\Product\end{math}};
        \node[NodeST](9)at(9.00,-7.50){\begin{math}\Product\end{math}};
        \draw[Edge](0)--(1);
        \draw[Edge](10)--(11);
        \draw[Edge](11)--(13);
        \draw[Edge](12)--(11);
        \draw[Edge](13)--(9);
        \draw[Edge](14)--(13);
        \draw[Edge](2)--(3);
        \draw[Edge](3)--(1);
        \draw[Edge](4)--(5);
        \draw[Edge](5)--(3);
        \draw[Edge](6)--(7);
        \draw[Edge](7)--(5);
        \draw[Edge](8)--(9);
        \draw[Edge](9)--(7);
        \node(r)at(1.00,2){};
        \draw[Edge](r)--(1);
    \end{tikzpicture}\hspace{-20pt}
\end{equation}
\end{minipage},
\begin{minipage}{7cm}
\begin{equation} \label{equ:rew_9}
    \begin{tikzpicture}[xscale=0.18,yscale=.2,Centering]
        \node(0)at(0.00,-2.14){};
        \node(10)at(10.00,-12.86){};
        \node(12)at(12.00,-12.86){};
        \node(14)at(14.00,-10.71){};
        \node(2)at(2.00,-4.29){};
        \node(4)at(4.00,-6.43){};
        \node(6)at(6.00,-10.71){};
        \node(8)at(8.00,-10.71){};
        \node[NodeST](1)at(1.00,0.00){\begin{math}\Product\end{math}};
        \node[NodeST](11)at(11.00,-10.71)
            {\begin{math}\Product\end{math}};
        \node[NodeST](13)at(13.00,-8.57)
            {\begin{math}\Product\end{math}};
        \node[NodeST](3)at(3.00,-2.14){\begin{math}\Product\end{math}};
        \node[NodeST](5)at(5.00,-4.29){\begin{math}\Product\end{math}};
        \node[NodeST](7)at(7.00,-8.57){\begin{math}\Product\end{math}};
        \node[NodeST](9)at(9.00,-6.43){\begin{math}\Product\end{math}};
        \draw[Edge](0)--(1);
        \draw[Edge](10)--(11);
        \draw[Edge](11)--(13);
        \draw[Edge](12)--(11);
        \draw[Edge](13)--(9);
        \draw[Edge](14)--(13);
        \draw[Edge](2)--(3);
        \draw[Edge](3)--(1);
        \draw[Edge](4)--(5);
        \draw[Edge](5)--(3);
        \draw[Edge](6)--(7);
        \draw[Edge](7)--(9);
        \draw[Edge](8)--(7);
        \draw[Edge](9)--(5);
        \node(r)at(1.00,2){};
        \draw[Edge](r)--(1);
    \end{tikzpicture}
    \hspace{-5pt} \Rew \hspace{-5pt}
    \begin{tikzpicture}[xscale=.22,yscale=.22,Centering]
        \node(0)at(0.00,-1.88){};
        \node(10)at(10.00,-11.25){};
        \node(12)at(12.00,-13.12){};
        \node(14)at(14.00,-13.12){};
        \node(2)at(2.00,-3.75){};
        \node(4)at(4.00,-5.62){};
        \node(6)at(6.00,-7.50){};
        \node(8)at(8.00,-9.38){};
        \node[NodeST](1)at(1.00,0.00){\begin{math}\Product\end{math}};
        \node[NodeST](11)at(11.00,-9.38)
            {\begin{math}\Product\end{math}};
        \node[NodeST](13)at(13.00,-11.25)
            {\begin{math}\Product\end{math}};
        \node[NodeST](3)at(3.00,-1.88){\begin{math}\Product\end{math}};
        \node[NodeST](5)at(5.00,-3.75){\begin{math}\Product\end{math}};
        \node[NodeST](7)at(7.00,-5.62){\begin{math}\Product\end{math}};
        \node[NodeST](9)at(9.00,-7.50){\begin{math}\Product\end{math}};
        \draw[Edge](0)--(1);
        \draw[Edge](10)--(11);
        \draw[Edge](11)--(9);
        \draw[Edge](12)--(13);
        \draw[Edge](13)--(11);
        \draw[Edge](14)--(13);
        \draw[Edge](2)--(3);
        \draw[Edge](3)--(1);
        \draw[Edge](4)--(5);
        \draw[Edge](5)--(3);
        \draw[Edge](6)--(7);
        \draw[Edge](7)--(5);
        \draw[Edge](8)--(9);
        \draw[Edge](9)--(7);
        \node(r)at(1.00,1.5){};
        \draw[Edge](r)--(1);
    \end{tikzpicture}\hspace{-25pt}
\end{equation}
\end{minipage},\\
\begin{minipage}{7cm}
\begin{equation} \label{equ:rew_10}
    \hspace{-2pt}\begin{tikzpicture}[xscale=.2,yscale=.17,Centering]
        \node(0)at(0.00,-5.00){};
        \node(10)at(10.00,-12.50){};
        \node(12)at(12.00,-12.50){};
        \node(14)at(14.00,-12.50){};
        \node(2)at(2.00,-5.00){};
        \node(4)at(4.00,-5.00){};
        \node(6)at(6.00,-7.50){};
        \node(8)at(8.00,-12.50){};
        \node[NodeST](1)at(1.00,-2.50){\begin{math}\Product\end{math}};
        \node[NodeST](11)at(11.00,-7.50)
            {\begin{math}\Product\end{math}};
        \node[NodeST](13)at(13.00,-10.00)
            {\begin{math}\Product\end{math}};
        \node[NodeST](3)at(3.00,0.00){\begin{math}\Product\end{math}};
        \node[NodeST](5)at(5.00,-2.50){\begin{math}\Product\end{math}};
        \node[NodeST](7)at(7.00,-5.00){\begin{math}\Product\end{math}};
        \node[NodeST](9)at(9.00,-10.00){\begin{math}\Product\end{math}};
        \draw[Edge](0)--(1);
        \draw[Edge](1)--(3);
        \draw[Edge](10)--(9);
        \draw[Edge](11)--(7);
        \draw[Edge](12)--(13);
        \draw[Edge](13)--(11);
        \draw[Edge](14)--(13);
        \draw[Edge](2)--(1);
        \draw[Edge](4)--(5);
        \draw[Edge](5)--(3);
        \draw[Edge](6)--(7);
        \draw[Edge](7)--(5);
        \draw[Edge](8)--(9);
        \draw[Edge](9)--(11);
        \node(r)at(3.00,2){};
        \draw[Edge](r)--(3);
    \end{tikzpicture}
    \hspace{-10pt} \Rew \hspace{5pt}
    \begin{tikzpicture}[xscale=.21,yscale=.2,Centering]
        \node(0)at(0.00,-4.29){};
        \node(10)at(10.00,-12.86){};
        \node(12)at(12.00,-12.86){};
        \node(14)at(14.00,-10.71){};
        \node(2)at(2.00,-4.29){};
        \node(4)at(4.00,-4.29){};
        \node(6)at(6.00,-6.43){};
        \node(8)at(8.00,-8.57){};
        \node[NodeST](1)at(1.00,-2.14){\begin{math}\Product\end{math}};
        \node[NodeST](11)at(11.00,-10.71)
            {\begin{math}\Product\end{math}};
        \node[NodeST](13)at(13.00,-8.57)
            {\begin{math}\Product\end{math}};
        \node[NodeST](3)at(3.00,0.00){\begin{math}\Product\end{math}};
        \node[NodeST](5)at(5.00,-2.14){\begin{math}\Product\end{math}};
        \node[NodeST](7)at(7.00,-4.29){\begin{math}\Product\end{math}};
        \node[NodeST](9)at(9.00,-6.43){\begin{math}\Product\end{math}};
        \draw[Edge](0)--(1);
        \draw[Edge](1)--(3);
        \draw[Edge](10)--(11);
        \draw[Edge](11)--(13);
        \draw[Edge](12)--(11);
        \draw[Edge](13)--(9);
        \draw[Edge](14)--(13);
        \draw[Edge](2)--(1);
        \draw[Edge](4)--(5);
        \draw[Edge](5)--(3);
        \draw[Edge](6)--(7);
        \draw[Edge](7)--(5);
        \draw[Edge](8)--(9);
        \draw[Edge](9)--(7);
        \node(r)at(3.00,2){};
        \draw[Edge](r)--(3);
    \end{tikzpicture} \hspace{-25pt}
\end{equation}
\end{minipage},
\begin{minipage}{7cm}
\begin{equation} \label{equ:rew_11}
    \begin{tikzpicture}[xscale=.22,yscale=.18,Centering]
        \node(0)at(0.00,-5.00){};
        \node(10)at(10.00,-10.00){};
        \node(12)at(12.00,-12.50){};
        \node(14)at(14.00,-12.50){};
        \node(2)at(2.00,-7.50){};
        \node(4)at(4.00,-7.50){};
        \node(6)at(6.00,-5.00){};
        \node(8)at(8.00,-7.50){};
        \node[NodeST](1)at(1.00,-2.50){\begin{math}\Product\end{math}};
        \node[NodeST](11)at(11.00,-7.50)
            {\begin{math}\Product\end{math}};
        \node[NodeST](13)at(13.00,-10.00)
            {\begin{math}\Product\end{math}};
        \node[NodeST](3)at(3.00,-5.00){\begin{math}\Product\end{math}};
        \node[NodeST](5)at(5.00,0.00){\begin{math}\Product\end{math}};
        \node[NodeST](7)at(7.00,-2.50){\begin{math}\Product\end{math}};
        \node[NodeST](9)at(9.00,-5.00){\begin{math}\Product\end{math}};
        \draw[Edge](0)--(1);
        \draw[Edge](1)--(5);
        \draw[Edge](10)--(11);
        \draw[Edge](11)--(9);
        \draw[Edge](12)--(13);
        \draw[Edge](13)--(11);
        \draw[Edge](14)--(13);
        \draw[Edge](2)--(3);
        \draw[Edge](3)--(1);
        \draw[Edge](4)--(3);
        \draw[Edge](6)--(7);
        \draw[Edge](7)--(5);
        \draw[Edge](8)--(9);
        \draw[Edge](9)--(7);
        \node(r)at(5.00,2){};
        \draw[Edge](r)--(5);
    \end{tikzpicture}
    \enspace \Rew \enspace
    \begin{tikzpicture}[xscale=.21,yscale=.19,Centering]
        \node(0)at(0.00,-4.29){};
        \node(10)at(10.00,-12.86){};
        \node(12)at(12.00,-12.86){};
        \node(14)at(14.00,-10.71){};
        \node(2)at(2.00,-4.29){};
        \node(4)at(4.00,-4.29){};
        \node(6)at(6.00,-6.43){};
        \node(8)at(8.00,-8.57){};
        \node[NodeST](1)at(1.00,-2.14){\begin{math}\Product\end{math}};
        \node[NodeST](11)at(11.00,-10.71)
            {\begin{math}\Product\end{math}};
        \node[NodeST](13)at(13.00,-8.57)
            {\begin{math}\Product\end{math}};
        \node[NodeST](3)at(3.00,0.00){\begin{math}\Product\end{math}};
        \node[NodeST](5)at(5.00,-2.14){\begin{math}\Product\end{math}};
        \node[NodeST](7)at(7.00,-4.29){\begin{math}\Product\end{math}};
        \node[NodeST](9)at(9.00,-6.43){\begin{math}\Product\end{math}};
        \draw[Edge](0)--(1);
        \draw[Edge](1)--(3);
        \draw[Edge](10)--(11);
        \draw[Edge](11)--(13);
        \draw[Edge](12)--(11);
        \draw[Edge](13)--(9);
        \draw[Edge](14)--(13);
        \draw[Edge](2)--(1);
        \draw[Edge](4)--(5);
        \draw[Edge](5)--(3);
        \draw[Edge](6)--(7);
        \draw[Edge](7)--(5);
        \draw[Edge](8)--(9);
        \draw[Edge](9)--(7);
        \node(r)at(3.00,2){};
        \draw[Edge](r)--(3);
    \end{tikzpicture}
\end{equation}
\end{minipage}.

\medbreak

\begin{Theorem} \label{thm:convergent_rewrite_rule_CAs_3}
    The set $\Rew$ of rewrite rules containing~\eqref{equ:rew_1},
    and \eqref{equ:rew_2}---\eqref{equ:rew_11} is a finite
    convergent presentation of~$\CAs{3}$.
\end{Theorem}
\begin{proof}
    Let us show that the rewrite relation $\RewContext$ induced by
    $\Rew$ is convergent. First, for every relation $\Tfr \Rew \Tfr'$,
    we have $\Tfr > \Tfr'$. Therefore, by
    Lemma~\ref{lem:prefix_word_termination}, $\RewContext$ is
    terminating. Moreover, the greatest degree of a tree appearing in
    $\Rew$ is~$7$ so that, from Lemma~\ref{lem:degree_confluence}, to
    show that $\RewContext$ is convergent, it is enough to prove that
    each tree of degree at most $13$ admits exactly one normal form.
    Equivalently, this amounts to show that the number of normal forms
    of trees of arity $n\leq 14$ is equal to $\#\CAs{3}(n)$. By computer
    exploration, we get the same sequence
    \begin{equation} \label{equ:dimensions_CAs_3}
        1, 1, 2, 4, 8, 14, 20, 19, 16, 14, 14, 15, 16, 17
    \end{equation}
    for $\#\CAs{3}(n)$ and for the numbers of normal forms of arity $n$,
    when $ 1 \leq n \leq 14$, which proves
    Theorem~\ref{thm:convergent_rewrite_rule_CAs_3}.
\end{proof}
\medbreak

The rewrite rule $\Rew$ has, arity by arity, the cardinalities
\begin{equation}
    0, 0, 0, 1, 1, 2, 3, 4, 0, 0, \dots~.
\end{equation}
We also obtain from Theorem~\ref{thm:convergent_rewrite_rule_CAs_3}
the following consequences.
\medbreak

\begin{Proposition} \label{prop:PBW_basis_CAs_3}
    The set of the trees avoiding as subtrees the ones appearing as
    left members of $\Rew$ is a Poincaré-Birkhoff-Witt basis
    of~$\CAs{3}$.
\end{Proposition}
\begin{proof}
    By definition of PBW bases and
    Theorem~\ref{thm:convergent_rewrite_rule_CAs_3}, the set
    $\NormalForms_{\RewContext}$ is a PBW basis of $\CAs{3}$ where
    $\RewContext$ is the rewrite relation induced by $\Rew$. Now, by
    Lemma~\ref{lem:normal_forms_avoiding}, $\NormalForms_{\RewContext}$
    can be described as the set of the trees avoiding the left members
    of~$\Rew$, whence the statement.
\end{proof}
\medbreak

\begin{Proposition} \label{prop:Hilbert_series_CAs_3}
    The Hilbert series of $\CAs{3}$ is
    \begin{equation} \label{equ:Hilbert_series_CAs_3}
        \HilbertSeries_{\CAs{3}}(t) = \frac{t}{(1 - t)^2}
        \left(1 - t + t^2 + t^3 + 2t^4 + 2t^5 - 7t^7 - 2t^8 + t^9 +
        2t^{10} + t^{11}\right).
    \end{equation}
\end{Proposition}
\begin{proof}
    From Proposition~\ref{prop:PBW_basis_CAs_3}, for any $n \geq 1$, the
    dimension of $\CAs{3}(n)$ is the number of trees that avoid as
    subtrees the left members of $\Rew$. Now, by using a result
    of~\cite{Gir18} (see also~\cite{Row10,KP15}) providing a system of
    equations for the generating series of the trees avoiding some sets
    of subtrees, we obtain Expression~\eqref{equ:Hilbert_series_CAs_3}
    for the considered family.
\end{proof}
\medbreak

For $n \leq 10$, the dimensions of $\CAs{3}(n)$ are provided by
Sequence~\eqref{equ:dimensions_CAs_3} and for all $n \geq 11$, the
Taylor expansion of~\eqref{equ:Hilbert_series_CAs_3} shows that
\begin{equation}
  \# \CAs{3}(n) = n + 3.
  \end{equation}
\medbreak

\Todo{S: réalisation combinatoire de $\CAs{3}$ à donner ici.}

%%%%%%%%%%%%%%%%%%%%%%%%%%%%%%%%%%%%%%%%%%%%%%%%%%%%%%%%%%%%%%%%%%%%%%%%
\subsubsection{Higher comb associative operads}
\label{sec:higher_comb_associative_operads}

We run the same algorithm for $\CAs{\gamma}$ when $\gamma \geq 4$.
We obtain the following results.
\medbreak

\Todo{S:remettre la table en forme.}
\begin{center}
  \begin{math} \label{arr:completions_CAs}
    \begin{array}{|c|l|}
      \hline
      \gamma & \mbox{The sequence of first cardinalities of }
       \CAs{\gamma} \mbox{completion}\\ \hline
      1 & 0, 1, 0, 0, \dots \\ \hline
      2 & 0, 0, 1, 0, 0, \dots \\ \hline
      3 & 0, 0, 0, 1, 1, 2, 3, 4, 0, 0, \dots \\ \hline
      4 & 0, 0, 0, 0, 1, 1, 0, 3, 4, 5, 18, 22, 11, 12, 15, 19, 25, 36,
       44, 52, 68, 79, 93, 105, 106, 109 \\ \hline
      5 & 0, 0, 0, 0, 0, 1, 1, 0, 0, 4, 5, 8, 18, 31, 36, 48, 73, 111,
       172, 272, 455, 783 \\ \hline
      6 & 0, 0, 0, 0, 0, 0, 1, 1, 0, 0, 0, 5, 6, 11, 23, 30, 48, 73,
      117, 204, 348, 589, 1004 \\ \hline
      7 & 0, 0, 0, 0, 0, 0, 0, 1, 1, 0, 0, 0, 0, 6, 7, 16, 24, 32, 49,
      88, 150, 261, 475, 854\\ \hline
      8 & 0, 0, 0, 0, 0, 0, 0, 0, 1, 1, 0, 0, 0, 0, 0, 7, 8, 21, 29,
      34, 53, 93, 172, 311, 565\\ \hline
      9 & 0, 0, 0, 0, 0, 0, 0, 0, 0, 1, 1, 0, 0, 0, 0, 0, 0, 8, 9, 28,
      30, 36, 57, 101, 185, 348, 648\\ \hline
    \end{array}
  \end{math}
\end{center}
\medbreak

From these computer explorations, we conjecture that this algorithm
does not provide a finite convergent presentation of $\CAs{\gamma}$,
when $\gamma \geq 4$. We point out that for $\CAs{4}$ new
rewrite rules still appear at degree $41$. Moreover, the total number of
rewrite rules at degree $41$ is $3149$.
\medbreak

However, the completion algorithm we used depends on the chosen order on
the trees. In order to find out if the completion algorithm leads to a
finite convergent presentation using a different order, we run the
following backtracking algorithm. For every branching pair
$(\Tfr_1, \Tfr_2)$ which is not joinable, we recursively try to find a
completion of $\Rew$ by adding either the rule
$\Tfr_1 \Rew  \Tfr_2$ or $\Tfr_2 \Rew  \Tfr_1$.
If at any moment the rewrite relation $\RewContext$ induced by $\Rew$
loops (that is $\RewContext$ is not antisymmetric), we simply reject it.
\medbreak

We did not find a finite presentation for $\CAs{4}$, $\CAs{5}$, and
$\CAs{6}$ until arity $12$. We conjecture that there is no finite
convergent presentation of $\CAs{\gamma}$ with one generator when
$\gamma \geq 4$. We discuss the cases with more than one generator in
the perspectives.
\medbreak

\Todo{S: la numérotation de la table n'apparait pas (mais je m'occuperai
de la faire marcher quand je ferai la remise en forme.}
Thanks to the partial completions presented in table
\eqref{arr:completions_CAs}, we can compute the following first
dimensions of $\CAs{\gamma}$.
\begin{center}
  \begin{math} \label{arr:dimensions_CAs}
    \begin{array}{|c|l|}
      \hline
      \gamma & \mbox{The sequence of first dimensions of }
       \CAs{\gamma}\\ \hline
      1 & 1, 1, 2, 5, 14, 42, 132, 429, 1430, 4862, \dots \\ \hline
      2 & 1, 1, 1, \dots \\ \hline
      3 & 1, 1, 2, 4, 8, 14, 20, 19, 16, 14,
          14, 15, 16, 17, \dots \\ \hline
      4 & 1, 1, 2, 5, 13, 35, 96, 264, 724, 1973, 5335, 14390, 38872,
          105141, 284929, 774254, 2111088,\\
        & 5778420, 15882100, 43837388, 121507230 \\ \hline
      5 & 1, 1, 2, 5, 14, 41, 124, 384, 1210, 3861, 12440, 40392, 131997,
          433782, 1432696, 4752857, \\
        & 15829261, 52905635, 177395246 \\ \hline
      6 & 1, 1, 2, 5, 14, 42, 131, 420, 1375, 4576, 15431, 52598, 180895,
          626862, 2186504, 7670138, \\
        & 27041833, 95764569 \\ \hline
      7 & 1, 1, 2, 5, 14, 42, 132, 428, 1420, 4796, 16432, 56966, 199444,
          704140, 2503914, 8959699, \\
        & 32236657, 116551168\\ \hline
      8 & 1, 1, 2, 5, 14, 42, 132, 429, 1429, 4851, 16718, 58331, 205632,
          731272, 2620176, 9449688, \\
        & 34276116, 124958386\\ \hline
      9 & 1, 1, 2, 5, 14, 42, 132, 429, 1430, 4861, 16784, 58695, 207452,
          739840, 2658936, 9620232, \\
        & 35011566, 128082515\\ \hline
    \end{array}
  \end{math}
\end{center}
\medbreak

\Todo{S:remettre la table en forme.}
