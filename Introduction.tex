%%%%%%%%%%%%%%%%%%%%%%%%%%%%%%%%%%%%%%%%%%%%%%%%%%%%%%%%%%%%%%%%%%%%%%%%
%%%%%%%%%%%%%%%%%%%%%%%%%%%%%%%%%%%%%%%%%%%%%%%%%%%%%%%%%%%%%%%%%%%%%%%%
%%%%%%%%%%%%%%%%%%%%%%%%%%%%%%%%%%%%%%%%%%%%%%%%%%%%%%%%%%%%%%%%%%%%%%%%
\section*{Introduction}
Associative algebras are spaces endowed with a binary product $\Product$
satisfying among others the associativity law
\begin{equation}
    \left(x_1 \Product x_2\right) \Product x_3
    =
    x_1 \Product \left(x_2 \Product x_3\right).
\end{equation}
It is well-known that the associative algebras are representations of
the associative (nonsymmetric) operad $\As$. This operad can be seen as
the quotient of the magmatic operad $\Mag$ (the free operad of binary
trees on the binary generator~$\Product$) by the operad
congruence~$\Congr$ satisfying
\begin{equation} \label{equ:congruence_as}
    \begin{tikzpicture}[xscale=.24,yscale=.24,Centering]
        \node(0)at(0.00,-3.33){};
        \node(2)at(2.00,-3.33){};
        \node(4)at(4.00,-1.67){};
        \node[NodeST](1)at(1.00,-1.67)
            {\begin{math}\Product\end{math}};
        \node[NodeST](3)at(3.00,0.00)
            {\begin{math}\Product\end{math}};
        \draw[Edge](0)--(1);
        \draw[Edge](1)--(3);
        \draw[Edge](2)--(1);
        \draw[Edge](4)--(3);
        \node(r)at(3.00,1.5){};
        \draw[Edge](r)--(3);
    \end{tikzpicture}
    \Congr
    \begin{tikzpicture}[xscale=.24,yscale=.24,Centering]
        \node(0)at(0.00,-1.67){};
        \node(2)at(2.00,-3.33){};
        \node(4)at(4.00,-3.33){};
        \node[NodeST](1)at(1.00,0.00)
                {\begin{math}\Product\end{math}};
        \node[NodeST](3)at(3.00,-1.67)
                {\begin{math}\Product\end{math}};
        \draw[Edge](0)--(1);
        \draw[Edge](2)--(3);
        \draw[Edge](3)--(1);
        \draw[Edge](4)--(3);
        \node(r)at(1.00,1.5){};
        \draw[Edge](r)--(1);
    \end{tikzpicture}\,.
\end{equation}
These two binary trees are the syntax trees of the expressions appearing
in the above associativity law.
\medbreak

In a more combinatorial context and regardless of the theory of operads,
the Tamari order is a partial order on the set of the binary trees
having a fixed number of internal nodes $\gamma$. This order is
generated by the covering relation consisting in rewriting a tree $\Tfr$
into a tree $\Tfr'$ by replacing a subtree of $\Tfr$ of the form of the
left member of~\eqref{equ:congruence_as} into a tree of the form of the
right member of~\eqref{equ:congruence_as}. This transformation is known
in a computer science context as the right rotation
operation~\cite{Knu98} and intervenes in algorithms involving binary
search trees~\cite{AVL62}. The partial order hence generated by the
right rotation operation is known as the Tamari order~\cite{Tam62} and
has a lot of combinatorial and algebraic properties (see for
instance~\cite{HT72,Cha06}).
\medbreak

A first connection between the associative operad and the Tamari order
is based upon the fact that the orientation of~\eqref{equ:congruence_as}
from left to right provides a convergent orientation (a terminating and
confluent rewrite relation) of the congruence $\Congr$. The normal
forms of the rewrite relation induced by the rewrite rule obtained by
orienting~\eqref{equ:congruence_as} from left to right are right comb
binary trees and are hence in one-to-one correspondence with the
elements of~$\As$. Following the ideas developed by Anick for
associative algebras~\cite{Ani86}, the description of an operad by mean
of normal forms provides homological information for this operad. One of
the fundamental homological properties for operads is
Koszulness~\cite{GK94}, generalizing Koszul associative
algebras~\cite{Pri70}: the convergent orientation of $\As$ proves that
it is a Koszul operad~\cite{LV12}.
\medbreak

This work is intended to study and collect the possible links between
the Tamari order and some quotients of the operad $\Mag$. In the long
run, we would like to study quotients $\Mag/_{\Congr}$ of $\Mag$ where
$\Congr$ is an operad congruence generated by equivalence classes of
trees of a fixed degree (that is, a fixed number of internal nodes). In
particular, we would like to know if the fact that $\Congr$ is generated
by equivalence classes of trees forming intervals of the Tamari order
implies algebraic properties for $\Mag/_{\Congr}$ (like the description
of orientations of its space of relations, nice bases, and Hilbert
series).
\medbreak

To explore this vast research area, we select to follow in this paper
the following directions. First, we consider the very general set of the
quotients of $\Mag$ seen as a operad in the category of vector spaces.
We show that these operads form a lattice, wherein its partial order
relation is defined from the existence of operad morphisms
(Theorem~\ref{thm:lattice_structure_of_QMag}). We also provide a
Grassmann formula analog relating the Hilbert series of the operads of
the lattice together with their infima and suprema
(Theorem~\ref{thm:Grassmann_formula_for_Hilbert_series_of_QMag}).
Besides, we focus on a special kind of quotients of $\Mag$, denoted by
$\CAs{\gamma}$, defined by equating the left and right comb binary trees
of a fixed degree $\gamma \geq 1$. These operads are called comb
associative operads. For instance, $\CAs{3}$ is the operad describing
the category of the algebras equipped with a binary product $\Product$
subjected to the relation
\begin{equation}
    \left(\left(x_1 \Product x_2\right) \Product x_3\right) \Product x_4
    =
    x_1 \Product \left(x_2 \Product \left(x_3 \Product x_4\right)\right).
\end{equation}
This is a kind of associativity law in higher degree $\gamma = 3$ so
that $\CAs{3}$ can be seen as a generalization of $\As$ (which is equal
to $\CAs{2}$). We first provide general results about the operads
$\CAs{\gamma}$. In particular, we show that these operads form a lattice
contained in the lattice of quotients of $\Mag$ aforementioned
(Theorems~\ref{thm:lattice_CAs} and~\ref{thm:inclusion_lattice_CAs}). We
focus in particular on the study of $\CAs{3}$. Observe that the
congruence of this operad is generated by an equivalence class of trees
which is not an interval for the Tamari order. As preliminary computer
experiments show, $\CAs{3}$ has oscillating first dimensions
(see~\eqref{equ:dimensions_CAs_3}), what is rather unusual among all
known operads. We provide a convergent orientation of the space of
relations of $\CAs{3}$ (Theorem~\ref{thm:convergent_rewrite_rule_CAs_3}),
a description of a basis of the operad, and prove that its Hilbert series
is rational.
%
\Todo{S: et parler de la réalisation si Christophe l'ajoute ?}
%
For all these, we use rewrite systems on trees~\cite{BN98} and the
Buchberger algorithm for operads~\cite{DK10}. Finally, we continue the
investigation of the quotients of $\Mag$ by regarding the quotients of
$\Mag$ obtained by equating two trees of degree $3$. This leads to ten
quotient operads of $\Mag$. We provide for some of these combinatorial
realizations, mostly in terms of integer compositions.
\medbreak

This text is presented as follows. Section~\ref{sec:operad_Mag} contains
preliminaries about operads, binary trees, the magmatic operad, and
rewrite systems on binary trees. We also prove and recall some important
lemmas about rewrite systems on trees used thereafter. In
Section~\ref{sec:Magmatic_operads}, we study the set of all the
quotients of $\Mag$ seen as an operad in the category of vector spaces
and its lattice structure. Section~\ref{sec:CAs_d} is the heart of this
article and is devoted to the study of the comb associative operads
$\CAs{\gamma}$. Finally, Section~\ref{sec:MAg_3} presents our results
about the quotients of $\Mag$ obtained by equating two trees of
degree~$3$.
\medbreak

Some of the results presented here were announced in~\cite{CCG18}.
\medbreak

%%%%%%%%%%%%%%%%%%%%%%%%%%%%%%%%%%%%%%%%%%%%%%%%%%%%%%%%%%%%%%%%%%%%%%%%
\subsubsection*{General notations and conventions}
For any integers $a$ and $c$, $[a, c]$ denotes the set
$\{b \in \N : a \leq b \leq c\}$ and $[n]$, the set $[1, n]$. The
cardinality of a finite set $S$ is denoted by~$\# S$.
\medbreak
