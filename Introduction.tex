%%%%%%%%%%%%%%%%%%%%%%%%%%%%%%%%%%%%%%%%%%%%%%%%%%%%%%%%%%%%%%%%%%%%%%%%
%%%%%%%%%%%%%%%%%%%%%%%%%%%%%%%%%%%%%%%%%%%%%%%%%%%%%%%%%%%%%%%%%%%%%%%%
%%%%%%%%%%%%%%%%%%%%%%%%%%%%%%%%%%%%%%%%%%%%%%%%%%%%%%%%%%%%%%%%%%%%%%%%
\section*{Introduction}
Associative algebras are spaces endowed with a binary product $\Product$
satisfying among others the associativity law
\begin{math}
    (x_1 \Product x_2) \Product x_3 = x_1 \Product (x_2 \Product x_3)
\end{math}.
It is well-known that the associative algebras are representations of
the associative (nonsymmetric) operad $\As$. This operad can be seen as
the quotient of the magmatic operad $\Mag$ (the free operad of binary
trees on the binary generator~$\Product$) by the operad congruence
$\Congr$ satisfying
\begin{equation} \label{equ:congruence_as}
    \begin{tikzpicture}[xscale=.24,yscale=.24,Centering]
        \node(0)at(0.00,-3.33){};
        \node(2)at(2.00,-3.33){};
        \node(4)at(4.00,-1.67){};
        \node[NodeST](1)at(1.00,-1.67)
            {\begin{math}\Product\end{math}};
        \node[NodeST](3)at(3.00,0.00)
            {\begin{math}\Product\end{math}};
        \draw[Edge](0)--(1);
        \draw[Edge](1)--(3);
        \draw[Edge](2)--(1);
        \draw[Edge](4)--(3);
        \node(r)at(3.00,1.5){};
        \draw[Edge](r)--(3);
    \end{tikzpicture}
    \Congr
    \begin{tikzpicture}[xscale=.24,yscale=.24,Centering]
        \node(0)at(0.00,-1.67){};
        \node(2)at(2.00,-3.33){};
        \node(4)at(4.00,-3.33){};
        \node[NodeST](1)at(1.00,0.00)
                {\begin{math}\Product\end{math}};
        \node[NodeST](3)at(3.00,-1.67)
                {\begin{math}\Product\end{math}};
        \draw[Edge](0)--(1);
        \draw[Edge](2)--(3);
        \draw[Edge](3)--(1);
        \draw[Edge](4)--(3);
        \node(r)at(1.00,1.5){};
        \draw[Edge](r)--(1);
    \end{tikzpicture}\,.
\end{equation}
These two binary trees are the syntax trees of the expressions appearing
in the above associativity law.
\medbreak

In a more combinatorial context and regardless of the theory of operads,
the Tamari order is a partial order on the set of the binary trees
having a fixed number of internal nodes $d$. This order is generated by
the covering relation consisting in rewriting a tree $\Tfr$ into a tree
$\Tfr'$ by replacing a subtree of $\Tfr$ of the form of the left member
of~\eqref{equ:congruence_as} into a tree of the form of the right member
of~\eqref{equ:congruence_as}. This transformation is known in a computer
science context as the right rotation operation~\cite{Knu98} and
intervenes in algorithms involving binary search trees~\cite{AVL62}. The
partial order hence generated by the right rotation operation is known
as the Tamari order~\cite{Tam62} and has a lot of combinatorial and
algebraic properties (see for instance~\cite{HT72,Cha06}).
\medbreak

A first connection between the associative operad and the Tamari order
is based upon the fact that the orientation of~\eqref{equ:congruence_as}
from left to right provides a convergent orientation (a terminating and
confluent rewrite relation) of the congruence $\Congr$. The normal
forms of the rewrite relation induced by the rewrite rule obtained by
orienting~\eqref{equ:congruence_as} from left to right are right comb
binary trees and are hence in one-to-one correspondence with the
elements of~$\As$.
\medbreak

This work is intended to be a first strike in the study of the eventual
links between the Tamari order and some quotients of the operad $\Mag$.
In the long run, we would like to study quotients $\Mag/_{\Congr}$ of
$\Mag$ where $\Congr$ is an operad congruence generated by equivalence
classes of trees of a fixed degree. In particular, we would like to know
if $\Congr$ is generated by equivalence classes of trees forming
intervals of the Tamari order leads to algebraic properties for
$\Mag/_{\Congr}$ (like the description of orientations of its space of
relations, nice bases and Hilbert series).
\medbreak

We focus here on one of these quotients $\CAs{3}$ which is the operad
describing the category of the algebras equipped with a binary product
$\Product$ and subjected to the relation
\begin{math}
    ((x_1 \Product x_2) \Product x_3) \Product x_4
    =
    x_1 \Product (x_2 \Product (x_3 \Product x_4))
\end{math}.
This is a kind of associativity law in higher degree $d = 3$. This
operad is generated by an equivalence class of trees which is not an
interval for the Tamari order. As preliminary computer experiments show,
$\CAs{3}$ has oscillating first dimensions
(see~\eqref{equ:dimensions_CAs_3}), what is rather unusual among all
known operads. In this paper, we provide an orientation of the space of
relations of $\CAs{3}$. For this, we use rewrite systems on
trees~\cite{BN98} and the Buchberger algorithm for operads~\cite{DK10}.
\medbreak

This text is presented as follows. Section~\ref{sec:operad_Mag} contains
preliminaries about the magmatic operad and rewrite relations on trees.
In Section~\ref{sec:CAs_d}, we define the operad $\CAs{3}$ as a
particular case of a more general construction of generalizations
$\CAs{d}$, $d \geq 1$, of $\As$. Finally, Section~\ref{sec:CAs_3}
contains the orientation of the space of relations of
$\CAs{3}$ (Theorem~\ref{thm:convergent_rewrite_rule_CAs_3}). As
consequences, we obtain for $\CAs{3}$
the description of one of its Poincaré-Birkhoff-Witt bases
(Proposition~\ref{prop:PBW_basis_CAs_3}) and the description of
its Hilbert series (Proposition~\ref{prop:Hilbert_series_CAs_3}).
\medbreak

%%%%%%%%%%%%%%%%%%%%%%%%%%%%%%%%%%%%%%%%%%%%%%%%%%%%%%%%%%%%%%%%%%%%%%%%
\subsubsection*{General notations and conventions}
For any integers $a$ and $c$, $[a, c]$ denotes the set
$\{b \in \N : a \leq b \leq c\}$ and $[n]$, the set $[1, n]$. The
cardinality of a finite set $S$ is denoted by~$\# S$.
\medbreak
