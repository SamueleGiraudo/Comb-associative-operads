%%%%%%%%%%%%%%%%%%%%%%%%%%%%%%%%%%%%%%%%%%%%%%%%%%%%%%%%%%%%%%%%%%%%%%%%
%%%%%%%%%%%%%%%%%%%%%%%%%%%%%%%%%%%%%%%%%%%%%%%%%%%%%%%%%%%%%%%%%%%%%%%%
%%%%%%%%%%%%%%%%%%%%%%%%%%%%%%%%%%%%%%%%%%%%%%%%%%%%%%%%%%%%%%%%%%%%%%%%

\section{Magmatic operads}

\Todo{Write mini intro.}

\subsection{Lattice of magmatic quotients}

We denote by $\IMag$ the set of operadic ideals of $\Mag$, and
\begin{equation} \label{equ:definition_of_QMag}
  \QMag =\Big\{\Mag/ I\mid I\in\IMag\Big\},
  \end{equation}
the set of quotients of the magmatic operad. Given two elements $\Oca_1$
and $\Oca_2$ of $\QMag$, we denote by $\text{hom}\left(\Oca_1,\Oca_2\right)$
the set of operads morphisms from $\Oca_1$ to $\Oca_2$.

\begin{Proposition} \label{prop:endomorphisms_of_magmatic_operads}
  The set $\emph{hom}\left(\Oca_1,\Oca_2\right)$ admits a vector space
  structure. Moreover, its dimension is equal to $0$ or $1$ and every
  nonzero morphism is surjective.
\end{Proposition}

\begin{proof}
  The operad $\Oca_1$ is spanned by the binary generator~$\Product$, so
  that a morphism $\varphi:\Oca_1\to\Oca_2$ is uniquely determined by
  $\varphi(\Product)$. Moreover, $\varphi(\Product)$ has arity $2$ is
  $\Oca_2$. Hence, $\varphi(\Product)$ belongs to the line spanned by
  the binary generator of $\Oca_2$, that is there exists a scalar
  $\lambda\in\K$ such that $\varphi(\Product)=\lambda\Product$. If there
  exists such a $\lambda$ different from zero, then for every nonzero
  scalar $\mu$, we have a well-defined operads morphism
  $\psi:\Oca_1\to\Oca_2$ satisfying $\psi(\Product)=\mu\Product=\left(
  \mu/\lambda\right)\varphi(\Product)$. Hence,
  $\text{hom}\left(\Oca_1,\Oca_2\right)$ is either reduced to the zero
  morphism or it is in one-to-one correspondance with $\K$, which
  proves that $\text{hom}\left(\Oca_1,\Oca_2\right)$ is a vector space
  of dimension at most $1$. Moreover, if $\varphi$ is different from $0$,
  that is there is a nonzero scalar such that
  $\varphi(\Product)=\lambda\Product$, we have
  $\varphi(1/\lambda\Product)=\Product$, so that $\varphi$ is
  surjective.
\end{proof}

We introduce the binary relation $\preceq$ on $\QMag$ as follows: we
have $\Oca_2\preceq\Oca_1$ if the dimension of
$\text{hom}\left(\Oca_1,\Oca_2\right)$ is equal to $1$.

\begin{Proposition} \label{prop:order_relations_on_QMag_and_ideals}
  Let $\Oca_1=\Mag/I_1$ and $\Oca_2=\Mag/I_2$ be two elements of $\QMag$.
  We have $\Oca_2\preceq\Oca_1$ if and only if $I_1\subseteq I_2$.
\end{Proposition}

\begin{proof}
  Since $\text{hom}\left(\Oca_1,\Oca_2\right)$ is a vector space of
  dimension at most $1$, it contains a nonzero morphism if and only if
  the morphism
  $\overline{\varphi}:\Oca_1\to\Oca_2$ satisfying
  $\overline{\varphi}(\Product)=\Product$ is well-defined, which means
  that $\Oca_2\preceq\Oca_1$ is equivalent to this condition. Moreover,
  by the universal property of the quotient, $\overline{\varphi}$ is
  well-defined if and only if $I_1$ is included in the kernel of the
  morphism $\varphi:\Mag\to\Oca_2$ defined by
  $\varphi(\Product)=\Product$. This kernel is equal to $I_2$, so that
  $\overline{\varphi}$ is well-defined if and only if $I_1$ is included
  in $I_2$, which concludes the proof.
\end{proof}

Recall that a \Def{lattice} is a tuple $\left(E,\leq,\wedge,\vee\right)$,
where $\left(E,\leq\right)$ is a poset such that any two elements $e$
and $e'$ of $E$ admit a lower-bound $e\wedge e'$ and an upper-bound
$e\vee e'$. In particular, $\left(\IMag,\subseteq,\cap,+\right)$ is a
lattice, where $\cap$ and $+$ are the intersection and the sum of
operadic ideals, respectively.

Given two elements $\Oca_1=\Mag/I_1$ and $\Oca_2=\Mag/I_2$ of $\QMag$,
we let
\begin{equation} \label{equ:definition_of_lattice_operations_of_QMag}
  \Oca_1\wedge \Oca_2=\Mag/\left(I_1+I_2\right)\ \ \text{and}\ \
  \Oca_1\wedge \Oca_2=\Mag/\left(I_1\cap I_2\right).
\end{equation}
Explicitely, for every integer $n$, $\left(\Oca_1\wedge\Oca_2\right)(n)$ (respectively, $\left(\Oca_1\vee\Oca_2\right)(n)$) is the quotient of vector spaces $\Mag(n)/\left(I_1(n)+I_2(n)\right)$ (respectively, $\Mag(n)/\left(I_1(n)\cap I_2(n)\right)$), and the operadic compitions are induced by the operadic compositions of $\Mag$.

\begin{Theorem} \label{thm:lattice_structure_of_QMag}
  The tuple $\left(\QMag,\preceq,\wedge,\vee\right)$ is a lattice.
\end{Theorem}

\begin{proof}
  The map $\mathcal{Q}:\IMag\to\QMag$ defined by $\mathcal{Q}(I)=\Mag/I$
  is a bijection by definition of $\QMag$. Moreover, from
  Proposition~\ref{prop:order_relations_on_QMag_and_ideals},
  $\Oca_2\preceq\Oca_1$ is equivalent to $I_1\subseteq I_2$, so that
  $\left(\QMag,\preceq\right)$ is a poset and $\mathcal{Q}$ is a
  decreasing bijection. The tuple $\left(\IMag,\subseteq,\cap,+\right)$
  being a lattice, the decreasing bijection $\mathcal{Q}$ induces
  lattice operations on $\QMag$, precisely $\wedge$ and $\vee$ by
  definition.
\end{proof}


%%%%%%%%%%%%%%%%%%%%%%%%%%%%%%%%%%%%%%%%%%%%%%%%%%%%%%%%%%%%%%%%%%%%%%%%
%%%%%%%%%%%%%%%%%%%%%%%%%%%%%%%%%%%%%%%%%%%%%%%%%%%%%%%%%%%%%%%%%%%%%%%%
\subsection{Hilbert series and magmatic quotients}

Given an operad $\Oca$, we denote by $H_{\Oca}$ the \Def{Hiblert serie} of $\Oca$, that is
\begin{equation} \label{equ:definition_of_Hilbert_series}
  H_{\Oca}(t)=\sum_{n\in\N}\dim{\left(\Oca(n)\right)}t^n.
\end{equation}

In Theorem~\ref{thm:Grassmann_formula_for__Hilbert_series_of_QMag}, we show the Grassman formula for the lattice $\left(\QMag,\preceq,\wedge,\vee\right)$. For that, we need the following lemma:

\begin{Lemma} \label{lem:lattice_structure_and_coefficients_of_Hiblert_series}
  Let $\Oca_1$ and $\Oca_2$ be two elements of $\QMag$. For every integer $n$, we have
  \begin{equation} \label{equ:lattice_structure_and_coefficients_of_the_Hiblert_series}
    \dim{\left(\Oca_1\wedge\Oca_2(n)\right)}+\dim{\left(\Oca_1\vee\Oca_2(n)\right)}=\dim{\left(\Oca_1(n)\right)}+\dim{\left(\Oca_2(n)\right)}.    \end{equation}
\end{Lemma}

\begin{proof}
  Let $I_1$ and $I_2$ be the two operadic ideals of $\Mag$ such that $\Oca_1$ and $\Oca_2$ are equal to $\Mag/I_1$ and $\Mag/I_2$, respectively. For every integer $n$, we have
  \[\begin{split}
  &\dim{\left(\Oca_1\wedge\Oca_2(n)\right)}+\dim{\left(\Oca_1\vee\Oca_2(n)\right)}\\
  &=\dim{\Big(\Mag(n)/\left(I_1+I_2\right)(n)\Big)}+\dim{\Big(\Mag(n)/\left(I_1\cap I_2\right)(n)\Big)}\\
  &=\dim{\left(\Mag(n)\right)}-\dim{\left(I_1(n)+I_2(n)\right)}+\dim{\left(\Mag(n)\right)}-\dim{\left(I_1(n)\cap I_2(n)\right)}\\
  &=\dim{\left(\Mag(n)\right)}-\dim{\left(I_1(n)\right)}+\dim{\left(\Mag(n)\right)}-\dim{\left(I_2(n)\right)}\\
  &=\dim{\Big(\Mag(n)/I_1(n)\Big)}+\dim{\Big(\Mag(n)/I_2(n)\Big)}\\
  &=\dim{\left(\Oca_1(n)\right)}+\dim{\left(\Oca_2(n)\right)}.
  \end{split}\]
  The third equality is due to the Grassmann formula applied to the subspaces $I_1(n)$ and $I_2(n)$ of $\Mag(n)$.
\end{proof}

\begin{Theorem} \label{thm:Grassmann_formula_for__Hilbert_series_of_QMag}
  Let $\Oca_1$ and $\Oca_2$ be two elements of $\QMag$. We have
  \begin{equation} \label{equ:lattice_structure_and_Hiblert_series}
    H_{\Oca_1\wedge\Oca_2}(t)+H_{\Oca_1\vee\Oca_2}(t)=H_{\Oca_1}(t)+H_{\Oca_2}(t).
    \end{equation}
  \end{Theorem}

\begin{proof}
  We have
  \[\begin{split}
  H_{\Oca_1\wedge\Oca_2}(t)+H_{\Oca_1\vee\Oca_2}(t)&=\sum_{n\in\N}\dim{\left(\Oca_1\wedge\Oca_2(n)\right)}t^n+\sum_{n\in\N}\dim{\left(\Oca_1\vee\Oca_2(n)\right)}t^n\\
  &=\sum_{n\in\N}\Big(\dim{\left(\Oca_1\wedge\Oca_2(n)\right)}+\dim{\left(\Oca_1\vee\Oca_2(n)\right)}\Big)t^n\\
  &=\sum_{n\in\N}\Big(\dim{\left(\Oca_1(n)\right)}+\dim{\left(\Oca_2(n)\right)}\Big)t^n\\
  &=\sum_{n\in\N}\dim{\left(\Oca_1(n)\right)}t^n+\sum_{n\in\N}\dim{\left(\Oca_2(n)\right)}t^n\\
  &=H_{\Oca_1}(t)+H_{\Oca_2}(t).
  \end{split}\]
\end{proof}
The third equality is due to Lemma~\ref{lem:lattice_structure_and_coefficients_of_Hiblert_series}.
% Hilbert series, enumeration, a little of rewrite systems, etc.

