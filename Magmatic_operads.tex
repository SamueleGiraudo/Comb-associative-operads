%%%%%%%%%%%%%%%%%%%%%%%%%%%%%%%%%%%%%%%%%%%%%%%%%%%%%%%%%%%%%%%%%%%%%%%%
%%%%%%%%%%%%%%%%%%%%%%%%%%%%%%%%%%%%%%%%%%%%%%%%%%%%%%%%%%%%%%%%%%%%%%%%
%%%%%%%%%%%%%%%%%%%%%%%%%%%%%%%%%%%%%%%%%%%%%%%%%%%%%%%%%%%%%%%%%%%%%%%%
\section{Linear magmatic operads} \label{sec:Magmatic_operads}
In this section, we equip the set of quotients of the linear magmatic
operad with a lattice structure. We also show the Grassmann formula
analog for this lattice.
\medbreak

%%%%%%%%%%%%%%%%%%%%%%%%%%%%%%%%%%%%%%%%%%%%%%%%%%%%%%%%%%%%%%%%%%%%%%%%
%%%%%%%%%%%%%%%%%%%%%%%%%%%%%%%%%%%%%%%%%%%%%%%%%%%%%%%%%%%%%%%%%%%%%%%%
\subsection{Lattice of linear magmatic quotients}
The \Def{linear magmatic operad}, written $\KMag$, is the free linear
operad over one binary generator. By definition, for each arity $n$,
$\KMag(n)$ is the vector space with basis $\Mag(n)$, and the
compositions maps of $\KMag$ extend by linearity the ones of $\Mag$.
\medbreak

We denote by $\IMag$ the set of operadic ideals of $\KMag$, and we set
\begin{equation} \label{equ:definition_of_QMag}
    \QMag := \Big\{\KMag/_I :  I\in\IMag\Big\},
\end{equation}
as the set of the quotients of the linear magmatic operad. Given
$\KMag/_I \in \QMag$ and $x \in \KMag$, we denote by $[x]_I$ the
$I$-equivalence of $x$. Observe that $\KMag/_I$ is generated as an
operad by $[\Product]_I$. Moreover, given two  elements $\Oca_1$ and
$\Oca_2$ of $\QMag$, we denote by $\Hom\left(\Oca_1,\Oca_2\right)$ the
set of linear operad morphism from $\Oca_1$ to~$\Oca_2$.
\medbreak

\begin{Proposition} \label{prop:endomorphisms_of_magmatic_operads}
    For any $\Oca_1, \Oca_2 \in \QMag$, the set
    $\Hom\left(\Oca_1,\Oca_2\right)$ admits a vector space structure.
    Moreover, its dimension is equal to $0$ or $1$ and every nonzero
    morphism is surjective.
\end{Proposition}

\begin{proof}
    Let $I_1, I_2 \in \IMag$ such that $\Oca_1 = \KMag/_{I_1}$ and
    $\Oca_2 = \KMag/_{I_2}$. Since $\Oca_1$ is generated by the binary
    element $[\Product]_{I_1}$, a morphism $\varphi:\Oca_1\to\Oca_2$ is
    uniquely determined by $\varphi\left([\Product]_{I_1}\right)$.
    Moreover, $\varphi\left([\Product]_{I_1}\right)$ has arity $2$ in
    $\Oca_2$. Hence, $\varphi\left([\Product]_{I_1}\right)$ belongs to
    the line spanned by the binary generator of $\Oca_2$, that is there
    exists a scalar $\lambda\in\K$ such that
    \begin{math}
        \varphi\left([\Product]_{I_1}\right)
        = \lambda[\Product]_{I_2}.
    \end{math}
    If there exists such a $\lambda$ different from zero, then for every
    nonzero scalar $\mu$, we have a well-defined operad morphism
    $\psi:\Oca_1\to\Oca_2$ satisfying
    \begin{math}
        \psi([\Product]_{I_1}) =\mu[\Product]_{I_2}
        = \left(\mu/\lambda\right)\varphi([\Product]_{I_1}).
    \end{math}
    Hence, $\Hom\left(\Oca_1,\Oca_2\right)$ is either reduced to the
    zero morphism or it is in one-to-one correspondence with $\K$, which
    proves that $\Hom\left(\Oca_1,\Oca_2\right)$ is a vector space of
    dimension at most $1$. Moreover, if $\varphi$ is different from $0$,
    that is there is a nonzero scalar such that
    $\varphi([\Product]_{I_1}) =\lambda[\Product]_{I_2}$, we have
    $\varphi(1/\lambda[\Product]_{I_1}) =[\Product]_{I_2}$, so that
    $\varphi$ is surjective.
\end{proof}
\medbreak

We introduce the binary relation $\OrdQMag$ on $\QMag$ as follows: we
have $\Oca_2\OrdQMag\Oca_1$ if the dimension of
$\Hom\left(\Oca_1,\Oca_2\right)$ is equal to~$1$.
\medbreak

\begin{Proposition} \label{prop:order_relations_on_QMag_and_ideals}
    Let $\Oca_1=\KMag/_{I_1}$ and $\Oca_2=\KMag/_{I_2}$ be two operads
    of $\QMag$. We have $\Oca_2\OrdQMag\Oca_1$ if and only if
    $I_1\subseteq I_2$.
\end{Proposition}

\begin{proof}
  Since, by Proposition~\ref{prop:endomorphisms_of_magmatic_operads},
  $\Hom\left(\Oca_1,\Oca_2\right)$ is a vector space of
  dimension at most~$1$, it contains a nonzero morphism if and only if
  the morphism $\overline{\varphi}:\Oca_1\to\Oca_2$ satisfying
  $\overline{\varphi}([\Product]_{I_1})=[\Product]_{I_2}$ is well
  -defined, which means that $\Oca_2\OrdQMag\Oca_1$ is equivalent to this
  condition. Moreover, by the universal property of the quotient,
  $\overline{\varphi}$ is well-defined if and only if $I_1$ is included
  in the kernel of the morphism $\varphi:\KMag\to\Oca_2$ defined by
  $\varphi(\Product)=[\Product]_{I_1}$. This kernel is equal to
  $I_2$, so that $\overline{\varphi}$ is well-defined if and only if
  $I_1$ is included in $I_2$, which concludes the proof.
\end{proof}
\medbreak

Recall that a \Def{lattice} is a tuple
$\left(E,\leq,\wedge,\vee\right)$, where $\leq$ is a partial order
relation such that any two elements $e$ and $e'$ of $E$ admit a
lower-bound $e\wedge e'$ and an upper-bound $e\vee e'$. In particular,
$\left(\IMag,\subseteq,\cap,+\right)$ is a lattice, where $\cap$ and $+$
are the intersection and the sum of operadic ideals, respectively.
\medbreak

Given two operad $\Oca_1=\KMag/_{I_1}$ and $\Oca_2=\KMag/_{I_2}$ of
$\QMag$, let us define
\begin{equation} \label{equ:definition_of_lattice_operations_of_QMag}
    \Oca_1\InfQMag \Oca_2 := \KMag/_{I_1+I_2}
    \qquad \mbox{and} \qquad
    \Oca_1\SupQMag \Oca_2 := \KMag/_{I_1\cap I_2}.
\end{equation}
Explicitly, for every positive integer $n$,
$\left(\Oca_1\InfQMag\Oca_2\right)(n)$ (resp.
$\left(\Oca_1\SupQMag\Oca_2\right)(n)$) is the quotient of vector spaces
$\KMag(n)/_{I_1(n)+I_2(n)}$ (resp.
$\KMag(n)/_{I_1(n)\cap I_2(n)}$).
\medbreak

\begin{Theorem} \label{thm:lattice_structure_of_QMag}
  The tuple $\LatQMag$ is a lattice.
\end{Theorem}

\begin{proof}
  First, we observe that the map $\Oca:\IMag\to\QMag$ defined by
  $\Oca(I) :=\KMag/_I$ is a bijection: it is surjective by definition of
  $\QMag$ and it is injective since $\Oca(I_1)=\Oca(I_2)$ implies that
  the kernel of the natural projection $\KMag\to\Oca(I_1)=\Oca(I_2)$ is
  both equal to $I_1$ and $I_2$. Moreover, from
  Proposition~\ref{prop:order_relations_on_QMag_and_ideals},
  $\Oca_2\OrdQMag\Oca_1$ is equivalent to $I_1\subseteq I_2$, so that
  $\OrdQMag$ is a partial order relation on $\QMag$ and $\Oca$ is a
  decreasing bijection. The tuple $\left(\IMag,\subseteq,\cap,+\right)$
  being a lattice, the decreasing bijection $\Oca$ induces lattice
  operations on $\QMag$, precisely $\InfQMag$ and $\SupQMag$ by
  definition.
\end{proof}
\medbreak

%%%%%%%%%%%%%%%%%%%%%%%%%%%%%%%%%%%%%%%%%%%%%%%%%%%%%%%%%%%%%%%%%%%%%%%%
%%%%%%%%%%%%%%%%%%%%%%%%%%%%%%%%%%%%%%%%%%%%%%%%%%%%%%%%%%%%%%%%%%%%%%%%
\subsection{Hilbert series and linear magmatic quotients}
In order to show the Grassmann formula analog for $\LatQMag$, we need the
following lemma.
\medbreak

\begin{Lemma}
    \label{lem:lattice_structure_and_coefficients_of_Hiblert_series}
    Let $\Oca_1$ and $\Oca_2$ be two operads of $\QMag$. For every
    positive integer $n$, we have
    \begin{equation}
    \label{equ:lattice_structure_and_coefficients_of_the_Hiblert_series}
        \dim{\left(\left(\Oca_1\InfQMag\Oca_2\right)(n)\right)}+
        \dim{\left(\left(\Oca_1\SupQMag\Oca_2\right)(n)\right)}=
        \dim{\left(\Oca_1(n)\right)}+\dim{\left(\Oca_2(n)\right)}.
    \end{equation}
\end{Lemma}

\begin{proof}
    Let $I_1, I_2 \in \IMag$ such that $\Oca_1 = \KMag/_{I_1}$ and
    $\Oca_2 = \KMag/_{I_2}$. For every integer $n$, we have
    \begin{equation}\begin{split}
        &\dim{\left(\Oca_1\InfQMag\Oca_2(n)\right)}+
            \dim{\left(\Oca_1\SupQMag\Oca_2(n)\right)}\\
        &=\dim{\Big(\KMag(n)/_{\left(I_1+I_2\right)(n)}\Big)}+
            \dim{\Big(\KMag(n)/_{\left(I_1\cap I_2\right)(n)}\Big)}\\
        &=\dim{\left(\KMag(n)\right)}-\dim{\left(I_1(n)+I_2(n)\right)}
            +\dim{\left(\KMag(n)\right)}
            -\dim{\left(I_1(n)\cap I_2(n)\right)} \\
        &=\dim{\left(\KMag(n)\right)}-\dim{\left(I_1(n)\right)}
            + \dim{\left(\KMag(n)\right)}-\dim{\left(I_2(n)\right)}\\
        &=\dim{\Big(\KMag(n)/_{I_1(n)}\Big)}
            +\dim{\Big(\KMag(n)/_{I_2(n)}\Big)}\\
        &=\dim{\left(\Oca_1(n)\right)}+\dim{\left(\Oca_2(n)\right)}.
    \end{split}\end{equation}
    The third equality is due to the Grassmann formula applied to the
    subspaces $I_1(n)$ and $I_2(n)$ of $\KMag(n)$.
\end{proof}
\medbreak

The Grassmann formula analog for $\LatQMag$ is given by the following
statement.
\medbreak

\begin{Theorem}
    \label{thm:Grassmann_formula_for_Hilbert_series_of_QMag}
    Let $\Oca_1$ and $\Oca_2$ be two operads of $\QMag$. We have
    \begin{equation} \label{equ:lattice_structure_and_Hiblert_series}
        \HilbertSeries_{\Oca_1\InfQMag\Oca_2}(t)
        +\HilbertSeries_{\Oca_1\SupQMag\Oca_2}(t)
        =\HilbertSeries_{\Oca_1}(t)+\HilbertSeries_{\Oca_2}(t).
    \end{equation}
\end{Theorem}

\begin{proof}
    We have
    \begin{equation}\begin{split}
        \HilbertSeries_{\Oca_1\InfQMag\Oca_2}(t)
        +\HilbertSeries_{\Oca_1\SupQMag\Oca_2}(t)&
        =\sum_{n\in\N}\dim{\left(\left(\Oca_1\InfQMag\Oca_2\right)(n)
            \right)}t^n
            +\sum_{n\in\N}\dim{\left(\left(\Oca_1\SupQMag\Oca_2\right)
            (n)\right)}t^n\\
        &=\sum_{n\in\N}\Big(\dim{\left(\Oca_1\InfQMag\Oca_2(n)\right)}
            +\dim{\left(\left(\Oca_1\SupQMag\Oca_2\right)(n)
        \right)}\Big)t^n\\
        &=\sum_{n\in\N}\Big(\dim{\left(\Oca_1(n)\right)}
            +\dim{\left(\Oca_2(n)\right)}\Big)t^n\\
        &=\sum_{n\in\N}\dim{\left(\Oca_1(n)\right)}t^n+\sum_{n\in\N}
            \dim{\left(\Oca_2(n)\right)}t^n\\
        &=\HilbertSeries_{\Oca_1}(t)+\HilbertSeries_{\Oca_2}(t),
    \end{split}\end{equation}
    where the third equality is due to
    Lemma~\ref{lem:lattice_structure_and_coefficients_of_Hiblert_series}.
\end{proof}
\medbreak

% Hilbert series, enumeration, a little of rewrite systems, etc.
